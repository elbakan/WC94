\chapter{Saudi Arabia}
\chapterprecis{Reproduced from UPI Reports}
\newline
\newline
Coach: Jorge Solari
\begin{figure}[H]
\begin{tabular}{c c l l c c c}
P & & & Club & Age & Caps & Goals \\ \hline
G & 1 & Mohammed al-Deayea & Al Tai & 22 & 43 & 0 \\
G & 21 & Hussein al-Sadiq & Al Quadesia & 20 & 31 & 0 \\
G & 22 & Ibrahim al-Hilwa & Al Riyad & 21 & 5 & 0 \\ \hline
D & 2 & Abdullah al-Dosari & Al Etefag & 25 & 68 & 1 \\
D &  3 & Mohammed al-Khlawi & Al Ittihad & 23 & 4 & 0 \\
D &  4 & Abdullah Zebermawi & Al Ahli & 20 & 30 & 3 \\
D & 13 & Mohammed Abdel-Jawad & Al Ahli & 31 & 143 & 20 \\
D & 15 & Saleh al-Dawod & Al Shabab & 25 & 38 & 0 \\
D & 17 & Yasser al-Taifi &  Al Riyad & 22 & 0 & 0 \\ \hline
M & 5 & Ahmed Madani & Al Ittihad & 24 & 93 & 4 \\
M & 6 & Fuad Amin & Al Shabab & 21 & 62 & 15 \\
M & 8 & Fahad al-Bishi & Al Nasr & 28 & 76 & 39 \\
M & 14 & Khalid al-Muwallid & Al Ahli & 23 & 51 & 10 \\
M & 16 & Talal al-Jibreen & Al Riyad & 20 & 13 & 0 \\
M & 18 & Awad al-Anazi & Al Shabab & 25 & 6 & 0 \\
M & 19 & Hamza Saleh & Al Ahli & 26 & 27 & 0 \\ \hline
F & 7 & Fahd al-Ghashayan & Al Helal & 21 & 19 & 11 \\
F & 9 & Majed Mohammed & Al Nasr & 35 & 166 & 118 \\
F & 10 & Saeed Owairan & Al Shabab & 26 & 37 & 22 \\
F & 11 & Fahad Mehalel & Al Shabab & 23 & 39 & 7 \\
F & 12 & Sami al-Jaber & Al Helal & 21 & 40 & 10 \\
F & 20 & Hamza Falatah & Ohud & 21 & 45 & 10 \\ \hline
\end{tabular}
\end{figure}
\section{WC 1994 Qualification}
\begin{figure}[H]
\begin{tabular}{l l c l c l}
1993 & & & & & \\
May 1 & Macao & WCQ1 & Won & 6-0 & Kuala Lumpur \\
May 3 & Malaysia & WCQ1 & Drew & 1-1 & Kuala Lumpur \\
May 5 & Kuwait & WCQ1 & Drew & 0-0 & Kuala Lumpur \\
May 14 & Macao & WCQ1 & Won & 8-0 & Riyadh \\
May 16 & Malaysia & WCQ1 & Won & 3-0 & Riyadh \\
May 16 & Kuwait & WCQ1 & Won & 2-0 & Riyadh \\
Oct 15 & Japan & WCQ2 & Drew  & 0-0 & Doha \\
Oct 18 & North Korea & WCQ2 & Won  & 2-1 & Doha \\
Oct 22 & South Korea & WCQ2 & Drew  & 1-1 & Doha \\
Oct 24 & Iraq & WCQ2 & Drew  & 1-1 & Doha \\
Oct 28 & Iran & WCQ2 & Won  & 4-3 & Doha \\
\end{tabular}
\end{figure}
\section{1994 Friendlies -- not all dates known}
\begin{figure}[H]
\begin{tabular}{l l l c l}
& Riyadh & drew & 1-1 & China \\
& Riyadh & win & 1-0 & China \\
& Jeddah & lost & 0-2 & Colombia \\
& Jeddah & drew & 1-1 & Colombia \\
& Riyadh & lost & 0-1 & Chile \\
& Riyadh & drew & 2-2 & Chile \\
& Paris & lost & 0-1 & Poland \\
& Toulon & won & 2-0 & Iceland \\
& Athens & lost & 1-5 & Greece \\
& Cannes & lost & 0-1 & Bolivia \\
May 25 & New Jersey & Drew & 0-0 & USA \\
\end{tabular}
\end{figure}
\section{Key Players and Coach}
Coach:
JORGE RAUL SOLARI, Age: 51. Former Argentine international who played in the
1966 World Cup finals. Until recently was assistant coach of Argentine team, 
but took over the No. 1 post in Saudi Arabia after abrupt departure of his two
predecessors. Brazilian Jose Candido was accused of ``too much meddling'', while
Dutchman Leo Beenhakker met with stiff resistance when he tried to change the
team's playing style. Solari's greatest asset may be his ability to form the
team into a solid unit and offer a sense of calm to a young, potentially 
nervous team whose every move is being carefully watched, and analyzed by an 
excited public.

MOHAMMED AL-DEAYEA (Goalkeeper, Al Tai):
A spectacular athlete, some of al-Deayea's acrobatic saves have become 
legendary among Saudi soccer fans. Odds-on favourite to be first choice keeper, 
although he was controversially substituted midway through the crucial 
qualifier against Iraq.

MOHAMMED ABDUL JAWAD (Defender, Al Ahli):
Vice-Captain and first choice left back. His penchant for attacking from the 
back is distinctively Brazilian, the country where his wife comes from.

KHALED AL-MUWALLID (Midfield, Al Ahli):
A prodigiously gifted left-sided midfielder who is still haunted by a failed 
penalty attempt that would have defeated Saudi's political nemesis Iraq in the 
qualifying rounds. Could be an eye-catcher in the U.S.

FUAD ANWAR AMIN (MIdfield, Shabab):
One of the players his teamamtes look to for inspiration. Despite his youth, 
has a great deal of international experience. Played for the under-16 Saudi 
national team that won the World Cup in 1989.

FAHD AL-HARIFY AL-BISHI (Midfield, Nasr):
Saudi soccer fans heaved a collective sigh of relief recently when Al-Bishi was 
pardoned from a one-year national team suspension to play in the World Cup. Led 
his club team to the King's Cup Trophy in May and has been described as the 
best player in the squad.
