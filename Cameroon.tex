\chapter{Cameroon}
\chapterprecis{Reproduced from UPI Reports}
\newline
\newline
Coach: (French national team, 58 caps, former manager of France)
\begin{figure}[H]
\begin{tabular}{c c l l c c c}
P & & & Club & Age & Caps & Goals \\ \hline
G & 1 &  Joseph-Antoine Bell & St. Etienne (Fra) & 39 & 76 & 1 \\
G & 21 & Thomas Nkono & unattached & 38 & & \\
G & 22 & Jacques Songo'o & Metz (Fra) & 30 & 43 & 0 \\ \hline
D & 2 & Andre Kana-Biyik & Le Havre (Fra) & 28 & 58 & 18 \\
D & 3 & Rigobert Bahanag Song & TKC Yaounde & 18  & 0  & 0 \\
D & 4 & Samuel Ekeme Ndiba  & Canon Yaounde & 27  &   & \\
D & 5 & Victor Ndip Akem & Olympic Mvolye & 26  & 54  & 0 \\
D & 13 & Raymond Kalla & Canon Yaounde & 19  & 0  & 0 \\
D & 14 & Stephen Tataw [C]  & Olympic Mvolye & 30  & 69  & 8 \\
D & 15 & Hans Agbo & Olympic Mvolye & 26  & 40  & 0 \\
D & 17 & Marc Vivien Foe &  Canon Yaounde & 19  & 5  & 0 \\ \hline
M & 6 & Thomas Libih & unattached & 27  &   &  \\
M & 8 & Emile Mbouh-Mbouh & Nadi Qatar (Qat) & 27  & 69  & 27 \\
M & 11 & Emmanuel Kessack Mabouang  & Rio Ave (Por) & 25  & 40  & 0 \\
M & 12 & Paul-Serge Loga &  Prevoyance & 24  & 6  & 0 \\
M & 18 & Jean-Pierre Fiala Fiala  & Canon Yaounde & 25  & 8  & 2 \\ \hline
F & 7 & Francois Oman-Biyik  & Lens (Fra) & 27  & 72  & 48 \\
F & 9 & Roger Milla & Tonnerre Yaounde & 42  & 38  & 17 \\
F & 10 & Louis Mf{\'e}d{\'e} & Canon Yaounde & 33  &   &  \\
F & 16 & Alphonse Tchami & OB Odense (Den) & 23  & 29  & 16 \\
F & 19 & David Embe & Belenenses (Por) & 20  & 8  & 2 \\
F & 20 & Georges Mouy{\'e}m{\'e}-Elong  & Troyes (Fra) & 23  & 0  & 0 \\ \hline
\end{tabular}
\end{figure}
\section{WC 1994 Qualifying}
\begin{figure}[H]
\begin{tabular}{l l c l c l}
1992 & & & & & \\
Oct 18 & Swaziland & WCQ1 & Won & 5-0 & Yaounde \\
1993 & & & & & \\
Jan 10 & Zaire & WCQ1 & Won & 2-1 & Kinshasa \\
Jan 17 & Swaziland & WCQ1 & Drew & 0-0 & Mbabane \\
Mar 1 & Zaire & WCQ1 & Drew & 0-0 & Yaounde \\
Apr 18 & Guinea & WCQ2 & Won & 3-1 & Yaounde \\
Jul 4 & Zimbabwe & WCQ2 & Lost & 0-1 & Harare \\
Jul 18 & Guinea & WCQ2 & Won & 1-0 & Conakry \\
Oct 10 & Zimbabwe & WCQ2 & Won  & 3-1 & Yaounde \\
\end{tabular}
\end{figure}
\section{Preview}
It must be remembered that although everybody thinks of Roger Milla when they 
recall Cameroon in the 1990 World Cup, he was merely the most visible member of 
a very talented team, and much of that talent still survives and is augmented 
by some promising new blood.
In Francois Omam-Biyik, Cameroon has a striker capable of scoring against
anyone -- as he showed against Argentina on that glorious June day in 1990.
The 28-year-old, who plays for Lens in the French First Division, scored twice 
in the vital 3-1 win over Zimbabwe and has a good all-round game. He is likely 
to be partnered by promising newcomer Alphonse Tchami, who plays in Denmark for
Odense. Attacking midfielder David Embe, who plays for Portugal's Belenenses, 
is another new addition with plenty of potential. Most of the squad now plays 
in Europe with Jean-Claude Pagal (Martigues), Andre Kana-Biyik (Le Havre) and 
'keeper Joseph-Antoine Bell (St Etienne) all plying their trade in the French 
League. The 39-year-old keeper has starred for five French clubs and is liked 
for his showmanship as well as his solid performances. He missed the 1990 
finals after a run-in with manager Valeri Nepomniachi and will be relishing 
his chance this time. He is not guaranteed a place though, as long-time rival 
Thomas Nkono is making a late push -- much to Bell's annoyance. However, the
make or break area for Cameroon is its defense. Ill-discipline in the 1990 
quarterfinal gave England two penalties and the chance of a semi-final spot was
gone. Even that team was severely weakened by suspensions after a series of 
yellow cards that blighted its otherwise crowd-pleasing performances.
Rejuvenated captain Stephen Tataw, full-back Victor Ndip-Aken and Jean-Claude
Pagal are all veterans of the 1990 finals but will need to show more self-
control than they did four years ago. In comparison with fellow-African 
qualifier Nigeria, Cameroon's build-up has been chaotic -- and that's a kind 
description. While the Nigerians fly to Holland for a month of intensive 
preparation, Cameroon is in a mess. An April match against Zambia was postponed 
at the last minute -- with no official reason given and the March 16 game 
against Egypt was treated as little more than a training run. ``I am completely 
disappointed with the performance of the team,'' said Alphonse Tchami.

Money -- or the lack of it -- is the root of the problem, and the bottom line
is that however well the players perform and whatever income is generated, the
subsequent profits disappear into a bureaucratic black hole. FIFA gave the 
Cameroon FA \$200,000 to help its preparation but little of that seems to be 
finding its way to the sharp end. Recently installed coach Henri Michel of 
France is a pawn in the game being played out by the warring factions of the 
national federation, and all the time national President Paul Biya is standing 
by with his casting vote.

For any other country, all the behind-the-scenes problems would be a distraction, 
but the Cameroonians seem to thrive on them. So, when the``Indomitable Lions" 
pull on their green shirts and run on to the pitch to face Sweden on June 19, 
their attention will be concentrated only on soccer -- just as it was four 
years ago against Argentina. And look what happened then.
\section{Pre World Cup Friendlies 1994}
\begin{figure}[H]
\begin{tabular}{l l c l}
Mar 16 & Egypt & A & 0-0 \\
May 5 & South Korea & A & 2-2 \\
May 9 & Greece & A & 3-0 \\
May 11 & Bolivia & Athens & 1-1 \\
\end{tabular}
\end{figure}
\section{Coach and Key Players for Cameroon}
Coach HENRI MICHEL, Age: 46. As a player he captained France and was capped 58
times in a distinguished career. Went on to coach the French national team in
the 1986 World Cup finals and to gold in the 1984 Olympics. Took over the
Cameroon side from Leonard Nseke in Dec 1993 after qualification for the 1994
finals was clinched.

JOSEPH-ANTOINE BELL (Goalkeeper, St Etienne(FRA)) 39 years old, 66 Caps:
Flamboyant veteran who is not afraid to speak his mind. Took over the No. 1 
jersey for Cameroon's last four World Cup qualifiers, allowing just three goals 
and should be first choice in the United States. Has played with 10 clubs in 
his career, playing predominantly in Africa until 1985 and in France since 1986. 
Earned his first cap in 1975 and has been named Cameroon's player of the year 
on three occasions. Has the distinction of being one of two professional players 
(along with Dutchman Ruud Gullit) serving on FIFA's ``Task Force 2000,'' a 
working party created to look at ways of improving the game.

STEPHEN TATAW (Defender, Olympic Mvoyle) 31 years old, 54 caps:
Long-serving sweeper and captain. Lost his place last year after a series of 
poor domestic performances but came back for the March match with Egypt and 
adds invaluable experience.

ANDRE KANA-BIYIK (Midfield, Le Havre(FRA)) 28 years old, 59 caps:
Has 18 international goals to his credit but is making slow progress this year
in recovering from injury.

EMILE MBOUGH-MBOUGH (Midfield, Nadi Qatar(QAT)) 28 years old, 55 caps:
Another long-term absentee who returned to the squad when World Cup 
qualification began to look likely. One of the stars of 1990, full of ideas but
sometimes loses concentration.

FRANCOIS OMAM-BIYIK (Forward, Lens(FRA)) 28 years old, 72 caps:
Etched his name in African soccer history with the goal that beat Argentina in 
the opening game of Italia '90. Also scored twice in the 1994 qualifying 
clincher against Zimbabwe and has 48 international goals. Since moving to Lens
in Sept. 1992 has developed into a top-class performer and is probably his 
country's best player.

ALPHONSE TCHAMI (Forward, Odense(DEN)) 23 years old, 7 caps:
One of the new generation who will add a spark to the team. Scored on his debut 
against Swaziland and was one of the stars of the qualifying campaign.

ROGER MILLA (Forward, Tonnerre Yaounde) 42 years old, 94 caps:
The most famous name in African soccer. Starred as 38-year-old super-sub in 
Italia '90 after coming out of retirement. After a turbulent four years he 
looks all set to do the same again this time -- 19 years after making his 
international debut -- thanks again to the personal intervention of Cameroon 
President Paul Biya.

