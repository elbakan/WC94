\chapter{Sweden}
\chapterprecis{Marek Repinski}
\newline
\newline
Coach: Tommy Svensson
\begin{figure}[H]
\begin{tabular}{c c l l c c c}
P & & & Club & Age & Caps & Goals \\ \hline
G & 1 & Thomas Ravelli & IFK Gothenburg & 35 & 107 & 0 \\
G & 12 & Lars Eriksson & IFK Norrkoping & 29 & 14 & 0 \\
G & 22 & Magnus Hedman & Solna AIK & 21 & 15 & 0 \\ \hline
D & 2 & Roland Nilsson & Sheffield Wednesday (Eng) & 31 & 59 & 1 \\
D & 3 & Patrik Andersson & Borussia M'ngladbach (Ger) & 23 & 20 & 0 \\
D & 4 & Joachim Bj{\"o}rklund & IFK Gothenburg & 23 & 19 & 0 \\
D & 5 & Roger Ljung & Galatasaray (Tur) & 28 & 45 & 2 \\
D &13 & Mikael Nilsson & IFK Gothenburg & 26 & 12 & 0 \\
D &14 & Pontus Kamark & IFK Gothenburg & 25 & 9 & 0 \\
D &15 & Jan Eriksson & FC Kaiserslauten (Ger) & 27 & 32 & 4 \\ \hline
M & 6 & Stefan Schwarz & Benfica (Por) & 25 & 27 & 5 \\
M & 8 & Klas Ingesson & PSV Eindhoven (Neth) & 26 & 40 & 10 \\
M & 9 & Jonas Thern & Napoli (Ita) & 27 & 44 & 6 \\
M & 16 & Anders Limpar & Everton (Eng) & 29 & 48 & 5 \\
M & 17 & Stefan Rehn & IFK Gothenburg & 28 & 38 & 6 \\
M & 18 & Hakan Mild & IFK Gothenburg & 23 & 11 & 3 \\
M & 21 & Jesper Blomqvist & IFK Gothenburg & 20 & 5 & 0 \\ \hline
F & 7 & Henrik Larsson & Feyenoord (Neth) & 23 & 6 & 4 \\
F & 10 & Martin Dahlin & Borussia M'ngladbach (Ger) & 25 & 26 & 16 \\
F & 11 & Tomas Brolin & Parma (Ita) & 26 & 28 & 16 \\
F & 19 & Kennet Andersson & Lille (Fra) & 27 & 22 & 11 \\
F & 20 & Magnus Erlingmark & IFK Gothenburg & 26 & 23 & 1 \\ \hline
\end{tabular}
\end{figure}
\section{History}
When you mention Sweden and soccer in the same sentence most people think of 
Sweden's fiasco in Italy '90 with three 1-2 loses (Brazil, Scotland and COSTA 
RICA!). It became a national disaster and many have tried to explain how a very
good team (probably the best team Sweden ever had) could fail so completely.
However Sweden is well out of it's ``black period'' (75-84) and has after all,
always been one of the better nations in WC-history:
\begin{figure}[H]
\begin{tabular}{c l}
1934 & lost in Quarterfinals \\
1938 & 4th place overall \\
1950 & 3rd place in Final Group (Bronze!!) \\
1958 & Runner-up (Silver!!!) \\
1970 & 3rd place in their Group \\
1974 & 5-6th place overall \\
1978 & Last in their Group \\
1986 & Last in their Group \\
\end{tabular}
\end{figure}
In the legendary team of 1958 were players later prised away by Italian clubs.
These included Kurre Hamrin, Gunnar Gren, Nisse Liedholm and Lennart ``Nack'' 
Skoglund.

In the late 60s and early 70s Swedish soccer changed tactic. They stopped
building the national team around individual players (like in '58), and team 
effort, good organization and loyalty became landmarks for Swedish soccer and 
this philosophy remains today. However, two strikers, Ralf Edstrom (70s) and 
Tobjorn Nilsson (80s) reached ``star''-status.

\section{WC qualifying}
\begin{figure}[H]
\begin{tabular}{r c l l}
Finland & 0-1 (0-0) & Sweden & Klas Ingesson 77p' \\
Israel & 1-3 (1-1) & Sweden & Limpar 37' Dahlin 58' Ingesson 75' \\
Sweden & 2-0 (0-0) & Bulgaria & Dahlin 63' Pettersson 82' \\
France & 2-1 (1-1) & Sweden & Cantona 40p' 81' Dahline 14' \\
Sweden & 1-0 (0-0) & Austria & Eriksson 50' \\
Sweden & 5-0 (2-0) & Israel & Brolin 14' 41' 65' Zetterberg 55' Landberg 89' \\
Sweden & 1-1 (0-0) & France & Sauzee 77' Dahlin 89' \\
Bulgaria & 1-1  (1-1) & Sweden & Stoichkov 21p' Dahlin 26' \\
Sweden & 3-2 (3-1) & Finland & Dahlin 27' 45' Larsson 40' Suominen '15 Litmanen 60' \\
Austria & 1-1 (0-0) & Sweden &  Mild 68' Herzog 71' \\
\end{tabular}
\end{figure}
\begin{figure}[H]
\begin{tabular} {l c c c c r c l c}
\textbf{Sweden} & 10 & 6 & 3 & 1 & 19 & - & 8 & 15 \\
\textbf{Bulgaria} & 10 & 6 & 2 & 2 & 19 & - & 10 & 14 \\
France & 10 & 6 & 1 & 3 & 17 & - & 10 & 13 \\
Austria & 10 & 3 & 2 & 5 & 15 & - & 16 & 8 \\
Finland & 10 & 2 & 1 & 7 & 9 & - & 18 & 5 \\
Israel & 10 & 1 & 3 & 6 & 10 & - & 27 & 5 \\
\end{tabular}
\end{figure}
France was the big favorite in this group but lost their last two games in 
Paris (Israel 2-3 and Bulgaria 1-2 (two real upsets)) so they blew it. Sweden 
on the other hand had a very comfortable ride all the way. Two ``safe'' draws in 
the most crucial games, home against France and away to Bulgaria. The defeat 
against France in Paris was due to a dubious penalty decision (a gift for 
France from the referee).

It's hard to estimate the chances of Sweden in the final playoffs. Sweden is 
\textbf{not} a major contender for the trophy. In fact, it would be regarded as a
very successful tournament for Sweden if they could reach the quarter finals. I 
believe that Sweden are certainly among the 16 best soccer nations in the 
tournament, but due to strange and outrageous drawing rules) Sweden has to play 
in the most difficult group, in my opinion with Brazil, Russia and Cameroon. 
Therefore, I'll only give them a 2/3 chance of reaching the second round, I 
remember Italia '90 too well.
\section{Friendlies}
\begin{figure}[H]
\begin{tabular}{l l l c l}
1994 & & & & \\
Feb 18 & Colombia & Drew & 0-0 & Miami \\
Feb 20 & USA & Won & 3-1 & Miami \\
Feb 25 & Mexico & Lost & 1-2 & A \\
Apr 20 & Wales & Won & 2-0 & A \\
May 5 & Nigeria & Won & 3-1 & H \\
May 26 & Denmark & Lost & 0-1 & A \\
June 5 & Norway & Won & 2-0 & H \\
\end{tabular}
\end{figure}
\section{Key Players}
Like I said before ``team effort'' is the Swedish philosophy. The coach Tommy 
Svensson's task is to make sure everybody knows their role and works well 
together. Still there are some players to watch out for:

MARTIN DAHLIN (Forward, Borussia Moenchengladbach(GER)):
A 25 year old striker originally from the south of Sweden signed by Borussia
from Malmo FF. He is \textbf{the} goal scorer in the team, not always an attractive
player but very effective. He was the top scorer in qualifying with 7 goals.
He is easy to recognize because he will probably be the only black player in 
the team (his father is a West Indian and mother Swedish).
He can shoot with both feet, is good with his head and knows exactly where the
goal is. Unfortunately he has a negative side. His temperament is questionable
and he uses his elbows and studs more than he should.

JONAS THERN (Midfield, Napoli(ITA)):
Also signed from Malmo FF, he is the dream of every coach. He is the heart of 
the Swedish engine, and brilliant both in defence and attack. He always gives
everything when playing for the national side. A great dribbler and passer of 
the ball, he seems to have a sixth sense of where Dahlin is, stemming from their
days together at Malmo.

TOMAS BROLIN (midfield-forward, Parma(ITA)):
The only good Swede to play well in Italia '90. Nowadays he is a ``star'' player 
for Parma. He is ranked among the 3 best foreigners in Italy and that says it 
all. Recently, he has been playing in an attacking midfield role instead of as
an out and out forward, but he still scores regularly. Unfortunately, he likes
fast cars, women and parties which begs the question, is he the Swedish George
Best?
