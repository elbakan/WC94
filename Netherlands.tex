\chapter{Netherlands}
\chapterprecis{Karel Stokkermans}
\newline
\newline
Coach: Dick Advocaat
\begin{figure}[H]
\begin{tabular}{c c l l c c c}
P & & & Club & Age \\ \hline
G & 1 & Ed de Goey & Feyenoord & 28 & 12 & 0 \\
G & 13 & Edwin van der Sar & Ajax & 24 & 0 & 0 \\
G & 22 & Theo Snelders & Aberdeen (Sco) & 31 & 1 & 0 \\ \hline
D & 2 & Frank de Boer & Ajax & 24 & 23 & 1 \\
D & 4 & Ronald Koeman (c) & Barcelona (Spa) & 31 & 70 & 13 \\
D & 14 & Ulrich van Gobbel & Feyenoord & 23 & 4 & 0 \\
D & 18 & Stan Valckx & Sporting Lisbon (Por) & 31 & 5 & 0 \\
D & 21 & John de Wolf & Feyenoord & 32 & 5 & 2 \\ \hline
M & 3 & Frank Rijkaard & Ajax & 32 & 67 & 7 \\
M & 5 & Rob Witschge & Feyenoord & 28 & 19 & 3 \\
M & 6 & Jan Wouters & PSV Eindhoven & 34 & 65 & 4\\
M & 8 & Wim Jonk & Inter Milan (Ita) & 28 & 13 & 3 \\
M & 12 & John Bosman & Anderlecht (Bel) & 29 & 25 & 11 \\
M & 15 & Danny Blind & Ajax & 32 & 24 & 1 \\
M & 16 & Arthur Numan & PSV Eindhoven & 25 & 4 & 0 \\
M & 20 & Aron Winter & Lazio (Ita) & 27 & 37 & 2 \\ \hline
F & 7 & Marc Overmars & Ajax & 21 & 11 & 1 \\
F & 9 & Ronald de Boer & Ajax & 24 & 7 & 3 \\
F & 10 & Dennis Bergkamp & Inter Milan (Ita) & 25 & 29 & 16 \\
F & 11 & Brian Roy & Nottingham Forest (Eng) & 24 & 19 & 4 \\
F & 17 & Gaston Taument & Feyenoord & 24 & 5 & 0 \\
F & 19 & Peter van Vossen & Ajax & 26 & 8 & 7 \\ \hline
\end{tabular}
\end{figure}
\section{History}
Thanks to the proximity to the British Isles, the Netherlands were one of the 
first nations on the continent to start playing football. The pioneer was Pim 
Mulier, who founded the first football club, H.F.C. from Haarlem, now called 
K.H.F.C. (Royal H.F.C.). In the years of innocence, just after the turn of the 
century, the Dutch side was one of the best on the continent, as three Olympic 
bronze medals prove. After the mid-twenties however, the football centre of 
Europe shifted to the South and East, and Dutch football sunk into 
insignificance, despite the occasional drubbing of arch rivals Belgium in the 
annual meetings and the appearance of the odd gifted player. Participation in 
two early World Cups (1934 and 1938) was restricted to one appearance, the 
defeat against Czechoslovakia after extra time in 1938 (0-3) a creditable 
performance.

Things started changing in the fifties. Up until then, amateurism was staunchly
adhered to in the Netherlands and the Dutch professionals playing abroad,like 
Wilkes, De Munck, and Rijvers, were not allowed to play for the Dutch national
side. A charity match for the victims of the 1953 flood between a chance team 
of Dutch professionals abroad and France (where most Dutch pros were playing) 
changed all that. The Dutch XI won 2-1 and the public saw how much better these
professionals had become by playing abroad. Pressure became too much for the 
Dutch FA, the KNVB. A rival professional organisation was set up in the summer
of 1954, and soon the two merged. A nationwide first division, the ``eredivisie'',
was formed in 1956 and slowly things took a turn for the better.

The sixties still saw an embarrassing exit at the hands of Luxembourg in the
European Nations Cup, but the club sides started improving rapidly. Feyenoord
was the first to win the European Cup, but it was Ajax who were to stamp their
mark on the international football scene with their \textit{totaalvoetbal}. The story
of the Dutch failure to win a World Cup in the seventies, twice losing to the 
home country in the final, needs no retelling. Personally, I found the failure
to win the European Championship in 1976 even more galling. After convincingly 
winning a qualification group including Poland and Italy (a group of death if 
ever there was one) and destroying Belgium 5-0 and 2-1 in the quarterfinal, an 
ill-tempered semifinal was lost to Czechoslovakia after extra time.  It's one 
of the (many) mysteries of the game that the side built around Cruyff, Van 
Hanegem, Neeskens, and Krol have no Cup to show for their efforts.

The successors of that golden generation faced an impossible task. After an 
indifferent performance at the EC 1980, the Dutch team narrowly missed 
qualification for a major tournament three times in succession, Spain's 12-1
victory over Malta in 1983 and Gruen's late away goal in 1985 particularly 
traumatic, but to be honest, the quality just wasn't there. Slowly however, a 
new, extraordinary generation had ripened under the guidance of Kees Rijvers, 
and Rinus Michels, the ``generaal'', picked the spoils in 1988.  Van Basten's 
winner in the Hamburg semifinal sent a whole nation into orgasm and sealed an 
uncanny revenge for the final defeat in 1974.  Uncanny in that the second half
of that semi closely mirrored the first half of that indigestible loss in 
Munich. Victory in the final over the USSR was a mere extra, embellished by Van
Basten's beautiful volley.

Hopes were high for more Cups to come, as the core of the team was still young.
And was \textit{our} fluent, attacking style not the way for football to go?  But after
comfortable qualifying campaigns, Oranje failed at the final tournaments.  In 
Italy, an atrocious group performance meant a second round match against the 
Germans.  In one of the few decent matches at Italia Novanta, our neighbours 
proved to be the clearly better side.  Revenge for that defeat was achieved in 
Sweden '92, but the Dutch players forgot that a semifinal had to be played 
before they were allowed to take on the Germans again.  There are but few days
that the Dane in my office fails to remind me.
\section{WC 1994 Qualifying}
Oslo 23 Sep 92        Norway      2-1 Netherlands
        9.Rekdal (p) 1-0 10.Bergkamp 1-1 78.Soerloth 2-1
     Menzo, Blind, R.Koeman, Silooy, F.de Boer, Rob Witschge, Rijkaard,
     Wouters (82' Kieft), Van 't Schip (80' Taument), Bergkamp, Van Basten.
The match that should have been a warning for things to come but was treated
 as a fluke due to atrocious conditions.

Rotterdam    14 Oct 92        Netherlands 2-2 Poland
        19.Kozminski 0-1 21.Kowalczyk 0-2 42.Van Vossen 1-2 48.Van Vossen 2-2
    Menzo, Van Aerle, R.Koeman, Rijkaard (80' Fraeser), Rob Witschge, Jonk,
     Bergkamp, Wouters, Numan (39' Vanenburg), Van Vossen, Van Basten.
Menzo is responsible for both early goals and is fortunately dropped from the
 team.  Van Vossen saves the day while Bergkamp and Van Basten fight a close
 contest in squandering chances.

Istanbul     16 Dec 92        Turkey      1-3 Netherlands
        57.Van Vossen 0-1 59.Gullit 0-2 60.Feyyaz 1-2 87.Van Vossen 1-3
     De Goey, Silooy, R.Koeman, Rijkaard, Wouters, Jonk (66' F.de Boer),
     Winter (78' Numan), Rob Witschge, Gullit, Van Vossen, Viscaal.
In the absence of Van Basten, Van Vossen again strikes twice in a match that
 had to be won.  Gullit proves his worth after a long free kick from Koeman.

Utrecht      24 Feb 93        Netherlands 3-1 Turkey
        5.Overmars 1-0 37.Feyyaz (p) 1-1 39.Witschge 2-1 52.Witschge 3-1
     De Goey, De Kock, R.Koeman, Silooy, Wouters (73' Winter), Jonk,
     Rob Witschge, Gullit, Bergkamp, Van Vossen (46' F.de Boer), Overmars.
A match only memorable for the selection of local crowd favourite De Kock.
    
Utrecht      24 Mar 93        Netherlands 6-0 San Marino
        3.Van den Brom 1-0 29.Canti (og) 2-0 53.De Wolf 3-0
        67.R.de Boer (p) 4-0 78.Van Vossen 5-0 83.De Wolf 6-0
     De Goey, De Wolf, F.de Boer, Winter, Wouters, Van den Brom, Rob Witschge,
     Overmars, Meijer, Eijkelkamp (46' R.de Boer), Blinker (68' Van Vossen).
A missed opportunity to improve goal difference against the group's minnows.

London       28 Apr 93        England     2-2 Netherlands
        2.Barnes 1-0 24.Platt 2-0 34.Bergkamp 2-1 85.Van Vossen (p) 2-2
     De Goey, Rijkaard, Blind, F.de Boer, Winter, Wouters, Bergkamp,
     Rob Witschge, Gullit (71' Van Vossen), Bosman (46' De Wolf), Overmars.
In retrospect the match that decided second place.  Oranje were outplayed
 for half an hour, and only after Wouters' elbow had knocked out Gazza, the
 tide slowly turned.  Gullit was substituted and refused to play any more
 under coach Advocaat.  Van Vossen once again proved his nerves and converted
 a late penalty that Overmars' speed had brought.  A lucky point.

Rotterdam     9 Jun 93        Netherlands 0-0 Norway
     De Goey, Van Gobbel (80' Winter), R.Koeman, Rijkaard, F.de Boer, Jonk,
     Wouters, Overmars, Bergkamp, Bosman (46' Van Vossen), Blinker.
The match that more or less sealed the Norway's ticket to the States.  They
 didn't do it prettily, but the Dutch side lacked the imagination to beat
 Drillo's efficient though unappealing tactics.

Bologna      22 Sep 93        San Marino  0-7 Netherlands
        1.Bosman 0-1 22.Jonk 0-2 44.Jonk 0-3 51.R.de Boer 0-4 66.Bosman 0-5
        77.Bosman 0-6 80.R.Koeman (p) 0-7
      De Goey, R.Koeman, F.de Boer, Rijkaard, Jonk, Wouters, Overmars, 
      Bergkamp, Bosman, Kieft (46' R.de Boer), Roy.
A match in which Oranje should have reached double figures for the first time
 in its history.  Kieft missed several sitters in probably his last match for
 the national side.  A sad end to the international career of a tremendous
 striker always in the shadow of Van Basten.

Rotterdam    13 Oct 93        Netherlands 2-0 England
        61.R.Koeman 1-0 68.Bergkamp 2-0
      De Goey, De Wolf, R.Koeman, F.de Boer, Rijkaard, Wouters, Bergkamp,
      E.Koeman, Overmars (73' Winter), R.de Boer (88' Van Gobbel), Roy.
D-Day.  A match marred by some controversial refereeing decisions, but in the
 end the better side won.  A good comeback into the team of Ronald's older
 brother Erwin.  Apart from that, enough has been written about this match in 
 r.s.s.

Poznan       17 Nov 93        Poland      1-3 Netherlands
        10.Bergkamp 0-1 14.Lesniak 1-1 56.Bergkamp 1-2 88.R.de Boer 1-3
      De Goey, Van Gobbel, R.Koeman, F.de Boer, Winter, Wouters, Bergkamp,
      E.Koeman, Overmars, R.de Boer, Roy.  
In front of a 90\% Dutch crowd, the required last point is never in danger
 against a Polish side disappointing throughout the qualification campaign.
\begin{figure}[H]
\begin{tabular}{l c c c c c c c}
European Group 2 & GP & W & D & L & GF & GA & Pts \\ \hline
\textbf{Norway} & 10 & 7 & 2 & 1 & 25 & 5 & 16 \\
\textbf{Netherlands} & 10 & 6 & 3 & 1 & 29 & 9 & 15 \\
England & 10 & 5 & 3 & 2 & 26 & 9 & 13 \\
Poland & 10 & 3 & 2 & 5 & 10 & 15 & 8 \\
Turkey & 10 & 3 & 1 & 6 & 11 & 19 & 7 \\
San Marino & 10 & 0 & 1 & 9 & 2 & 46 & 1 \\ \hline
\end{tabular}
\end{figure}
\section{1994 Preparation}
January 19, 1994, Tunis.

Tunis        19 Jan 94        Tunisia     2-2 Netherlands
12' Rouissi 1-0  31' Rijkaard 1-1  51' Hamrouni 2-1  59' R.Koeman 2-2

Line-up:  De Goey, Blind (32' Winter), R.Koeman, F.de Boer, Rijkaard,
 Wouters (46' Jonk), Bergkamp (82' E.Koeman), Numan,
 Overmars (59' Gillhaus), R.de Boer, Blinker.

Glasgow      23 Mar 94        Scotland    0-1 Netherlands 
23' Roy 0-1

Line-up:  De Goey, Van Gobbel, Blind, F.De Boer, Witschge, Jonk, Rijkaard,
 Bosman (45' Winter), Taument (45' Overmars), Bergkamp (45' Gillhaus), 
 Roy.
  
Tilburg      20 Apr 94        Netherlands 0-1 Ireland   
55' Coyne 0-1

Line-up:  De Goey, Valckx, R.Koeman (46' De Wolf), F.de Boer, Rijkaard, 
 Bergkamp (46' Taument), Jonk (46' Winter), Davids, Overmars, 
 R.de Boer, Roy.

Utrecht      27 May 94        Netherlands 3-1 Scotland 
16' Roy 1-0  60' Van Vossen 2-0  68' Irvine (og) 3-0  81' Shearer 3-1

Line-up:  De Goey, Jonk, Valckx, F.de Boer, Wouters, Winter, Witschge, 
 R.de Boer (Numan), Gullit (Van Vossen), Roy (Taument), Overmars.

Eindhoven    1 Jun 94Netherlands 7-1 Hungary  
9' Illes(p) 0-1 13' Bergkamp 1-1  18' Roy 2-1  23' Koeman(p) 3-1  47' Taument 4-1
59' Rijkaard 5-1  79' Rijkaard 6-1  90' Bergkamp 7-1

Line-up:  De Goey, R.Koeman, Rijkaard, F.de Boer (Valckx), Winter (Van Gobbel),
 Jonk (De Wolf), Rob Witschge, Overmars (Taument), Bergkamp, 
 R.de Boer (Van Vossen), Roy.

Toronto      12 Jun 94        Canada      0-3  Netherlands
6' Bergkamp 0-1  13' Overmars 0-2  38' Rijkaard 0-3

Line up: De Goey, R.Koeman, F.de Boer, Rijkaard (Van Gobbel), 
Valckx (Witschge), Wouters, Jonk, Bergkamp, Overmars, Roy (Bosman), 
R.de Boer (Van Vossen).
\section{Preview}
This time around, the Dutch qualified the hard way. They made a bad start, and 
after two matches their World Cup chances were in serious jeopardy. Only a 
true team effort helped them out of the mess, under the undisputed guidance of 
Dick Advocaat, the team coach who proved his mettle by taking unpopular but 
necessary decisions. Even so, the help of some inexplicable selections by the 
England manager and an occasional refereeing decision was more than welcome. 
In the end, Oranje deserved its qualification. If the team spirit built 
survives the ``Cruijff discussion'', and the players all accept, as they now seem 
to, Advocaat as the coach in the World Cup, Oranje can go all the way. The 
talent is there and the only problem will be in defence. If Advocaat can find 
two fast defenders able to cover Koeman's lack of pace, this is solvable. He has
already found Van Gobbel - hopefully the other will be found before June.

Failure to qualify for the second round is unthinkable and unacceptable. In
the knock-out stage, anything may happen, but there are but few teams that I
think capable of stopping Oranje on their day. I can only hope they are not
going to be involved in a penalty shoot-out.
\section{Key Players}
ED DE GOEY (Goalkeeper, Feyenoord):
A worthy successor to Van Breukelen with the potential to become one of the
world's best keepers.

RONALD KOEMAN (Defender, Barcelona(SP)): 
Best attacking defender in the world, but too slow to be just a good defender.
He can make perfect passes over 70 or 80 yards, and he has a lethal free-kick.

JAN WOUTERS (Defender, PSV Eindhoven): 
You rarely spot him on the field, but he`s more important then the other 10 
players put together, as he is the binding agent in the team. He never gives up 
and has been known to get physical.

FRANK RIJKAARD (Midfield, Ajax):
A man who can make all the difference, as long as he keeps his nerves under 
control.

DENNIS BERGKAMP (Forward, Inter Milan(ITA)):
The star of the team, who can score fantastic goals with ease.

MARC OVERMARS (Forward, Ajax):
An extremely fast striker starting to make his mark.
