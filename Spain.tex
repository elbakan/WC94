\chapter{Spain}
\chapterprecis{Fernando Rodriguez Pereyra}
\newline
\newline
Coach: Javier Clemente
\begin{figure}[H]
\begin{tabular}{c c l l c c c}
P & & & Club & Age & Caps & Goals \\ \hline
G & 1 & Andoni Zubizarreta (c) & Barcelona &  33 & 86 & 0 \\
G & 13 & Jos{\'e} Luis Ca{\~n}izares & Celta de Vigo & 25 & 2 & 0 \\
G & 22 & Julian Lopetegui & Logrones & 28 & 1 & 0 \\ \hline
D & 2 & Albert Ferrer & Barcelona & 24 & 15 & 0 \\
D & 3 & Jorge Otero & Celta de Vigo & 25 & 4 & 0 \\
D & 4 & Francisco Jos{\'e} Camarasa & Valencia & 27 & 7 & 0 \\
D & 5 & Antuna Fernandez & Abelardo Gijon & 24 & 7 & 2 \\
D & 12 & Sergi Barjuan & Barcelona & 22 & 1 & 1 \\
D & 17 & Salvador Gonzales ``Voro'' & Deportivo La Coruna & 31 & 3 & 0 \\
D & 18 & Rafael Alkorta & Real Madrid & 26 & 17 & 0 \\
D & 20 & Miguel {\'A}ngel Nadal & Barcelona & 28 & 11 & 0 \\ \hline
M & 6 & Fernando Hierro & Real Madrid & 26 & 22 & 6 \\
M & 7 & Jan Andoni Goikoetxea & Barcelona & 29 & 20 & 0 \\
M & 8 & Julen Guerrero & Athletic Bilbao & 20 & 7 & 4 \\
M & 9 & Josep Guardiola & Barcelona & 23 & 9 & 1 \\
M & 10 & Jos{\'e} Maria Bakero & Barcelona & 31 & 25 & 7 \\
M & 11 & Aitor Beguiristain & Barcelona & 30 & 19 & 5 \\
M & 15 & Jos{\'e} Luis Perez Camonero & Atl{\'e}tico Madrid & 27 & 3 & 2 \\
M & 21 & Luis Enrique Martinez & Real Madrid & 24 & 4 & 0 \\ \hline
F & 14 & Juan Castano & Gijon & 23 & 3 & 1 \\
F & 16 & Felipe Minambres & Tenerife & 29 & 3 & 1 \\
F & 19 & Julio Salinas & Barcelona & 32 & 41 & 16 \\ \hline
\end{tabular}
\end{figure}
\section{Preview}
Spain's first official international match was played in 1920 in the Olympic 
Games held in Antwerp, and the team went on to take the silver medal. Spain
missed the first World Cup in 1930 because they couldn't afford to get to 
Uruguay for the tournament and therefore their first World Cup was in 1934 
where they were eliminated by Italy in the quarter-finals. In 1938 Spain was 
engaged in civil war, so their next appearance in World Cup finals was in 1950,
where they defeated England for the first time in their history and finished in
4th place.

They failed to qualify for the next two World Cups, but in 1962 they had their 
best team ever with the likes of Di Stefano, Gento and other members of the all
conquering Real Madrid team. However, they were placed in a group with Brazil
and Czechoslovakia (the eventual finalists) and were knocked out in the first
round. Two years later, they made up for this disappointment by beating USSR in
the final of the European Championship, but they were again handed a tough 
group (Argentina, West Germany and Switzerland) in the 1966 finals and failed 
to progress from the first round.

There followed another 12 year wait before Spain qualified again in 1978, and 
incredibly, they failed to progress from a group containing Brazil, Austria and
Sweden, but in 1982, Spain hosted the World Cup finals and hopes were high that
they could finally put their underachievement on a World stage firmly behind
them.

Spain's group looked fairly straightforward as it contained Honduras, 
Yugoslavia and Northern Ireland, but it wouldn't turn out that way. After 
beating Yugoslavia 2-1 and disappointingly drawing 1-1 with Honduras, Spain 
needed to draw with Northern Ireland in their final match to top the group and 
have an easier group in the second phase. However, it was not to be and to 
cries of disbelief, Gerry Armstrong scored for Northern Ireland and Spain were 
in a second phase group containing West Germany and England with only the top 
team qualifying for the semi-finals. Having lost the first game 2-1 to West 
Germany and the seen West Germany hold England to a 0-0 draw, Spain were 
eliminated before they played the final group match against England who they 
held to a 0-0 draw, so disappointment for the Spanish fans again.

Two years later, Spain managed to reach the final of the European Championship,
playing some sparkling football along the way, but a Platini inspired France 
were too strong in the final winning 2-0.

The 1986 World Cup in Mexico was an up and down affair for Spain where they 
competed in a first round group with Brazil, Northern Ireland and Algeria. The
first game against Brazil was a controversial affair because Spain had a shot
which came down off the underside of the bar and crossed the line, but the goal
was not given and Brazil ran out 1-0 victors. Spain won their remaining 2 
games, including a 2-1 victory over Northern Ireland to take revenge for the 
defeat 4 years earlier, to qualify for the next phase in second place. Here 
they met the most exciting team in the tournament, Denmark, and absolutely
destroyed them. At half-time the score was 1-1 with no hint of the demolition
job that was to come in the second half. After the break, though, Spain's star
striker Emilio Butragueno found that everything he touched turned to gold and 
in scoring 4 goals, he helped the Spanish run out 5-1 winners. Spain were now 
being talked about as possible winners of the trophy, particularly as their 
quarter final match was against a solid but unspectacular Belgian side. 
However, Spain again flattered to deceive and were held 1-1 by the Belgians who
went on to take the match 5-4 in a penalty shootout.

Onto 1990 and Spain were drawn in what looked a tricky group with Belgium, 
Uruguay and South Korea. The first game against Uruguay was a nervy affair with
both teams missing chances and the final score was 0-0. The second game against
South Korea looked an easy win on paper, but Spain didn't impose themselves 
until the second half, finally winning 3-1. Gaining revenge against Belgium, 
2-1 in the last game meant that they qualified for the round of 16 as group
winners. This paired them with Yugoslavia and in a game which either side could
have won in normal time, Spain again regretted missed opportunities as they 
lost 2-1 in extra time.

Two years ago, Spain won the Olympic football tournament in Barcelona by 
beating Poland 3-2 in the final, and a number of the players from that team 
make up the team that Spain will field in this year's World Cup finals.

In June 1992, Javier Clemente became the new coach of the Spanish team 
replacing Vicente Miera who was in charge when Spain beat Albania 3-0 in the
first of their qualifying games. 

Official games since June 1992:
\begin{itemize}
\item Latvia-Spain 0-0
\item N.Ireland-Spain 0-0
\item Spain-Ireland 0-0
\item Spain-Latvia 5-0
\item Spain-N.Ireland 3-1
\item Denmark-Spain 1-0
\item Spain-Lithuania 5-0
\item Lithuania-Spain 0-2
\item Albania-Spain 1-5
\item Ireland-Spain 1-3
\item Spain-Denmark 1-0
\end{itemize}
Friendly games since January 1994:
\begin{itemize}
\item Spain-Portugal 2-2
\item Spain-Poland  1-1
\item Spain-Croatia  0-2
\item Finland-Spain  1-2 
\item Canada-Spain   0-2
\end{itemize}

When Clemente took over, he made drastic changes to the team, the ``quinta del
Buitre'' (Michel, Butragueno \& Co.) was discarded and he tried to blend 
experience with young blood from the Olympic gold medal winning team. It was a
tough start because of the team's inexperience and youth, and they had a lot of
bad luck in their first two qualifying games in Latvia and Northern Ireland but
managed to come away from both games with goalless draws. The home match 
against the Republic of Ireland also produced a goalless draw, which wasn't a 
bad result as Spain played most of the game with 10 men. After good home 
victories against Latvia and Northern Ireland, Spain went to Denmark and after
having 2 good penalty appeals turned down, they lost 1-0. Spain then won their 
next four games including a superb 3-1 victory in Dublin against the Republic
of Ireland, which left them needing to avoid defeat at home to Denmark to 
qualify for the finals. Their cause wasn't helped when goalkeeper Zubizaretta
was sent off early in the game, but Spain still managed to take the lead and
the replacement goalkeeper was quite superb pulling off a series of heroic 
saves, particularly in the second half.

Clemente has changed the national side from a technically gifted side with
very little strength under pressure to a side who are far stronger mentally
with very courageous players. This is similar to the Spanish style of the
1930's and 1950's. Spain have been drawn in a relatively tough group with 
Germany, Bolivia and South Korea, but they should progress to the second phase
without too much difficulty. From here, anything less than a place in the 
quarter finals is a failure for this team, but with this team they could be a
very dangerous outsider and go even further.
\section{Key players}
JOSEP GUARDIOLA (Midfield, Barcelona):
A young player who has extraordinary vision and passing ability.

FERNANDO HIERRO (Utility, Real Madrid):
A fine striker of the ball who can score from anywhere on the pitch.
