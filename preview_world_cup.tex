\chapter{Preview of World Cup 1994}
\chapterprecis{Simon Gleave}
\newline
\newline
If you're a football fan, some of the great memories come from the World Cup 
finals held every four years - I first became enthusiastic about football when
I was 9 years old, sitting in front of the TV screen mesmerised by the 
fantastic Dutch side of 1974. 20 years on, I'm as obsessed as ever with the 
World Cup Finals, one of the greatest sporting events around.

The first question about this year's tournament is who's going to cause a 
shock? Well, this is impossible to answer because if we expected it, it 
wouldn't be a shock but Nigeria can certainly cause a lot of problems in Group 
D and the recent form of the Irish cannot be taken lightly - after winning in 
the Netherlands and then inflicting Germany's first home defeat in six years, 
they must have a chance of overturning Italy, and recording their finest ever 
hour. The real shocks though cannot be predicted - four years ago, who would 
have thought that Costa Rica would reach the second round and Cameroon would 
beat the World Champions, Argentina in the opening game and go on to win their 
group, making them the second African side in a row to top their first round 
group! Will it happen again?

Of course, most of the teams in the finals will give us some entertainment 
before being eliminated and flying back to their homes. Only a minority of the
nations can actually lift the World Cup trophy and more often than not, the 
best team in the tournament will fail to win the World Cup - remember the
Dutch side of 1974 and 1978, the marvellous Brazilians in 1982, the Danish
maestros of 1986, and the Italian side who threw it away in 1990. This year,
the exciting sides are likely to be the Swiss, Romanians, Colombians and
Nigerians, but I don't think for a minute that any of these teams are capable 
of actually winning the coveted title. There are only 6 teams in my opinion
who are capable of winning the World Cup - they are Germany, Brazil, Italy,
Netherlands, Argentina and Spain. All (with the exception of Spain) have 
reached the World Cup final itself and all (barring the Netherlands and Spain) 
have lifted the trophy in the last 25 years.

So, let's look more closely at the six teams who should challenge for the World
Cup - Germany have reached the semi-final at least in every World Cup finals 
bar one since 1966, actually appearing in the final on the last three 
occasions. For this reason alone, they cannot be taken lightly as they always 
appear to produce a team at just the right time. However, this time around, 
they appear weaker than usual with an ageing team and some very patchy results 
in the run up to the finals - I'm not suggesting an early exit, but I think 
that this German team will struggle to match up to it's recent counterparts. 
The Brazilians are associated with flair and attacking power, but in the last 
World Cup they revealed a new meaner style which didn't do them much good - 
they are thought to have reverted back to the expressive Brazilian style that 
we all know and love and with strikers like Romario and Bebeto they must be 
feared. However, it's now 24 years since the great side of 1970 and they 
haven't reached the final since and their reputation appears to be based on the
teams of old, not on their present side.

Argentina have reached 3 of the last 4 finals, but disgraced themselves in 1990
with a physical and ugly style which culminated in 2 sendings off in the worst
final to have ever graced this great stage. After losing the final, Argentina 
had an incredible unbeaten streak of 33 games which only ended last year in 
Colombia. Since then, they have lost to Colombia again and Brazil, but still
remain a team to take seriously. However, their great years coincide with the
presence of Diego Armando Maradona, the World's greatest player of the last 15
years. He will be playing in the USA but after his problems with drugs, he is
surely not the same player. The Netherlands are another exciting team to watch,
but they have a woefully slow defence which should be found out by the first 
good all round team that they meet. I personally think that they are being 
overestimated by everybody and I'll be surprised if they go further than the 
quarter finals - this team is nothing like as good as the finalists of 1974 
and with Ruud Gullit recently walking out on the team and Marco Van Basten 
injured, they have more problems than most.

That leaves 2 teams, Italy and Spain, my personal choice to win the World Cup
and my 'dark horse' selection. I know every South American is going to say
that no European team has ever won the World Cup outside Europe - this is true,
but what might be the reason for this? The only reasonable one that I can find 
is that the heat and humidity have been difficult for the top European teams of
the time to adjust to. This should be no problem for Italy or Spain if they 
have to play in the more humid climbs of the USA.

Spain have been the perennial under-achievers for many years; their club sides
have always been in Europe's top echelon, but the national side have failed on
the big stage. However, a recent change in management and a clearout of the 
team's ageing stars may have made a big difference. There is said to be more 
steel in the team than ever before and with a number of the victorious Olympic
side of 1992 in the squad, they know how to win a major trophy. This could be
the year that the Spanish finally fulfil their potential.

Italy are my tip for this year's World Cup - their defence is undoubtedly the 
soundest of any of the finalists and with Roberto Baggio weaving his magic up
front, they should be a match for anybody. They start in an extremely tough
group which should be very good for them as they won't be taking any of their
opponents lightly. They know all about the Norwegians having played them in 
qualifying for the Euro Championships in 1992, and the Irish will be no 
surprise after their closely fought match in the last World Cup. By the time 
they play Mexico, they should have done enough to progress. From there on, it
is a case of luck with the draw and composure on the field - the Italians 
should have won the last World Cup, but they were unable to be composed when
they came up against a tough opponent in Argentina - this shouldn't happen
this year, because their first round opponents will be as difficult as anybody
else. Forza Italia!!
