\chapter{Preview of Group E}
\chapterprecis{Simon Gleave}
\newline
\newline
\begin{figure}[H]
\small
\begin{tabular}{l c c c c c c c l}
Group E & Played & Won & Drawn & Lost & For & Agst & Apps & Best Performance\\ \hline
Ireland & 5 & 0 & 4 & 1 & 2 & 3 & 1 & Quarter-Final (1990) \\
Italy & 54 & 31 & 12 & 11 & 89 & 54 & 12 & Winners ('34, '38, 82) \\
Mexico & 29 & 6 & 6 & 17 & 27 & 64 & 9 & Quarter-Final (1970, 1986) \\
Norway & 1 & 0 & 0 & 1 & 1 & 2 & 1 & First Round (1938) \\
\end{tabular}
\normalsize
\end{figure}
In every World Cup finals, there is a ``group of death'' and group E in this 
year's tournament keeps the tradition going. It's safe to say that there will 
be very few goals in this group because Ireland, Italy and Norway are all very 
safe at the back, whereas none of them are exactly prolific in attack. Mexico 
are the most difficult team to judge, but they are reputed to have one of their
strongest ever sides which simply adds to the confusion.

Two months ago, I would have unhesitatingly tipped Ireland to finish bottom of
this group - their team was ageing and the way they had been taken apart by 
Spain in qualifying suggested that the good times were over. However, since 
then, they have put together the most impressive set of friendlies of any of 
the finalists by winning three and drawing one against four of their fellow
finalists including winning 1-0 against a near full strength Dutch side in the
Netherlands, and inflicting Germany's first home defeat for 6 years. Add to 
this the discovery of new young talent in Gary Kelly, Phil Babb and Jason 
McAteer and the Irish look like they'll be having another huge party.

The Italians, on the other hand, struggled to qualify and have had the most
appalling run of friendlies, losing at home to France, being well outplayed
by the holders Germany and then suffering a humiliating defeat by an Italian
fourth division club side. However, doesn't this remind us all of 1982 when
Italy stumbled to qualification then lost to everybody in sight before reaching
the World Cup final where they comprehensively beat the Germans. The Italians
seem to perform better when they don't have the notorious Italian press 
on their backs and with their recent results, a humiliation is expected. This 
could work in their favour, after all there is no doubt that they have some of 
the World's finest players like Franco Baresi, Paolo Maldini and Roberto 
Baggio.  

Norway are playing in their first World Cup finals since 1938, when they were
unluckily knocked out by Italy. They were impressive in qualification winning
a group containing The Netherlands and England, and there is no doubt that
coach Egil Olsen has performed miracles to turn a group of fairly ordinary 
players into a World beating team. They are extremely difficult to score 
against and for this reason, they have to be respected. They play a similar
style to Ireland in that the emphasis is on defence and quick counter attack, 
so it'll be interesting to see who comes out on top in the game between these
two (the game itself will probably be the most stultifyingly boring game in
World Cup history).

Mexico are the latest of a run of teams to have an eccentric goalkeeper, but 
Jorge Campos doesn't just run up to the halfway line (like Quiroga of Peru in 
1978) or try to beat opposing forwards (as Higuita of Colombia did in the last
World Cup), he likes to play in attack! In club matches, Campos switches from
goalkeeping duties to scoring duties if he's bored in goal, but we'll have to
wait and see if he'll do this in the heat of a World Cup finals. I think this
group is just too tough for the Mexicans and although they should give us 
plenty of entertainment, it's difficult to see them getting results against 
their three rivals.

Italy, Ireland and Norway will fight it out for Group E, with all three 
qualifying for the second round and Italy scraping the top position.
