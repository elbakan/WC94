\chapter{World Cup Summaries}
\chapterprecis{Matt Huggins}
\newline
\newline
\section{1st World Cup - Uruguay, 1930}
After various problems and hold-ups, the World Cup finally got underway in 
Uruguay on 13th July 1930. The cream of European football was missing, but 3 
continents were represented as the competition took on some kind of world 
flavour. 3 of the 4 seeds reached the semi-finals, only Yugoslavia upsetting 
Brazil prevented 100% success for the seeded countries. The two South American
teams had little trouble, Argentina beating the U.S.A., and Uruguay beating 
Yugoslavia, both games decided by 6-1 margins. In the final, Uruguay took the
lead, but by half-time the host nation were 2-1 down. By the 65th minute, 
Uruguay were 3-2 up, but the game was finely balanced until Castro scored in
the dying seconds. A national holiday was declared, whilst in Buenos Aires, the
Uruguayan consulate was stoned by an angry mob. The World Cup had arrived.
\section{2nd World Cup - Italy, 1934}
The first World Cup to have a qualifying competition, and the first (and only)
time the holders have declined to defend their title. A straight knock-out 
system was used, with only European countries making round two. Italy overcame
Spain in a replay after two tough matches, whilst Austria beat Hungary in 
another difficult match. Germany beat Sweden 2-1, whilst the best match of the
round saw the elegant Czechoslovakian team harried all the way by Switzerland 
before winning an end-to-end game 3-2. A heavy pitch and hard opponents made 
sure Austria lost to Italy in the semi-final, with the Czechoslovakians having 
relatively little trouble beating Germany 3-1. The final was quiet until 20 
minutes from the end, when the Czech Puc scored. Twelve minutes later, Italy 
equalized, and in extra time, Schiavio scored a winner which gave Italy victory
and Mussolini a great boost in popular acclaim.
\section{3rd World Cup - France, 1938}
After Austria's late withdrawal, there were seven first round ties to be 
played, with Sweden receiving a bye into the second round. The best game of the
round was Brazil overcoming Poland 6-5 in Strasbourg. Switzerland surprisingly 
overcame Germany, but the real shock came when Cuba beat Romania 2-1 in a 
replay. Cuba were humbled in the next round, losing 8-0 to Sweden. The 
defending champions went through against the hosts, and the Swiss lost to 
Hungary in Lille. In Bordeaux, Brazil and Czechoslovakia played a torrid match,
which ended 1-1, 2-1 to Brazil in sendings-off, and left the Czechs with two 
players with broken bones. The replay was more sedate, with Brazil winning 2-1.
In the semis, Italy made light work of the tired Brazil, and a 2-1 scoreline 
flattered the South Americans. Hungary had little trouble against the Swedes, 
winning 5-1 after conceding an early goal. In the final, the Italians went 
into a 3-1 half-time lead, and from then on were never in serious trouble of 
losing their trophy. They won 4-2 and with the onset of World War Two, kept the
Cup for another 12 years.
\section{4th World Cup - Brazil, 1950}
Back to the group system, with only 13 teams participating. In Group 1, Brazil 
had little trouble qualifying for the final pool. Group 2 saw maybe the most 
amazing result in World Cup history. After beating Chile, England, favourites 
for the Cup, faced the U.S.A., and despite having all of the pressure, lost 
1-0, with a goal by Gaetjens in the 37th minute, and so Spain went through from 
this pool. Sweden and Uruguay qualified reasonably easily from groups 3 and 4 
respectively, Uruguay only having to play 1 match.

In the final pool, Brazil played breathtaking football in beating Sweden 7-1, 
and Uruguay were struggling in their matches. However, the final pool match 
between Brazil and Uruguay became, effectively, the World Cup Final. Before the
highest ever crowd for a World Cup match, 199,854, Brazil were favourites, but 
Uruguay defended resolutely, and by half-time, were beginning to show some 
strengths themselves. After the restart, Brazil scored through Friaca, with 
Uruguay equalising midway through the second half, and ten minutes later, 
scoring again through Ghiggia. Brazil attacked furiously, but to the dismay of
the crowd, the final score was 2-1 to Uruguay, and the Cup returned to 
Montevideo.
\section{5th World Cup - Switzerland, 1954}
The group system was used again, but with an added twist. Two teams were seeded
in each group, and therefore there were only 4 matches. This had a profound 
effect on where the Cup ended up. In the group matches, Brazil and Yugoslavia 
qualified easily from Group 1, and the impressive Hungarians and West Germany 
had little trouble in Group 2. Austria wobbled against Scotland, but they and 
Uruguay qualified from Group 3. England were shaken by Belgium in Group 4, but 
came back against Switzerland, who qualified at the expense of Italy in a play-
off.

In the quarter finals, Uruguay, West Germany and Austria overcame their 
opponents, but it was the other match, between Brazil and Hungary, that grabbed
the headlines for all the wrong reasons - The Battle of Berne. Hungary went 2-0
up within minutes, then resorted to a series of fouls on the Brazilian 
forwards, which culminated in a penalty and a goal for the Brazilians. 2-1 at 
half-time, with temperatures rising. The fouls continued in the second half, 
and after a further goal for each side, Bozsik and Santos were sent off for 
fighting. In the final seconds, Hungary wrapped up the match with a further 
goal, and Tozzi was sent off. After the final whistle, there was further 
fighting in the dressing rooms. FIFA, in it's wisdom, swept the whole affair 
under the carpet...

In the semis, Hungary and West Germany beat their opponents. West Germany had 
little trouble against Austria, winning 6-1, while the Hungary-Uruguay game was
a classic, with Hungary eventually winning 4-2. Hungary were weakened for the 
final against Germany. Puskas played, but he was not fit, and as the game went
on, he became a passenger. The game was locked at 2-2 until 7 minutes from the
end, as Rahn scored for Germany. Hungary put the ball in the net, but it was 
ruled offside, and the final whistle came not long after. The ``Magical Magyars''
had been defeated, and the Cup was West Germany's for the first time.
\section{6th World Cup - Sweden, 1958}
Reverting back to the``all play all'' group format, the competition was less 
controversial. In Group 1, West Germany and Northern Ireland qualified, against
the odds, after winning a play-off with Czechoslovakia. France, with the 
goalscoring of Just Fontaine, had little trouble qualifying, and Yugoslavia 
came with them from Group 2. A fairly dour Group 3 saw Sweden and Wales 
progress, after beating the post-revolution, disillusioned Hungarian side in a
play-off. Brazil qualified with the U.S.S.R. from Group 4, after the Soviets 
had needed a play-off to see off England. The quarter finals saw Brazil, 
France, West Germany and Sweden all progress with little difficulty, although 
Wales did well in holding the Brazilians to a 1-0 deficit. Sweden, helped by 
their supporters, overpowered the Germans 3-1 in one semi-final, whilst in the 
other the young Pel{\'e} scored a hat-trick as the Brazilians overwhelmed the 
French 5-2.

So to the final, where the Swedish crowd were unexpectedly quiet, after the 
cheerleaders that had been there for the semi against West Germany had been 
removed at the request of the Brazilians. Sweden scored first, but within 
minutes Brazil were on level terms, and soon after, ahead. The half-time score
was 2-1. Brazil really turned on the magic in the second half, Pel{\'e} scoring 
maybe the best ever goal in a World Cup final. By the end, the score was 5-2 to
Brazil, and as the team ran around the pitch waving the flags of both 
countries, the crowd applauded them. For the first time, the best team in the 
competition had won the title.
\section{7th World Cup - Chile, 1962}
The same format as 1958, except that now goal difference would be used, and not
the play-off system. Group 2 was marked by the ``Battle of Santiago'', between 
Italy and Chile, which saw 2 sendings off, several injuries including a couple
of broken bones. This time the players involved were either suspended or 
severely admonished by FIFA. Chile and West Germany qualified for the quarter 
finals. They were joined by the U.S.S.R. and Yugoslavia from Group 1, despite 
the fact that the Soviets had lost a 4-1 lead, drawing 4-4 with Columbia, in 
their second group match. Brazil, Czechoslovakia, Hungary and England were the 
other quarter-finalists.

Yugoslavia overcame West Germany 1-0 with a goal in the last 5 minutes, as 
Chile beat the U.S.S.R. 2-1, in an unbelievable atmosphere. Brazil turned on 
the style against England, winning 3-1. Czechoslovakia met Hungary in the other
tie. They went ahead after 13 minutes, and then put up the shutters. The 
Hungarians played some dazzling football, but could not beat Schroiff in the 
Czech goal, and the Czechs held on. The two semis saw a great contrast in crowd
numbers. There were 76,594 to see Chile fight bravely, but ultimately lose, 
against a far better Brazillian side, but only 5,890 to see Czechoslovakia 
have little trouble against Yugoslavia, winning 3-1. The final saw the 
underdogs, Czechoslovakia, take the lead in the 16th minute with a goal by 
Masopust, but Brazil struck back within just two minutes, and went on to win 
3-1 with two more goals after the interval. Two trophies in a row for the South
Americans. Who could stop them?
\section{8th World Cup - England, 1966}
The competition followed exactly the same format as the previous World Cup.
England and Uruguay qualified from Group 1, the only cost being the loss to 
England of Jimmy Greaves through injury. West Germany and, surprisingly, 
Argentina went through from Group 2. The Spainish were expected to get through,
but the Argentine combination of a sound defence and some ruthless tackling got
them into the quarter finals. Portugal, playing the best football in the 
tournament, and Hungary qualified from Group 3, eliminating Brazil in the 
process. Group 4 was the surprise group, as the North Koreans beat Italy, to 
qualify with the Soviet Union.

In the quarter finals, England overcame Argentina by a single goal in a match 
marred by fouls and bad temper. West Germany beat Uruguay, and the Soviet Union
beat Hungary in a tight match. In the other match, North Korea went into a 3-0
lead against Portugal, but the Portuguese regained their compusure, and came 
back to win 5-3. The USSR. - West Germany semi final was a dour affair, with
the Germans winning 2-1. The same score was the result in England's favour at 
Wembley the next day, but the game was far better, with Bobby Charlton playing 
perhaps his finest ever game in an England shirt.

In the final, West Germany opened the scoring after 13 minutes, but England 
pulled back, and went ahead, with goals from Hurst and Peters. This looked to 
be the final score, until Weber scored in the dying seconds. The game went to 
extra time, and ten minutes into the first period, Hurst fired a shot against 
the underside of the German bar. It bounced down, then out. The goal is debated
to this day, but it was given (after initially being ruled out), and to seal
matters, Hurst scored in the final minute. The only ever hat-trick in a World
Cup final had seen the Cup come back to where organised football began.
\section{9th World Cup - Mexico, 1970}
The same format as the 1966 competition, but this time substitutes were 
allowed for the first time.

There were many problems, chiefly to do with the altitudes and temperatures at
which some games were played, but the USSR. and Mexico had little trouble 
qualifying from Group 1. Group 2 saw Italy and Uruguay go through, and Group 3
ended in England and Brazil qualifying. The game between the two countries was
probably the best played in the tournament, Brazil winning 1-0, with perhaps 
the best save ever being made by Gordon Banks from a Pel{\'e} header. West Germany 
and Peru qualified from Group 4.

The first quarter final was between West Germany and England, a re-run of the 
1966 Final. West Germany eventually won 3-2, having been 2-0 down. This was 
partly due to some strange tactical decisions by Alf Ramsey, who had taken off 
Bobby Charlton (to save him for the semi), when Cooper was virtually on his 
knees. Brazil and Italy had little trouble overcoming Peru and Mexico 
respectively, and Uruguay overcame the Soviets in extra time. Brazil beat 
Uruguay 3-1 in the first semi, and Italy beat West Germany in a game which had
been a dull one until the 91st minute, when it was taken into extra time, Italy
eventually winning 4-3. The final became a lesson in football. Brazil dominated
from virtually the start, won 4-1, and kept the Jules Rimet trophy permanently.
It was fitting that possibly the greatest side ever should keep the first 
trophy.
\section{10th World Cup - West Germany, 1974}
The group system was enlarged, to cover the quarter finals. The 8 teams that 
qualified would now go into 2 groups of four, the winners of these groups 
meeting in the final. A new World Cup was on offer to the winners, but never 
again to be retained permanently. East and West Germany qualified from Group 1,
joined by Yugoslavia and Brazil from Group 2. Scotland went out on goal 
difference, having scored 4 points, the same as the two qualifiers. The 
Netherlands danced their way through in Group 3, with Sweden joining them. 
Poland dominated Group 4, and with them qualified Argentina.

Group A in the second round began with a Brazilian victory against East 
Germany, but it was the Netherlands who dominated the group, beating Argentina 
4-0, then East Germany, and finally Brazil, to reach the final. Group B saw 
West Germany finally play to their full potential, as they beat Poland, Sweden
and Yugoslavia to meet the Dutch in the final. The Netherlands took the lead 
in the final before a German had even touched the ball. They were awarded a 
penalty after an attack from the kick-off had resulted in a foul on Cruyff by 
Hoeness. However, the Germans recovered from this early setback, and scored two
goals before half-time. This was how things remained, and the best team in the 
competition had to be content with second place, as the determined West 
Germans won the World Cup.
\section{11th World Cup - Argentina, 1978}
Despite the various misgivings about having Argentina as hosts, the tournament
began as scheduled, following the same format as four years earlier. Italy, 
Argentina, Poland and West Germany had little trouble in qualifying for the 
semi-final groups. Brazil, despite internal disagreements and arguments, went 
through from Group 3, along with Austria, at the expense of Spain. In Group 4,
Scotland started by losing 3-1 to Peru, then drew 1-1 with Iran before beating
the Dutch 3-2. However, they were again let down by their results against the 
lesser nations and went out on goal difference as Peru and the Netherlands went
through.

In the semi-final groups, the Dutch impressed in coming through against Italy, 
West Germany and Austria. There has been speculation that Argentina bribed 
their way into the final as they needed to win their final Group B game against
Peru by four clear goals. The performance of the Peruvian goalkeeper that night
and the final score of 6-0 had people asking questions about the validity of 
the game but nothing was ever proved and the hosts went through to contest the 
final at the expense of Brazil.

The final got underway, with the Argentines using gamesmanship, as the Dutch 
went in for some ruthless tackling. Argentina, with the crowd and referee 
behind them, took the lead in the 38th minute through Kempes. In the second 
half, the Netherlands reorganised, and scored through Nanninga. They almost 
snatched the game by hitting a post in the final minute of normal time.
So to extra time, when Kempes scored after 14 minutes, and as they pressed for 
the equaliser, the Dutch were caught again, as Bertoni put the game beyond 
doubt. At the final whistle, the crowd celebrated, a happy end to an ill-
tempered and dreary final.
\section{12th World Cup - Spain, 1982}
The biggest change in World Cup history, as 24 teams contested the final series
for the first time. The teams were put into 6 groups of 4, with half the teams 
going out in the first round, as the top two teams in each group went through 
to the second group stage. The main talking points of the first round were 
Cameroon's draw with Italy, and West Germany's loss to Algeria, followed by the
disgraceful match between Austria and the West Germans, where both sides were 
assured of qualification for the next stage if West Germany won 1-0. Hrubesch
scored an early goal to put West Germany 1-0 up, and the two teams spent the 
rest of the match passing the ball around and rarely putting their opponents'
goal under any threat. This lead FIFA to change the rules for the final round
of group games and have them kicking off at the same time. In Group 4, 
England's Bryan Robson scored the quickest goal in World Cup final history, 27 
seconds into the game against France. There was also controversy in this group 
with the walk-off by Kuwait's players against France, which was stopped by 
Prince Fahid, who made his players return to the pitch! There were no real 
surprise qualifiers into the second round.

The ridiculous group of three system meant each team only played 2 games in 
this group stage. Poland qualified easily from Group A, joined by West Germany
from Group B. Italy came through Group C after a gripping battle against Brazil
which saw a hat-trick for Paolo Rossi, and France progressed from Group D. The
semi-final between France and West Germany has gone down as a classic. As well
remembered for the foul by Schumacher on Battiston as the final score, the game
went all the way to penalties, the first time in a World Cup, which West 
Germany won 5-4. Italy overcame Poland in the other semi. After a rarity in the
World Cup, a good third-place match, Italy met West Germany in the final. Italy
had a bad start, losing Graziani in the 7th minute, and then missing a penalty 
in the 24th. However, Rossi started the scoring for Italy in the second half, 
and they went into a 3-0 lead. West Germany pulled one back in the last 8 
minutes, but it was too late, and the Italians became the first European 
country to win the World Cup three times.
\section{13th World Cup - Mexico, 1986}
Colombia was the original choice for the 13th World Cup, but they could not
raise the finance required, and so Mexico took over. The second group stage
was abolished, and it became a straight knock-out tournament after the first 
round, which saw 24 teams in 6 groups of 4, with the top two teams and the
4 best third place teams proceed to round two. There were few shocks in the 
opening group matches, except perhaps the inability of Scotland to overcome a
Uruguay side that was reduced to 10 men in the very first minute. A win would
have seen Scotland qualify for the second phase for the very first time. 
Group F saw England give very mediocre displays against Portugal and Morocco, 
but then come good with a fine display against Poland, going 3-0 up after just
35 minutes, and keeping this scoreline, to ensure passage into the next round. 
Morocco topped the group.

The second round game between the USSR and Belgium proved to be possibly 
the best of the competition, Belgium eventually winning an end-to-end game 4-3
in extra time. Mexico and Brazil had little trouble, as did Argentina and 
England, who would now meet for the first time since the Falklands conflict. 
France and West Germany came through tough ties against Italy and Morocco 
respectively, and Spain, led by Butragueno, thrashed Denmark 5-1. The Brazil-
France quarter final was another classic, France eventually winning the penalty
shoot-out 4-3. West Germany beat Mexico 4-1 in a penalty shootout, whilst 
Belgium also required a penalty shoot-out to overcome Spain, 5-4 after the 
score was 1-1 at the end of extra time. All eyes were focused on Argentina vs 
England. The security forces turned up in force, but all the action was on the 
pitch. The first goal for Argentina, after 50 minutes, has been talked about 
ever since. Maradona chased a sliced back-pass from Hodge, challenged Shilton, 
and fisted the ball into the net. The goal was given, and 5 minutes later, 
Maradona added to it with a spectacular goal, which involved running half the 
pitch and beating at least 3 England defenders. England scored through Lineker,
but it was too late. West Germany and Argentina made the final, beating France 
and Belgium respectively. Argentina took the lead through Brown, and then after
half-time Valdano increased the deficit. West Germany looked down and out, but 
goals from Rummenigge and Voeller brought them level. Three minutes after their
second goal, Germany were split by a pass from Maradona, and Burrachaga went 
through to score the goal that gave Argentina the World Cup.
\section{14th World Cup - Italy, 1990}
The same format as 1986, as Italy became the second country to host two World 
Cup competitions. The shock of the first round was the defeat of World 
champions Argentina by Cameroon, after the Africans had been reduced to 10 men,
and then 9. Costa Rica beat Scotland, who again failed to make the latter 
stages after almost holding out against Brazil, only to lose 1-0 with a goal 
for Muller in the last 10 minutes, followed by an astonishing save by Taffarel 
from Johnston in the dying seconds. West Germany looked impressive against the 
U.A.E. and Yugoslavia, and Uruguay squeezed through in the last minute of their
last group game against South Korea, with a goal from Fonseca. Group F saw 
England qualify, with Holland and Ireland drawing lots to determine their final
position in the group, although the nightmare scenario of drawing lots to 
determine the qualifiers was avoided.

Cameroon beat Colombia to reach the quarter finals, and were joined by 
Czechoslovakia and Argentina, who undeservedly beat Brazil. The Netherlands/
West Germany game will be remembered for the incident of Rijkaard spitting at 
Voeller, but the Germans had the last say, winning 2-1. Eire overcame Romania 
on penalties, and Italy beat Uruguay. Yugoslavia looked good in beating Spain, 
and England were staring a penalty shoot-out in the face until they overcame 
Belgium with a goal from Platt with the last kick of the game.

Argentina turned in another poor performance in beating Yugoslavia on penalties
in the first of the quarter finals. Italy beat the Irish by a single goal, and 
England came back from the dead to beat Cameroon 3-2 in a thrilling match in 
Naples. West Germany beat Czechoslovakia 1-0 in the other tie.

In the semis, Argentina, for the first time, played good, attractive football,
but couldn't shake Italy until the penalty shoot-out. 1-1 after extra time, 
Argentina won the penalties 4-3, and were on their way to another final. The 
following day saw the finest game of the series. England rocked West Germany as
they attacked from the start, but it was the Germans that took the lead on the 
hour with a free kick that took a cruel deflection off Parker standing in the 
wall. England fought back, and equalised through Lineker with 10 minutes of 
normal time left. Extra time saw both teams have chances, but the game went to
penalties, which West Germany won 4-3.

The third place match saw Italy beat England 2-1 in an excellent game, in sharp
contrast to what was to follow. The worst game of the series, the final was 
full of niggling fouls by the Argentines, and blatent gamesmanship by the 
West Germans. Argentina had Monzon and Dezotti sent off, and Germany scored a 
penalty 5 minutes from time that won them the Cup. A sad game to end what had 
been an exciting, if not always pretty, tournament.
\section{Summary}
\begin{figure}[H]
\tiny
\begin{tabular}{l l l l l l c c}
Year & Winning Nation & Winning Captain & Winning Manager & Leading Scorer & Nation & Games & Goals\\ \hline
1930 & Uruguay & Jose Nassazzi & Odino Viera & Guillermo Stabile & Argentina & 4 & 8 \\
1934 & Italy & Giampiero Combi & Vittorio Pozzo & Angelo Schiavio & Italy & 4 & 4 \\
& & & & Oldrich Nejedly & Czechoslovakia & 4 & 4 \\
& & & & Edmund Conen & Germany & 4 & 4 \\
1936 & Italy & Guiseppe Meazza & Vittorio Pozzo & Da Silva Leonidas & Brazil & 4 & 8 \\
1950 & Uruguay & Obdulio Varela & Colonel Volpe & Marques Ademir & Brazil & 6 & 7 \\
1954 & West Germany & Fritz Walter & Sepp Herberger & Sandor Kocsis & Hungary & 5 & 11 \\
1958 & Brazil & Hilderaldo Bellini & Vincente Feola & Just Fontaine & France & 6 & 13 \\
1962 & Brazil & Ramos De Oliviera Maura & Aymore Moreira & Drazen Jerkovic & Yugoslavia & 6 & 5 \\
1966 & England & Bobby Moore & Sir Alf Ramsey & Ferreira Eusebio & Portugal & 6 & 9 \\
1970 & Brazil & Carlos Alberto & Mario Zagalo & Gerd Muller & West Germany & 6 & 10 \\
1974 & West Germany & Franz Beckenbauer & Helmut Shoen & Grzegorz Lato & Poland & 7 & 7 \\
1978 & Argentina & Daniel Passarella & Cesar Luis Menotti & Mario Kempes & Argentina & 7 & 6 \\
1982 & Italy & Dino Zoff & Enzo Bearzot & Paolo Rossi & Italy & 7 & 6 \\
1986 & Argentina & Diego Maradona & Carlos Bilardo & Gary Lineker & England & 5 & 6 \\
1990 & West Germany & Lothar Matthaus & Franz Beckenbauer & Salvatore Schillaci & Italy & 7 & 6 \\ \hline
\end{tabular}
\normalsize
\end{figure}

\begin{figure}[H]
\small
\begin{tabular}{l l c c c c c}
Year & Host Nation & Matches & Attendance & Average & Goals & Average \\ \hline
1930 & Uruguay & 18 & 434,500 & 24,139 & 70 & 3,88 \\
1934 & Italy & 17 & 395,500 & 23,235 & 70 & 4.11\\
1938 & France & 18 & 483,000 & 26,833 & 84 & 4.66 \\
1950 & Brazil & 22 & 1,337,000 & 60,772 & 88 & 4.00 \\
1954 & Switzerland & 26 & 943,000 & 36,270 & 140 & 5.38 \\
1958 & Sweden & 35 & 868,000 & 24,800 & 126 & 3.60 \\
1962 & Chile & 32 & 776,000 & 24,250 & 89 & 2.78 \\
1966 & England & 32 & 1,614,677 & 50,458 & 89 & 2.78 \\
1970 & Mexico & 32 & 1,673,975 & 52,312 & 95 & 2.96 \\
1974 & West Germany & 38 & 1,774,022 & 46,685 & 97 & 2.55 \\
1978 & Argentina & 38 & 1,610,215 & 42,374 & 102 & 2.68 \\
1982 & Spain & 52 & 1,766,277 & 33,967 & 146 & 2.81 \\
1986 & Mexico & 52 & 2,285,498 & 43,952 & 132 & 2.54 \\
1990 & Italy & 52 & 2,512,900 & 48,325 & 115 & 2.21 \\ \hline
\end{tabular}
\normalsize
\end{figure}
