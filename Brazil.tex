\chapter{Brazil}
\chapterprecis{Mauricio Dziedzic}
\newline
\newline
Coach: A.C. Parreira (former coach of Kuwait and Saudi Arabia)
\begin{figure}[H]
\begin{tabular}{c c l l c c c}
P & & & Club & Age & Caps & Goals \\ \hline
G & 1 & Taffarel & Reggiana (Ita) & 28 & 79 & 0 \\
G & 12 & Zetti & Sao Paulo & 28 & 25 & 0 \\
G & 22 & Gilmar & Flamengo & 35 & 22 & 0 \\ \hline
D & 2 & Jorginho & Bayern Munich (Ger) & 29 & 69 & 5 \\
D & 3 & Ricardo Rocha & Vasco de Gama & 31 & 58 & 0 \\
D & 4 & Ricardo Gomes & Paris St Germain (Fra) & 30 & 63 & 4 \\
D & 6 & Branco & Fluminense & 30 & 75 & 8 \\
D & 13 & Aldair & AS Roma (Ita) & 28 & 23 & 1 \\
D & 14 & Cafu & Sao Paulo & 24 & 47 & 1 \\
D & 15 & M{\’a}rcio Santos & Girondins Bordeaux (Fra) & 24 & 37 & 3 \\
D & 16 & Leonardo & Sao Paulo & 24 & 27 & 0 \\ \hline
M & 5 & Mauro Silva & Deportivo la Coruna (Spa) & 26 & 40 & 0 \\
M & 8 & Dunga &  VfB Stuttgart (Ger) & 30 & 52 & 5 \\
M & 9 & Zinho & Palmeiras & 27 & 36 & 2 \\
M & 10 & Rai & Paris St Germain (Fra) & 28 & 52 & 13 \\
M & 17 & Mazinho & Palmeiras & 27 & 50 & 0 \\
M & 18 & Paulo Sergio & Bayer Leverkusen (Ger) & 25 & 19 & 2 \\ \hline
F & 7 & Bebeto & Deportivo la Coruna (Spa) & 30 & 81 & 31 \\
F & 11 & Rom{\’a}rio & Barcelona (Spa) & 28 & 52 & 24 \\
F & 19 & Muller & Sao Paulo & 28 & 67 & 12 \\
F & 20 & Ronaldo & Cruzeiro & 17 & 6 & 1 \\
F & 21 & Viola & Corinthians & 25 & 9 & 1 \\ \hline
\end{tabular}
\end{figure}
\section{History}
Two sentences say it all: Brazil is the only country to have played in all 
World Cup finals, and, by virtue of having been the first nation to ever win 
three world cup titles, it is the rightful owner of the Jules Rimet trophy 
(prior to 1970, the trophy would be in the possession of the prevalent World 
Champions).

The yellow jerseys of Brazil evoke magic memories of World Cups past, and God's
gifted football wizards that wore them are household names around the globe. A 
Brazilian travelling abroad has only to mention his nationality and a stream of 
names will start pouring from the mouths of people, from cab drivers and 
waiters to scientists.  Pel{\'e}, Garrincha, Rivelino, Jairzinho, Socrates, Zico, 
and Falcao are only a few of the players that are regularly mentioned.

The exodus of the country's top players to European teams which started during
the 1980's can be highlighted as a major factor preventing the formation of a 
cohesive side.  Different schedules between European and Brazilian tournaments,
plus the unwillingness of clubs to release players for training or friendly 
internationals have all contributed to Brazil's troubles during this period. 
Add to that the fact that players sent to the World Cup finals who are still 
playing on home soil tend to use the opportunity to showcase their talents and 
land multi million dollar contracts, therefore play for individual glory rather 
than for the team's benefit.

However, it is felt that the tide may be about to turn. European teams are no 
longer as well off financially as they were, while Brazilian clubs are being 
restructured and now offer better rewards for the home talent. Also, while the 
team that represented Brazil in Italy in '92 has now matured, a new generation 
of players has emerged and there is no shortage of skill to choose from.

\section{WC qualifying}
\begin{itemize}
\item Ecuador 0 x 0   Brazil
\item Venezuela 1 x 3 Brazil
\item Bolivia 2 x 0   Brazil
\item Uruguay 1 x 1   Brazil
\item Brazil  2 x 0   Ecuador
\item Brazil  6 x 0   Bolivia
\item Brazil  4 x 0   Venezuela
\item Brazil  2 x 0   Uruguay Rom{\’a}rio (2)
\end{itemize}
\section{Friendlies}
\begin{itemize}
\item Germany 2 x 1   Brazil
\item Mexico  0 x 1   Brazil     (Rivaldo)
\item Brazil  2 x 0   Argentina  Bebeto(2)
\item Brazil  3 x 0   Iceland
\item Canada  1 x 1   Brazil       (Rom{\’a}rio)
\item Brazil  8 x 2   Honduras  (Rom{\’a}rio 3, Bebeto 2, Dunga, Cafu, Rai)
\end{itemize}
\section{Preview}
A convincing and thorough beating of Uruguay in the last game of the qualifying
series lifted Brazil to the top of its group and ensured the continuation of a 
perfect participation record in the W.C. finals. However, the defeat by Bolivia, 
partly avenged by the six-goal win in the home leg, was Brazil's first ever 
defeat in a qualifying competition.

Brazil's play during the first half of the tournament, all away matches, 
clearly showed that it takes more than individual talent to win a game. Some of 
the media said it takes a coach too.  The performances were irregular and 
reflected the consequences of assembling a team just prior to the matches. The 
side should have been put together during the Copa America, but European clubs 
refused to release players at that time. Many Brazilians feared that the team
might not make it to USA '94.

The 1-1 draw against Uruguay in Montevideo showed a glimpse of better days to 
come, as the team started to shape up and only allowed Uruguay to equalise as a
result of an individual mistake by newcomer Antonio Carlos, who took the pitch
as a substitute without time for a proper warm-up.

The home matches hoisted Brazil to the top of the group, buying Carlos Alberto 
Parreira, the coach, more time to prove his defensive style. Most fans would 
rather see Tele Santana lead the team, but apparently he'll not get another 
chance after blowing it in 1982. Tele, now coaching Sao Paulo, the current 
Toyota Cup champions, tends to assemble teams whose style pleases the crowds 
and play the traditional Brazilian ``happy'' football. Parreira, aided by 
Zagalo who coached the brilliant 1970 team, has a different philosophy, leaning 
towards a more defensive style of play. Dunga, a player much criticized by most 
of the media and fans, personifies this style and seems to have a guaranteed 
place in the squad.

\section{Key Players}
ROM{\’A}RIO (Forward, Barcelona(SP)):
Missed the 1990 finals due to injury, and was only recalled to action for the 
last match of the qualifying round, when he promptly scored twice and appointed
himself as the savior of Brazilian football. His explosive personality, and at
times dumb remarks have been the cause of much aggravation and kept him from 
joining the team earlier. It is feared that his self-centred personality and 
little regard for his team mates might disturb the whole atmosphere of the 
team. However, he should shine in the US and is expected to be one of the top 
goal scorers.

BEBETO (Forward, Deportivo La Coruna(SP)):
Bebeto is a proven goal scorer both at international and club level. His 
teaming-up with Romario should produce a very effective combination and bring 
samba-flavoured football to the tournament.

MULLER (Forward, Sao Paulo):
A very skilled, quick winger who has yet to perform to his ability with the 
national team and is expected to fulfil his potential in this year's World Cup
finals.

CAFU (Defender/Midfield, Sao Paulo):
Has shown great form and skill during recent matches. At club level plays in 
midfield, but has performed very well at right back when substituting for
Jorginho.

