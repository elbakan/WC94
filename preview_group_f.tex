\chapter{Preview of Group F}
\chapterprecis{Simon Gleave}
\newline
\newline
\begin{figure}[H]
\small
\begin{tabular}{l c c c c c c c l}
Group F & Played & Won & Drawn & Lost & For & Agst & Apps & Best Performance\\ \hline
Belgium & 25 & 7 & 4 & 14 & 33 & 49 & 8 & Fourth (1986) \\
Morocco & 7 & 1 & 3 & 3 & 5 & 8 & 2 & Second Round (1986) \\
Netherlands & 20 & 8 & 6 & 6 & 35 & 23 & 5 & Runners-Up (1974, 1978) \\
Saudi Arabia & & & & & & & & \\ \hline    
\end{tabular}
\normalsize
\end{figure}
Group F consists of 2 strong European nations with a fierce rivalry together
with the third (and weakest) of the Africans and Saudi Arabia, widely thought
of as the the weakest nation in the tournament. So, it's a simple task for
Belgium and the Netherlands to progress, right? Well, yes, although the 
Netherlands to tend to be over confident against weak nations and this could
be their undoing against Morocco.

Belgium looked to be coasting to qualification before they came up against 
Wales in Cardiff where they were well beaten 2-0. After this, they looked very
nervy and finally came through with a goalless home draw against the Republic 
of Czechs and Slovaks. They have an excellent World Cup pedigree in recent 
years having reached the semi-final in 1986 and been unlucky second round 
losers to England in 1990 having hit the post twice. This time around should
be no different and they are my choice to win the group with Enzo Scifo finally
revealing his towering talent on a World stage.

Morocco were the first African nation to reach the second phase of the World 
Cup finals when, in 1986, they topped a group containing England, Poland and 
Portugal but were then eliminated in a tight match against eventual finalists,
West Germany. This time around, they are not as strong and they are arguably up
against far tougher opponents than in 1986.

The Netherlands are an interesting side who are blessed with magnificent skill
in some quarters, Ruud Gullit, Dennis Bergkamp and Ronald Koeman for example, 
but they suffer from having the slowest defenders playing in international
football today. This could be their eventual undoing in this year's finals.
However, they should progress easily enough from this group as only Belgium
are a real threat.

Saudi Arabia are the latest in a long line of teams from the Middle East to 
launch an assault on the World Cup finals. None have set the tournament alight
and it's difficult to see Saudi Arabia being the first, although, as usual, 
the players have been offered vast financial rewards as usual.

So, Belgium to take the group with the Netherlands second, Morocco third and 
Saudi Arabia bringing up the rear.
