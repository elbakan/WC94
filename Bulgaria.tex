\chapter{Bulgaria}
\chapterprecis{Reproduced from UPI Reports}
\newline
\newline
Coach: Dimitar Penev
\begin{figure}[H]
\begin{tabular}{c c l l c c c}
P & & & Club & Age & Caps & Goals \\ \hline
G & 1 & Borislav Mikhailov & Mullhouse (Fra) & 32 & 70 & 0 \\
G & 12 & Plamen Nikolov & Levski Sofia & 32 & 3 & 0 \\ \hline
D & 2 & Emil Kremenljev & Levski Sofia & 24 & 5 & 0 \\
D & 3 & Trifon Ivanov & Xamax Neuchatel (Swi) & 29 & 38 & 4 \\
D & 4 & Tsanko Tsvetanov & Levski Sofia & 24 & 16 & 0 \\
D & 5 & Petar Hubtsjev & Hamburg (Ger) & 20 & 16 & 0 \\
D & 15 & Nikolaj Ilijev & Stade Rennes & 31 & 51 & 0 \\
D & 16 & Ilijan Kirjakov & Lerida & 27 & 29 & 0 \\ \hline
M & 6 & Zlatko Yankov & Levski Sofia & 28 & 27 & 3 \\
M & 9 & Jordan Letskjov & Hamburg (Ger) & 27 & 14 & 1 \\
M & 11 & Daniel Borimirov & Levski Sofia & 27 & 7 & 0 \\
M & 19 & Georgij Georgijev & Mullhouse (Fra) & 28 & 0 & 0 \\
M & 20 & Krasimir Balakov & Sporting Lisbon (Por) & 28 & 34 & 4 \\ \hline
F & 7 & Emil Kostadinov & Porto (Por) & 26 & 31 & 11 \\
F & 8 & Hristo Stoichkov & Barcelona (Spa) & 28 & 42 & 13 \\
F & 10 & Nasko Sirakov & Levski Sofia & 32 & 55 & 19 \\
F & 13 & Ivailo Jordanov & Sporting Lisbon (Por) & 26 & 9 & 0 \\
F & 14 & Bontsjo Guentsjev & Ipswich (Eng) & 29 & 4 & 0 \\
F & 17 & Petar Mijtarski &  Pirin & 27 & 3 & 0 \\
F & 18 & Petar Aleksandrov & Levski Sofia & 32 & 26 & 5 \\
F & 21 & Velko Jotov & Espanyol (Spa) & 24 & 5 & 0 \\
F & 22 & Ivailo Andronov &  CSKA Sofia & 26 & 3 & 0 \\ \hline
\end{tabular}
\end{figure}
\section{WC 1994 Qualifying}
\begin{figure}[H]
\begin{tabular}{l r c l l}
1992 & & & & \\
May 14 & Finland & 0-3 & Bulgaria & Balakov 61', Kostadinov 70', 85' \\
Sep 9 & Bulgaria & 2-0 & France & Stoichkov (p) 21', Balakov 29' \\
Oct 7 & Sweden & 2-0 & Bulgaria & Dahlin 56', Petterson 76' \\
Dec 2 & Israel & 0-2 & Bulgaria & Sirakov 56', Penev 83' \\
1993 & & & & \\
Apr 14 & Austria & 3-1 & Bulgaria & Pfeifenberger 11', Kuehbauer 25', Polster 89'; Ivanov 53' \\
Apr 28 & Bulgaria & 2-0 & Finland & Stoichkov (p) 15', Yankov 43' \\
May 12 & Bulgaria & 2-2 & Israel & Stoichkov (p) 35', Sirakov 60'; R.Harazi 52', Rosenthal 53' \\
Sep 8 & Bulgaria & 1-1 & Sweden & Stoichkov (p) 21'; Dahlin 26' \\
Oct 13 & Bulgaria & 4-1 & Austria & Penev 6', 76', Stoichkov (p) 33', Letchkov 89;  Herzog 51' \\
Nov 17 & France & 1-2 & Bulgaria & Cantona 32'; Kostadinov 37', 90' \\
\end{tabular}
\end{figure}
Interesting to note that Stoichkov converted a penalty in \textit{all} Bulgaria's
home matches (and always between the 15th and 35th minutes).
\section{1994 Preparation}
\begin{figure}[H]
\begin{tabular}{l r c l}
Jan 19 & Bulgaria & 1-1 & Mexico \\
Apr 15 & Oman & 1-1 & Bulgaria \\
Apr 28 & Kuwait & 2-2 & Bulgaria \\
\end{tabular}
\end{figure}
\section{Preview}
The East Europeans have a dismal record in World Cup finals having failed to
secure one win in five tournaments, and were a minute away from missing out on 
the USA edition altogether. But with France seemingly through in the final 
qualifying match, Emil Kostadinov struck an awesome last-minute goal to lift 
Bulgaria to a 2-1 victory in Paris and a place in Group D alongside Argentina, 
Greece and Nigeria. ``We know how our rivals play and I think that we have a 
great chance to qualify for the next round with Argentina. The most dangerous 
team, who may stop us, is Nigeria, not Greece.''

Hristo Stoichkov is the team's star player, he moved from CSKA Sofia for 
\$4 million four years ago and has been one of the stars of the Spanish League 
ever since. The 28-year-old is Bulgaria's prince. He brings the experience of 
playing in four straight Spanish championship-winning teams at Barcelona, and 
is an inspiration to his Bulgarian teammates. Though volatile and argumentative, 
Stoichkov is the driving force in a strong Bulgarian forward line. However, he 
looked tired and jaded in Barca's Champions Cup final defeat by AC Milan on 
May 18 and Penev will be well-advised to let him have as long a rest as 
possible.

Liubosslav Penev, the coach's nephew, has been at Valencia since 1989 and 
flourished in the later qualifying rounds. He scored twice as Bulgaria came off
the ropes in the penultimate match against Austria to win 4-1 and recover from 
a 3-1 loss in Vienna. His achievments are all the more remarkable because, in 
January, he underwent an emergency operation after the discovery that he was 
suffering cancer of the testes.

Kostadinov hit both goals against the French, breaking a terrible streak of
seven qualifying games without hitting the target, never a problem for his
Portuguese club team, Porto.

Krasimir Balakov has also made his mark on the Portugese league and is Sporting 
Lisbon's leading scorer. The four top-line strikers have shown their overseas 
class and, significantly, all have more than 30 caps.

The defense is similarly strong, with Real Valladolid's Zlatko Yankov the 
pivotal man in the center. He missed the first two qualifiers but has not been
out of Penev's side since. Tifon Ivanov, who scored a rare goal in his debut in 
1990, has developed into a strong marker while Zanko Zvetanov, of Levski, has 
shown promising form in his brief international career.

The front-line is quick, the back-line suitably dour -- and much depends on 
Bulgaria's unreliable midfield providing a solid link. Experienced keeper Boris 
Mikhailov, who also captains the team can traditionally be relied upon to 
provide a solid last line. However, despite his 70 caps he has looked 
uncomfortable lately and the leap from playing in the French Second Division 
with FC Mulhouse to facing the cream of the world's strikers will not be 
straightforward.

The side has endured a difficult post-qualifying period, with players spread
throughout Europe and little funding available in a country with no strong
international pedigree. ``We don't have any problems psychologically, so getting 
our proficiency up is our task,'' Penev said. There is little pressure from home 
supporters. Getting to the World Cup is no new story; now it's up to Stoichkov 
to turn the page.
\section{The Coach and Key Players of Bulgaria}
Dimitar Penev, 45. Played in Bulgarian league for 15 years notching up 364
games. Capped 90 times for his country and played in 1966, 1970 and 1974 World
Cup finals. Won eight national titles as a player and took over Bulgarian
coaching job after stints with Dimitrovgrad, Kuwait and CSKA.

BORIS MIKHAILOV (Goalkeeper, FC Mullhouse(FRA)) 32 years old, 71 caps:
Steady and calm goalkeeper who inspires his teammates. Played nine of the 
qualifying games, missing one through illness, and has World Cup experience 
from Mexico 1986. Played for Bulgarian champions Levski before moving to France.

TRIFON IVANOV (Defender, Xamax Neuchatel(SWI)) 28 years old, 38 caps:
Hard playing, untiring sweeper with a fiery temper. Key man in last gasp 
qualifying win over France by keeping Jean-Pierre Papin out of the play.

IVAILO ANDRONOV (Midfield, CSKA) 26 years old, 2 caps: 
Second top goalscorer in Bulgarian league this season. Has had few 
international opportunities but desperate to unleash his eye for goal on the 
World Cup stage.

EMIL KOSTADINOV (Forward, Porto(POR)) 26 years old, 33 caps:
Career highlight were his two goals in Paris qualifying match which put 
Bulgaria into finals at the expense of France. Had a barren qualifying series 
before then, with just one other goal, but starred for Porto, who reached the 
European Champions Cup semifinal before losing to Barcelona.

HRISTO STOICHKOV (Forward, Barcelona(SP)) 28 years old, 41 caps:
World class striker ranked second best in Europe last season. Joined Barcelona 
for \$4 million from CSKA four seasons ago and has proved himself in European
competition. Clinical shooter, determined runner and undoubted star of the
Bulgarian lineup.
