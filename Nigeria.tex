\chapter{Nigeria}
\chapterprecis{Boye Olusanyo}
\newline
\newline
Coach: Clemens Westerhoff (from Netherlands)
\begin{figure}[H]
\begin{tabular}{c c l l c c c}
P & & & Club & Age & Caps & Goals \\ \hline
G & 1 & Peter Rufai & Go Ahead Eagles (Neth) & 30 & 17 & 0 \\
G &16 & Alloy Agu & RC Liege (Bel) & 27 & 51 & 0 \\
G & 22 & Wilfred Agbonabare & Rayo Vallenaco (Spa) &  28 & 10 & 0 \\ \hline
D & 2 & Augustine Eguavoen & Courtrai (Bel) & 29 & 33 & 1 \\
D & 3 & Benedict Iroha & Vitesse Arnhem (Neth) & 25 & 16 & 1 \\
D & 4 & Stephen Keshi (c) & RWD Molenbeek (Bel) & 32 & 57 & 5 \\
D & 5 & Uche Okechukwu & Fenerbahce (Tur) & 27 & 18 & 3 \\
D & 6 & Chidi Nwanu & Anderlecht (Bel) & 27 & 0 & 0 \\
D &13 & Emeka Ezeugo & Kispest Honved (Hun) & 28 & 0 & 0 \\
D &19 & Mike Emenalo & RWD Molenbeek (Bel) & 28 & 0 & 0 \\
D &20 & Uche Okafor & ACB Lagos & 28 & 27 & 1 \\ \hline
M & 8 & Thompson Oliha & Africa Sport (Ivo) & 26 & 24 & 3 \\
M & 10 & Augustine Okocha & Eintracht Frankfurt (Ger) & 21 & 7 & 1 \\
M & 12 & Samson Siasia & Nantes (Fra) & 27 & 38 & 13 \\
M & 15 & Sunday Oliseh & Liege (Bel) & 20 & 4 & 0 \\
M & 18 & Efan Ekoku & Norwich City (Eng) & 27 & 1 & 0 \\
M & 21 & Mutiu Adepoju & Racing Santander (Spa) & 23 & 24 & 5 \\ \hline
F & 7 & George Finidi & Ajax Amsterdam (Neth) & 23 & 21 & 3 \\
F & 9 & Rasheed Yekini & Vitoria Setubal (Por) & 31 & 46 & 29 \\
F & 11 & Emmanuel Amunike & Zamalek FC (Egy) & 24 & 7 & 1 \\
F & 14 & Daniel Amokachi &  Bruges (Bel) & 22 & 21 & 7 \\
F & 17 & Victor Ikpeba & AS Monaco (Fra) & 21 & 5 & 0 \\ \hline
\end{tabular}
\end{figure}
\section{Preview}
So near and yet so far has been the story of Nigerias various attempts in the 
last 25 years to qualify for a place in the World Cup finals. Finally we have 
made it but below are details of our numerous past near misses or not so near
as the case may be.

Nigeria is the most populous black nation in the world with a population of 88
million for whom football is in ``Shankly terms’’, a matter of life and death.

We were colonised by the British who introduced us to football. The game 
acquired some kind of structure after the Second World War with the 
introduction of the challenge cup, an equivalent of the English F.A. Cup. This
competition encouraged rivalry between the already diverse tribes in the 
country and ensured that football had a high profile.

Africa had no official representatives in the World Cup prior to 1970 (Egypt 
played a pre war World Cup without having to qualify). Qualification for the 
World Cup became an obsession in 1969 with the visit of Pele and Santos to play
some friendly matches. Nigeria got through to the latter stages of the 1970 
qualification series but lost out by a point to Morocco who went on to play in 
Mexico.

The arrival of television and images of that 1970 world cup stuck with Nigeria 
where the Brazilian team were revered as Gods of football. The intensity of the
obsession of qualification grew immensely after 1970 but Nigerian football was
in a state of decline for the next few years. In 1974 our arch rivals, Ghana 
beat us both at home and away in the second round of the qualification series.
Zaire went on to infamy in the finals in Germany.

A resurgence in Nigerian football took place around 1976. The FA was 
restructured and two powerful domestic teams IICC from the West and Rangers 
from the East emerged and produced a rivalry in football terms that set Nigeria
up for what has been 18 years of relative success. In 1975 Rangers lost in the 
final of the African Champions Cup to Hafia of Guinea but we were on our way. 
We qualified for the 1976 African Nations Cup with a highly passionate 3-1 
victory over Morocco in Lagos. The team then went on to the African finals in 
Ethiopia to take a highly creditable 3rd place.

This team went on to qualify for the final three way playoffs for the 1978 
World Cup finals in Argentina along with Egypt and Tunisia. In what I still 
term the best ever performance by the nation we took Egypt apart in Lagos, 4-0
in the first game. Our next two games took us away to Egypt where we lost 
disappointingly 3-1 and to Tunisia where we drew 0-0. The results of the 
matches between Tunisia and Egypt meant that in the final match in Lagos a draw 
would be sufficient to see us through. On a bitterly disappointing Saturaday in 
1977 we proceeded to play Tunisia off the park and fail to score. Their 
goalkeeper made some inspirational saves, and their only meaningful attack 
resulted in an own goal from our best defender and we were out. Images of tears 
shed in Lagos that day still stick in the mind. I have no doubt that the 1977 
side would have done us proud in Argentina amd that it is still the best side 
ever produced by the nation.

By the time the series of qualifiers for the 1982 finals came around, the 1978
team was two years past its sell by date. The team had gone on to throw away 
the African Nations Cup in 1978 in Accra by losing a semi-final against Uganda
when we were coasting in the game and to reach its pinnacle by winning the 1980
African Nations Cup played in Nigeria. Two professionals from England were 
added to the 1978 team , John Chiedozie then at Orient and later Tottenham and
Tunji Banjo also then at Orient. As fate would have it our first match was 
against Tunisia. We lost 2-0 away from home and in another fascinating game 
beat Tunisia 2-0 in Lagos in the return leg. We went through on penalties. A 
victory over Tanzania saw us play Algeria whom we had beaten in the finals of 
the African Nations Cup in Lagos 1 year earlier. This was for a place in the 
in the finals in Spain. The first leg was in Lagos and with the team past it's
sell by date a fast maturing Algeria won the tie by 2-0. The second leg in 
Algiers was a mere formality for the Algerians who won that as well and went on
to do Africa proud in Spain. Our only claim to fame was to have been coached by
Otto Gloria who was coach of the great Benfica team of the late 1960's.

A massive revamp of the team was needed. This was carried out but it took some 
time for the rewards to come. In the 1982 African Nations Cup in Libya we 
failed to qualify for the semi-finals for the first time in 6 years. We however
went on to qualify for the next series in Ivory Coast in 1984 and played very 
well getting through to the final and losing to a team that was to haunt us 
for the next 10 or so years in Cameroon. A massive blunder by the bureaucracy 
resulted in the bulk of the 1984 players being at odds with the establishment. 
They were then banned from the National side and this resulted in our worst 
ever performance in the World Cup qualifiers for 12 years. we lost in the 3rd 
round and did not make the final elimination series due to our dear friends 
from Tunisia.

In 1985 however a positive sign of things to come was observed when in the 
first ever under 17 world cup finals in China, Nigeria beat Germany 2-0 in the 
finals and were the inaugural champions. It was to take some seven years for 
this success to be fully translated to the senior side. We went on to qualify 
for the next series of the U17's in canada and to lose in the final on 
penalties to Russia. Our U21 side also came third in the U21 world cup finals 
in Russia losing to Brazil in a wonderful semi-final 2-0. Two years earlier we 
were privileged to witness a sight to befall many teams later when we played 
the Netherlands in the U21 finals in Mexico. Marco Van Basten was brilliant 
against us. We were knocked out in the first round without a win. For the first
time in 10 years however the senior side failed to qualify for the African 
Nations Cup losing out to Morocco. We however used the core of the U17 and U21
sides to qualify for 1988 African Nations Cup series where we went on to lose 
to Cameroon 1-0 in the final.

By the time the World Cup qualification series came around we seemed ready. We
ended up in a 4 team group with Gabon, Angola and Cameroon. We beat Cameroon 
2-0 in Lagos and seemed set on our way. A 2-2 draw in Angola after being 2-0 
down constituted a wonderful result in the circumstances. Howeever a 1-0 loss 
in Gabon proved fatal. We beat the Gabonese at home, and then the Angolans in a
match remembered for the death of Sam Okwaraji, a professional from Germany on
the football field at home. Our last game was to be away against Cameroon. We 
needed a draw to qualify , however panic set in and the current coach Clement 
Westerhoff was appointed for this game. We lost 1-0 and Cameroon went on to win
their final series of games and qualify along with Egypt for the finals in 
Italy. Had we defeated a poor Gabon away or even drawn we would have qualified.
Westerhoff stayed on and took us to the 1990 African Nations Cup finals in 
Algiers where despite losing our first game 5-1 to Algeria, we went on to lose 
1-0 to them in the final.

This time however we have made it. Winning our first round group from South 
Africa and Congo including a 4-0 demolition of South Africa in Lagos and a 0-0
draw away. From a second round group with Algeria and the Ivory Coast, 
qualifying despite a disappointing 2-1 defeat in Ivory coast having led 1-0 for
78 minutes by winning both home ties 4-1 and getting the required draw 1-1 in 
Algiers to qualify for the finals with five points along with Morocco and 
Cameroon.

Nigeria were also the only one of the three African World Cup qualifiers 
present in this year's African Nations Cup in Tunisia. The team impressed by
winning the competition, beating Zambia 2-1 in the final.

In 1991 our U17 team won the World Cup in Japan by beating Ghana 2-0 in the 
final.

Only two or three domestic league players will be on the plane to the USA this 
summer and none will get into the first team. This has its advantages as we 
will be playing in the world cup with players who have sampled real big game 
atmospheres and have played week in and week out against some of the best 
players in the world.
\section{Key Players}
RASHEED YEKINI (Forward, Vitoria Setubal(POR)):
Our most dangerous player and the proverbial one man forward line. He is now 
31, but his main assets are his pace and strength. He, however seems to need 
ten attempts to get a goal. His success or lack of it will determine how far 
Nigeria go in the tournament. Has been top scorer in the last two African
Nations Cups.

STEPHEN KESHI (Defender, RWD Molenbeek(BEL)): 
Now well past his best, but the only dependable Nigerian defender. His 
composure and experience on the pitch or the bench is invaluable. Half fit, he 
came in for our critical qualifying game against Ivory Coast in Lagos and 
steadied a fast sinking ship. Did not play most of the recent African Nations
Cup in Tunisia.

CHIDI NWANU (Defender, Anderlecht(BEL)):
Did not play the recent African Nations Cup due to a difference of opinion 
with the caoch but with Keshi struggling, is most definitely needed.

AUGUSTINE OKOCHA (Midfield, Eintracht Frankfurt(GER)): 
Our play maker has not been at his best for the last few months, but is 
crucial as he is the most creative player. Needs to rediscover his form for 
the World Cup and still only 22 years old.

MUTIU ADEPOJU (Midfield, Racing Santander(SPA)): 
As Okacha above, the country needs one of them playing well to have any impact 
in the finals.

DANIEL AMOKACHI (Forward, Bruges(BEL)): 
Has pace and more skill then Yekini but fails to score enough goals. If he can 
start scoring then the pressure on Yekini decreases. Did not score a goal in 
this year's African Nations Cup despite playing all the matches.

EMMANUEL AMUNIKE (Winger, Zamelak(EGY)), 
VICTOR IKPEBA (Winger, Monaco(FRA)), 
GEORGE FINIDI (Winger, Ajax(NETH)):
Our wingers and a key part of the Nigerian system. They have incredible pace 
but their crossing still leaves a lot to be desired. If they are on form, a 
quarter final place is not beyond the team.

The form of the above players as well as team spirit and relations with the 
Football Association (NFA) will determine how far the team goes. The current 
Dutch coach Westerhoff has been one match away from being sacked for almost a 
year and angry confrontations have taken place between government ministers who 
support and oppose him. He needed to win the recent African Cup of Nations to 
survive. Luckily for him they did.

Nigeria tend to play a 4-2-4 system with two wingers. This plays to their 
strengths of power and pace upfront but means they never control the midfield 
and thus the pace of the game. They, like Ireland tend to do a lot of running 
around and fitness will be very important. Against quality teams with 4 
midfield players, I fancy us to struggle but not many teams will prevent us 
from scoring. Our defence is hopeless and looks worse with very little cover 
in front of it. This is one team that is not going to win games 1-0.
\section{1994 Preparation}
\begin{figure}[H]
\begin{tabular}{l l r c l}
Mar 9 & Lagos & Nigeria & 0-0 & Ghana \\
Mar 26 & Tunisia & Nigeria & 3-0 & Gabon \\
Mar 30 & Tunisia & Nigeria & 0-0 & Egypt \\
Apr 2 & Tunisia & Nigeria & 2-0 & Zaire \\
Apr 6 & Tunisia & Nigeria & 2-2 & Ivory Coast \\
Apr 10 & Tunisia & Nigeria & 2-1 & Zambia \\
Apr 17 & Armenia & Colombia & 1-0 & Nigeria \\
May 5 & Stockholm & Sweden & 3-1 & Nigeria \\
May 25 & Bucharest & Romania & 2-0 & Nigeria \\
\end{tabular}
\end{figure}
