\chapter{Italy}
\chapterprecis{Loris Magnani}
\newline
\newline
Coach: Arrigo Sacchi (former coach of AC Milan)
\begin{figure}[H]
\begin{tabular}{c c l l c c c}
P & & & Club & Age & Caps & Goals \\ \hline
G & 1 & Gianluca Pagliuca & Sampdoria & 27 & 17 & 0 \\
G & 12 & Luca Marchegiani & Lazio & 28 & 5 & 0 \\
G & 22 & Luca Bucci & Parma & 25 & 0 & 0 \\ \hline
D & 2 & Luigi Apolloni & Parma & 27 & 13 & 0 \\
D & 3 & Antonio Benarrivo & Parma & 25 & 7 & 0 \\
D & 4 & Alessandro Costacurta & AC Milan & 28 & 19 & 1 \\
D & 5 & Paolo Maldini & AC Milan & 26 & 51 & 2 \\
D & 6 & Franco Baresi (c) & AC Milan & 34 & 76 & 1 \\
D & 7 & Lorenzo Minotti & Parma & 27 & 2 & 0 \\
D & 8 & Roberto Mussi & Torino & 30 & 1 & 0 \\
D & 9 & Mauro Tassotti & AC Milan & 34 & 4 & 0 \\ \hline
M &11 & Demetrio Albertini & AC Milan & 22 & 14 & 0 \\
M &13 & Dino Baggio & Juventus & 23 & 12 & 4 \\
M &14 & Nicola Berti & Inter Milan & 27 & 25 & 3 \\
M &15 & Antonio Conte & Juventus & 24 & 1 & 0 \\
M &16 & Roberto Donadoni & AC Milan & 30 & 50 & 5 \\
M &17 & Alberigo Evani & Sampdoria & 30 & 9 & 0 \\ \hline
F & 10 & Roberto Baggio & Juventus & 27 & 35 & 19 \\
F & 18 & Pier Luigi Casiraghi & Lazio & 25 & 16 & 4 \\
F & 19 & Daniele Massaro & AC Milan & 23 & 7 & 0 \\
F & 20 & Giuseppe Signori & Lazio & 26 & 15 & 6 \\
F & 21 & Gianfranco Zola & Parma & 27 & 6 & 0 \\ \hline
\end{tabular}
\end{figure}
\section{History}
Italy did not attend the 1930 World Cup along with several other European 
powers.  However, on Oct. 8, 1932 the Italian delegation convinced FIFA to  
assign the 1934 World Cup to Italy. Italy had just recently won the first
edition of the International Cup (a round-robin tournament between Austria,
Czechoslovakia, Hungary, Italy, and Switzerland) and in 1929 the Italian
First Division (Serie A) had begun play in the home-away format (girone
unico) which is still in use throughout most of Europe. Italy continued the
tradition established by Uruguay by winning the World Cup as hosts. They
easily trounced the U.S. 7-1 with a hat-trick by Schiavio and then proceeded
to eliminate Spain in 2 epic contests. The first game stood at 1-1 after
regulation and the overtime periods did not change the score so that a 
replay the next day was necessary (no PKs here!). Italy won the replay 1-0 
thanks to a goal by the legendary Giuseppe Meazza (the San Siro stadium in 
Milan was renamed in his honor years ago). In the semis, Italy met its \textit{bete 
noire}, Austria, a team that had consistently had the better of the Italians 
both in Austria and in Italy. Austria had beaten a tough Hungarian team 2-1 to 
advance to the semis. A controversial goal by Guaita (one of Italy's oriundi - 
foreign born players who were considered Italians because one of their parents 
was Italian) gave Italy a narrow 1-0 victory and access to the finals. There 
they would meet Czechoslovakia who had eliminated Germany 3-1. The final was 
played in front of Benito Mussolini and a packed house in Rome. Things looked 
grim for Italy at the 70' when a goal by Puc put Czechoslovakia ahead, but Orsi
(another Argentinian of Italian descent) tied the game 10 minutes later. In 
extra-time, Schiavio gave Italy the victory and the World Cup, 2-1.

The 1938 World Cup held in France saw Italy qualifying as defending champions.
A grim atmosphere hung over Europe foreshadowing the war which was to come.
Italy's first game featured the amateurs from Norway who were lightly
regarded by both the italian players and the press. Italy scored in the second 
minute and a rout seemed to be at hand. However, Norway tied the game in the 
83rd and actually scored a goal two minutes later which was called back because
of offside! The game went to overtime where Piola gave Italy a very laboured 
2-1 victory. The defence of the trophy won in 1934 was off to a very bad start.
Things did not look better in the next game with the home side, France. Again, 
Italy took a quick lead but France equalized a minute later. With the score 1-1
at halftime and Italy playing lacklustre football there was the hint of 
elimination in the air. However, Italy came out strongly in the second half and
two goals by Piola at the 52' and 72' rounded out a deserved 3-1 victory over 
the hosts. Italy moved on to the semifinals against mighty Brasil who had 
eliminated Czechoslovakia by the score of 2-1 in a replay game after the first
game had ended 1-1. After a scoreless first half, goals by Colaussi (55') and 
Meazza (60') gave Italy a 2-0 lead which held until a penalty kick by Romeu at 
the 87' closed out the score at 2-1.  Italy would face Hungary in the final,
since the latter had easily disposed of Sweden by a 5-1 score. The game began 
with a bang as Colaussi scored for Italy on the 5' mark only to see Titkos 
equalize two minutes later. A goal by Piola on the 16' mark put Italy ahead by 
2-1 and the Italians kept pressing and scored a third goal in the 35th minute 
again by Colaussi. Midway through the second half the Hungarians were brought 
back into the game by Sarosi only to have Piola close out the scoring in the 
82' with his second goal of the final and 5th overall. Despite the impressive 
total, Piola would finish second to the great Brasilian Leonidas who finished 
with 8. The 4-2 victory over Hungary gave Italy its second title. The war would
prevent all of the Italian players from the 1938 squad from defending the title
and a brand new Italian team would go to Brasil in 1950 for the 4th World Cup.

Although the postwar recovery was slow in Italy, a sporting disaster pretty 
much sealed the fate of the 1950 Italian expedition and overshadowed any social
and economic difficulties as far as the Azzurri were concerned.  The entire
Torino team, which had won 5 scudetti in a row and who formed the backbone of 
the Italian National Team was wiped out in a horrible plane crash in the spring
of 1949. The great Torino team from that period (il Grande Torino) often formed
9/11 of the Azzurri and included the legendary Valentino Mazzola (his son, 
Sandro Mazzola, was also one of the greats of Italian soccer, though probably 
not at the level of Valentino). The repercussions of that crash lasted many 
years. One of the immediate consequences was that the team travelled to Brasil 
by ship in 1950 rather than by plane; the two weeks at sea did nothing to help 
the physical condition of the Azzurri. The format of the Cup had been changed 
from head-to-head elimination to a first round involving a group of teams which
would play a round-robin with the group leader advancing. Italy's group 
consisted of Sweden and Paraguay and Italy seemed the clear favorite. However, 
an initial loss to Sweden by 3-2 made a 2-0 victory over Paraguay meaningless.
Sweden advanced to the next round and Italy's participation in the World Cup 
was over almost before it began. The dark days for the Azzurri had begun and 
would last two decades as far as the World Cup was concerned.

The 1954 World Cup involved a qualification match for Italy. The only other 
team in Italy's group was Egypt who were eliminated easily by scores of 1-2 in 
Cairo and 5-1 in Milan. The final round was in neighboring Switzerland and 
Italy's group consisted of England, Switzerland, and Belgium.  However the 
formula for advancement to the second round did not provide (strangely) for a 
game with each opponent, but rather, with a game against only two opponents. 
Italy's opponents would be Switzerland and Belgium, and a straightforward 
qualification to the second round was expected.  It was not to be. Switzerland 
actually beat Italy in the first game by a 2-1 score and lost to England by 2-0.
By beating Belgium 4-1, Italy would meet Switzerland in a playoff to see who 
would join England in the second round. The Italians must have been fairly 
confident of getting through so the Swiss victory by a heavy 4-1 margin was 
surely a shock. The 1950s were proving to be the Azzurri's lost decade as the 
1958 World Cup would dramatically indicate.

In fact, Italy did not even qualify for the 1958 final round in Sweden. In a 
relatively ``easy’’ group with Northern Ireland and Portugal, Italy managed to 
lose heavily to Portugal in Lisbon by 3-0 and to Northern Ireland in Belfast by
2-1. The victories in Rome and Milan against these two opponents were to no 
avail as Northern Ireland's 1-1 tie in the first game of the group with 
Portugal in Lisbon was the difference. The only comfort for Italy was that the 
Swedish team which finished second fielded several players who would play in 
the Serie A, and even the victors, Brasil, played with Dino Sani and Jose 
Altafini who would spend many seasons in Serie A.

Italy's qualification for the 1962 World Cup was straightforward. Romania were 
in the group but withdrew, leaving Italy to easily eliminate Israel by 2-4 and 
6-0 scores. The Italian team was, on paper, the strongest since the days of the
Grande Torino team from more than a decade before. With players like Bulgarelli, 
Cesare Maldini (the father of Paolo Maldini), Rivera and the two oriundi, Jose 
Altafini (who was born in Brasil and actually played for the 1958 World 
Champions!), and Omar Sivori (who was born in Argentina and played many games 
for the albicelestes), Italy looked like one of the favorites. The team 
performed only on paper, however. A group composed of West Germany, Chile (the 
hosts), and Switzerland looked fairly difficult but not impossible. Impossible 
it was to be. A lacklustre 0-0 draw with the Germans led to the infamous Chile-
Italy game. A hostile crowd and an incompetent refereeing job by the Englishman,
Aston, led Italy to defeat. Italy finished with 9 men due to the gutlessness of 
Aston, who dared not send off the Chileans despite their endless provocations 
and intimidation. However, he did not hesitate to send off the Italian, Maschio,
whose ``foul’’ was that his nose connected with Leonel Sanchez' fist (Maschio's 
nose was broken by the blow but he could nurse his injury in the locker room 
thanks to the red card he received...). The subsequent 3-0 Italian victory over
the Swiss was no consolation. Italy went home, feeling frustrated and cheated.
But the most infamous Italian World Cup performance was to come...

Italy's qualification for the 1966 World Cup was not as straightforward as 
usual. In a group with Scotland, Poland, and Finland, a loss to Scotland in 
Glasgow by 1-0 forced Italy to win or tie the last game against the Scots in 
Naples. They won handily in Naples by 3-0 and went into the finals in England 
with one of their most talented post-war teams. Their roster reads like a Who's
Who of Italian soccer: Albertosi, Fachetti, Bulgarelli, Rivera, Mazzola, 
Lodetti, Burgnich.  An All-Star team which would win the European Championship 
two years later in 1968. The group Italy was in seemed easy, consisting of the 
USSR, Chile, and the ``weak-sister’’ North Korea.  With two teams qualifying for 
the second round, the USSR and Italy appeared certain to advance. The first 
game against Chile was an easy 2-0 victory giving Italy some revenge for what 
transpired four years earlier. A difficult game against the Soviets in which 
Italy squandered many chances (or were stopped by the great Yashin) resulted in
a 1-0 loss to a goal on the 58' mark by Cislenko. However, the qualification 
seemed to be at hand since Italy only required a draw with North Korea. What 
should have been an easy game started poorly as Bulgarelli had to leave almost 
immediately due to the aggravation of a previous injury. In those days there 
were no substitutions so Italy had to play the remainder of the game with 10 
men. Italy dominated play throughout the game squandering a plethora of 
opportunities, but things took a distinct ominous turn when Pak Doo Ik scored a
goal on a quick counterattack in the 41'. As Paul Gardner writes in his book, 
The Simplest Game, Italy had many chances to equalize in the second half, 
``…but were destined to miss them all’’…And so the Darkest Day in Italian 
soccer ended with a 1-0 loss and elimination at the hands of North Korea. The 
return trip to Italy was quite interesting. The plane on which the Azzurri were
returning home had to be diverted from Milan to Genova because an angry mob had
gathered at the Milan airport. Enough fans found out about the change of 
itinerary to greet the Azzurri at Genova with a dense hail of rotten fruit and 
vegetables...In Italy, when you want to talk about a disastrous defeat, you do 
not invoke Waterloo, but rather, Korea.

The 1970 World Cup in Mexico brought an even stronger Italian team which had 
qualified easily in a group with Wales and East Germany. The Italian hopes 
rested on their great defense, anchored by Alberosi, Fachetti and Burgnich, a 
``\textit{stafetta}'' between Mazzola and Rivera for the \#10 shirt (i.e., one player would
play each half and substitute the other), and the great Italian bomber, Gigi 
Riva (still the alltime leader in goals scored for the national team). Italy's 
group was a bit difficult with Sweden and Uruguay as the main adversaries and 
lowly Israel bringing up the rear. This group produced some of the worst 
football seen at that glorious, offensive-minded World Cup. In 6 games there 
were only 6 goals scored...and Italy's brand of defensive ``\textit{catenaccio}'' yielded 
a 1-0 win against Sweden and two 0-0 draws with Uruguay and Israel. Italy won 
it's group by scoring only one goal and went into the second round to meet the 
hosts, Mexico, at the Azteca Stadium in front of 100,000 plus Mexicans. Those 
who feared another Italian quick-exit from the WC could be forgiven, but this 
Azzurri team had a bit more mettle than their predecessors. With a fabulous 
performance by Riva and Rivera, Italy trounced Mexico 4-1 to arrive at the 
semi-finals against the fearsome Germans who had eliminated England after being
down 2-0. What would follow would be one of the greatest games in World Cup
history. The German team featured many players who would win the WC four years 
later, Maier, Vogts, Beckenbauer, Muller, Overath...Truly a magnificent squad. 
The Italian team took an early lead with a goal by Boninsegna in the seventh 
minute and grimly held on behind their \textit{catenaccio} defense. The game seemed to 
be heading towards an ugly 1-0 Italian victory when in injury time, the great 
German fullback Karlheinz Schnellinger (who played for AC Milan) scored the 
equalizer in the 92nd minute. The game would go into extra time. The Germans 
wasted no time and Muller scored in the 94'. Burgnich and Riva scored two goals
for Italy for a temporary 3-2 lead, but Muller once again equalized in the 109'.
All seemed lost for Italy, when Rivera (who had been somewhat responsible for 
the Muller equalizer) scored the match-winner one minute later. Five goals had 
been scored in extra time and Italy would go through to the final game after a 
monumental 4-3 victory over the 1966 runners-up and the eventual 1974 winners. 
Italy's opponents in the Final, unfortunately for them, were the great 
Brasilian team of 1970 led by Pele, Tostao, Jairzinho, and too many others to 
mention. Certainly one of the greatest football teams of all time. There really
wasn't much to the final. Brasil won, going away, by a 4-1 score after the 
match stood at 1-1 at halftime. The only recriminations the Italians had was 
that at the beginning of the second half (when the score was 1-1) Rivera was 
not brought on to substitute Mazzola. He was finally let onto the pitch with 
only six minutes to go and the score a disheartening 3-1. Could Rivera have 
prevented the Brasilian victory ? Probably not, but his ``non-use’’ in the final 
caused fury among his supporters. Despite Italy's second place finish, the 
Italian fans were incensed with the Azzurri and only Rivera managed to ``save’’ 
himself from the wrath of the fans and the press.

Italy entered the final stages of the 1974 World Cup rather easily, eliminating
Luxembourg, Switzerland, and Turkey. Italy, along with Germany and Brasil were 
considered one of the favourites given their streak of 12 games without 
conceding a goal. Included in that streak were 2 victories over England, 
including one at Wembley, and 1 over Brasil. With stars such as Zoff, 
Facchetti, Benetti, Burgnich, Mazzola, Capello, Chinaglia, Rivera, and Riva, 
surely Italy would reach the final four. The dream ended immediately. First, 
Zoff's unbeaten streak was ended by Haiti, who took a 1-0 lead before losing 
3-1. Then a rather fortunate tie with Argentina (1-1), led to the game with the
surprising Polish team who had won their first two games. All that both teams 
needed to advance was a draw, but the Poles would have none of that and by 
halftime they were winning 2-0. A late goal by Capello made the score 
respectable, but after the success of 1970, Italy was once again faced with 
early elimination, and the by now traditional welcome home of rotten fruits and 
vegetables, and insults.

With a forty year streak of unsuccessful performances (with the possible 
exception of the 1970 tournament), Italy faced a brutal qualification group with
England as the rival to be beaten for the only spot in the final 16 of the 1978
Cup to be held in Argentina. The other members of the group were Luxembourg and
Finland and it turned out to be the English and Italian performance against 
these ``minnows'' which decided who would advance since Italy and England had 
won 2-0 at home against the other. On the strength of a 6-1 thrashing of the 
Finns in Turin, Italy squeaked through on goal difference, 18-4 against 15-4 
for England. Their misfortune seemed to continue when they were paired up with 
Argentina, France, and Hungary in Group 1 of the Final Tournament. A string of 
poor performances before the World Cup had turned the press against Bearzot who
was bringing two new faces to the Azzurri, a handsome left fullback by the name
of Cabrini and a small striker by the name of Paolo Rossi. The Azzurri seemed 
to be heading for another early exit when France took a 1-0 lead in the first 
few seconds of play of the first game, but a gutsy Italian performance produced
the 2-1 victory with the tying goal scored by Paolo Rossi. Rossi continued his
magic against Hungary as Italy won 3-1. Since both Italy and Argentina had
qualified, their last game in the group was considered academic except that the
loser would end up in the second round in the same group as Brazil.. a fate to 
be avoided at all costs. Despite this incentive, and playing in front of a 
fanatical crowd in Buenos Aires, Argentina went down to defeat 1-0 by a 
brilliant goal from Bettega off a pass from Rossi. Italy would go into the 
second round in a ``European'' group with Germany, Holland, and Austria while 
Argentina would have to confront its destiny with Brasil, Poland, and Peru. 
Italy opened with a tactical 0-0 draw with Germany with each side preferring
to avoid exposing themselves. A 1-0 victory over Austria with a goal by Rossi
(his third of the campaign) placed Italy in the uncomfortable position of
having to beat Holland to reach the finals. A tie would put the Dutch through.
The game began with a score for Italy as a desperate Brandts slid a ball away 
from Bettega, but unfortunately for the former, into his own net. This was the 
halftime score. The Italian coach, Enzo Bearzot, then committed a colossal 
blunder. The Italian playmaker, Franco Causio, was replaced by Claudio Sala 
(ostensibly to rest Causio for the final...). Such arrogance is often punished 
and the wrath of the gods was made clear: First by  Brandts in the 50' with a 
shot from outside the box and then by an impossible shot by Haan from almost 
40 metres...The brilliant italian team found itself playing only for third 
place against Brasil. This game would also be lost by a 2-1 score with both 
goals again coming from outside the Italian goal area. The unexpected 
performance and brilliant play shown by Italy for once caught the fancy of 
their fans who welcomed the fourth-place Italians home (something which had 
been denied to the finalists of 1970).

The 1982 World Cup in Spain started a bit inauspiciously for the Italian squad 
which was a virtual carbon copy of the 1978 team except that Bettega was 
injured and replaced by Graziani, and the great Italian midfielder, Antognoni 
could play (he had been injured in 1978). The qualification had been rather 
difficult since Italy were in a group with Yugoslavia, Denmark, Greece, and 
Luxembourg with the top two qualifying. Italy started strongly with 4 straight 
2-0 victories and then limped to a second place finish behind Yugoslavia. The 
pre-World Cup exhibitions were much worse. Italy lost to France, East Germany, 
and drew with Switzerland. Some members of the Italian press even suggested 
that Italy should stay at home and save themselves from embarrassment. These 
naysayers seemed to be right as Italy qualified for the second round on goal 
difference over Cameroon (the other two teams in the group were Peru and Poland,
with the latter qualifying in first place). When Italy found themselves in the 
second round in a group with Argentina and the favourites, Brasil, a quick exit
seemed certain. In the first game against Argentina, a continuous barrage of 
fouls by the not-so-gentle Gentile stopped Maradona and Italy scraped by in a 
nasty game by a 2-1 margin. Since Brasil had beaten Argentina 3-1, the last
game between Italy and Brasil would decide a place in the semifinals with
the Brasilians needing only a tie. This game would mark the awakening of 
``Pablito'', Paolo Rossi, who with a memorable hat trick sunk the mighty 
Brasilians 3-2. Italy would play Poland for a place in the final. Once again, a
surging Rossi scored two goals for a 2-0 victory and a showdown in the final 
against West Germany who had eliminated France on penalties in their semi final.
The game began poorly with the Italians playing very defensively and actually
missing an early penalty kick. With the score 0-0 at halftime, the outcome 
seemed wide-open. Early in the second half with Italy gaining confidence, Paolo
Rossi struck again with his sixth consecutive Italian goal, and the final 
turned Italy's way. A score by Tardelli in the 68' and a late goal by Altobelli
at the 81' sealed Germany's fate with a late goal by Breitner only a consolation
for the Germans. Italy had won its third World Cup, equalling Brasil's total,
with a 3-1 victory over the Germans.  The heroes for Italy, in addition to the 
leading goal scorer of the finals, Rossi, were Antognioni (who did not play in 
the final because of injury), Cabrini, Zoff, Tardelli, and Paolo Conti who many
considered the best player of the tournament.

Italy's glory was to be short lived. In the very first game after they had won 
the World Cup, a friendly against Switzerland in Rome, Italy lost 1-0. They 
failed to qualify for the 1984 European Championship and coach Bearzot's 
insistence on relying on his trusty veterans would prove to be his undoing at 
the 1986 World Cup. As titleholders, Italy qualified automatically along with 
the hosts Mexico. Their first group consisting of Argentina, Bulgaria, and 
South Korea was not difficult since two teams would qualify and, as expected, 
both Italy and Argentina went to the second round. While Maradona would lead 
Argentina to the final, Italy's dreams of repeating were dashed in the first 
game of the second round as France soundly beat them 2-0. The Mexican adventure 
was over and Bearzot would be replaced by Vicini and a new Italian team would 
attempt to win the 1990 World Cup which was to be held in their own homeland.

Qualifying automatically as hosts, the Italian team featured star players like 
Zenga in goal, Franco Baresi as sweeper, and other stars such as Maldini, 
Giannini, Roberto Baggio, and the great hope, striker Luca Vialli. It was not 
to be Vialli's World Cup, however, as a relatively unknown forward by the name 
of Toto Schillaci would steal the limelight from the Sampdoria forward. Since 
Italy had a strong team and were playing at home, a victory in the World Cup 
final was the only result which would be acceptable. Italy's group was very 
easy, consisting of Austria, the United States and Czechoslovakia. Despite a 
few problems against both Austria and the U.S., Italy went into the second 
round with 3 victories, 4 goals for and 0 against. The first opponent in the
second round was Uruguay, but goals by Schillaci and Serena eliminated the 
South Americans and brought Italy to the quarterfinals against Eire. A difficult 
game was resolved again by a goal from Schillaci and Italy would meet a 
lacklustre Argentina in Naples in one of the two semifinals. Italy started out 
well with another goal by Schillaci, but in the second half the team started 
to disintegrate. Maradona was looking ever more dangerous and finally in the 67' 
Caniggia would end Zenga's unbeaten streak and tie the game at 1-1. A tired 
Italy lost the iniative to a growing Argentina in extra time and when the game 
went to penalty kicks, the dream ended. Missed kicks by Donadoni and Serena 
(both experts in the practice) doomed Italy to elimination and relegated them 
to the third place game with England who had also been eliminated by the Germans 
on penalty kicks. The meaningless 2-1 victory over the English was no 
consolation to the Italians. Although there was no rotten fruit this time 
(perhaps the fans were finally maturing) the third place finish was not enough.
The 1990 World Cup was yet another Italian failure.
\section{WC 94 Qualification}
\begin{figure}[H]
\begin{tabular}{l c c c c c c c}
European Group 1 & GP & W & D & L & GF & GA & Pts \\ \hline
\textbf{Italy} & 10 & 7 & 2 & 1 & 24 & 8 & 16 \\
\textbf{Switzerland} & 10 & 6 & 3 & 1 & 23 & 6 & 15 \\
Portugal & 10 & 6 & 2 & 2 & 19 & 7 & 14 \\
Scotland & 10 & 4 & 3 & 3 & 14 & 13 & 11 \\
Malta & 10 & 1 & 1 & 8 & 3 & 23 & 3 \\
Estonia & 10 & 0 & 1 & 9 & 1 & 27 & 1 \\ \hline
\end{tabular}
\end{figure}
Although they finished first, the qualification was bitterly difficult for
Sacchi and the Azzurri with the issue in doubt even on the final day. The 
Swiss caused the most problems for the Italians getting off to a strong start
by destroying Estonia in Tallin by 6-0 and beating Scotland in Switzerland by
3-1. Despite this, Italy was very confident as they faced the Swiss in Cagliari
in late 1992. Two quick goals by the Swiss forward Chapuisat after two hideous 
breakdowns in the Italian defence shattered that confidence. A late goal by 
Baggio brought Italy back in the game at 2-1 and a miraculous shot by Eranio in
injury time brought Italy level. But the 2-2 draw at home meant that the 
qualification would be a bit more difficult than Sacchi and the Italian press 
had anticipated. An away trip to Scotland brought a 0-0 deadlock to the general
satisfaction of everyone and the upcoming visit to Malta would surely provide 
the necessary goals to adjust the goal difference which had begun to tilt 
heavily in Switzerland's favour. In arguably the worst performance under Sacchi,
Italy only managed a 2-1 victory in Malta, and Pagliuca had to save it by 
stopping a penalty kick!  The criticism arising in Italy was becoming 
threatening and Sacchi was reacting poorly to it. In this atmosphere of crisis 
the impending match in Lisbon against the dangerous Portuguese loomed 
terrifyingly for the Azzurri. However, a wonderful performance by Casiraghi and
the two Baggios gave Italy a rather unexpected 1-3 win away from home. The 
cloud which had formed around Sacchi seemed to dispel after a 6-1 trouncing of 
Malta in Italy and a very easy 2-0 victory over Estonia where the Estonian 
goalie prevented a repeat of the hammering given to Malta.  The upcoming May 1,
1993 meeting against the Swiss was viewed as an opportunity to avenge the 
embarrassing tie in Sardinia. It was not to be. Italy played a marvellous first
half and seemed on the verge of scoring several times. However, the Swiss held 
on and at the end of the half a harsh decision by the Spanish referee to red-
card Dino Baggio for a nasty foul in midfield left Italy facing the second half
with 10 men. An early Swiss goal could not be answered and Italy went down to 
their first defeat in the group. With the games against the other two contenders,
Scotland and Portugal at home, Italy's situation was tense but not desperate 
especially since the Swiss had to visit both Scotland and Portugal. In the fall
of 1993 an easy victory against Estonia coincided with a problematic 1-1 draw 
between Scotland and Switzerland. That result all but eliminated the Scots but 
it projected the Swiss into a very favourable position to obtain one of the two
slots for America. Italy was placed in a situation of having to win its last 
two games against Scotland and Portugal to ensure its qualification. The game 
against the Scots was a very good Italian performance with a victory for the 
Azzurri by a score of 3-1. However, the surprising 1-0 victory by the Portuguese
over the group leaders, Switzerland, complicated matters terribly for Italy.
The situation at this point was the following: The Swiss could assure themselves
of a berth by beating Estonia at home in their last game. The Italians could 
assure themselves of a berth by beating Portugal in Italy in their last game...
but the Portuguese had a game in hand againt Estonia. Given the goal difference,
a 4 goal or greater margin of victory by Portugal over Estonia would allow the
former to TIE their last game in Italy and advance on superior goal difference.
Italy and Sacchi were sweating because while very few people envisioned a 
Portuguese victory in Italy, a tie was another matter entirely. When the 
crucial game against Estonia arrived, in Lisbon, Portugal failed to get the 
necessary 4 goal margin, winning ``only’’ by 3-0.  A draw in the final game would
put the Italians through... On Nov. 17th, Portugal met Italy in a packed San 
Siro stadium. Italy played well and managed a 1-0 victory and qualification with
a late goal by Dino Baggio. Up until that goal (in the 83') the spectre of a 
late Portuguese goal kept Italian fans sweating. However, finally, the 
qualification had been confimred and despite the difficulty, Italy had finished
first as predicted. The real problem, however, was that the various 
unconvincing Italian performances against the Swiss and Malta had turned the 
difficult Italian press against Sacchi. The exhibition games in the spring of 
1994 only exacerbated this trend: a home loss to France 0-1, a defeat by 
Germany in Stuttgart (2-1), and an amazing loss to Pontedera, a semipro team 
in the Italian fourth division have convinced most of Italy that Sacchi is 
leading the Azzurri to yet another World Cup humiliation. Whether this is so, 
or not, will be seen in the next month.
\section{Key Players}
IANLUCA PAGLIUCA (Goalkeeper, Sampdoria):
Italy's starting goalkeeper. An excellent goalie who is very good in playing 
with a zone defense in front of him. Has some trouble with high balls and with 
set pieces but can be spectacular at times. A bit more inconsistent than Zenga 
and Zoff, he is nevertheless the best goalie in Italy at the moment.

FRANCO BARESI (Sweeper, AC Milan):
The backbone of the Italian defense and one of the greatest sweepers ever. Not
quite at the level of a Beckenbauer or a Scirea even at his prime, he was 
certainly on the next level with people like Krol, Passarella, etc. 
Unfortunately, age has taken its toll on Baresi (he's 34!) and he has lost a 
few steps. He makes up for his diminished speed by his tremendous experience 
which manifests itself as an uncanny ability to be in the right place at the 
right time. He is prone to fouling since he is a rather rough player and may 
have to sit out a game for accumulation of yellows or even a red card. He will 
command the Italian offside trap which will be sprung often. He has won 
practically every trophy imaginable and was part of the 1982 Italian World Cup
squad though he did not play. He will be the Captain of the Azzurri.

PAOLO MALDINI (Defender, AC Milan):
The starting left fullback and one of the true talents of Italian soccer.
Certainly one of the top three left fullbacks in the world. Can do it all, 
dribble, defend, score, set up goals. When the occasion presents itself he will
push down the left side and create opportunities for the attack. Has tremendous
international experience (50+ caps and still only 26) and should be one of the 
stars of the 1994 World Cup.

DEMETRIO ALBERTINI (Midfield, AC Milan):
Counted on to play in front of the back four and act as a first line of defense
and start up the distribution of ball to the midfield and forwards. Although he
has a lot of experience despite his young age, he is prone to inconsistency and
will sometimes have truly bad games (such as the recent friendly against 
Germany). He is one of the keys to the Azzurri's success and he will have to be
at the top of his form if they are to advance. Has a potent long-range shot.

NICOLA BERTI (Midfield, Inter):
A wonderful addition to the Italian midfield. Has recovered recently from a 
nasty injury and seems to be in fine form. An attacking midfielder who is known
for his courage and intensity. Could provide the emotional and physical spark 
that Italy needs in the midfield (sort of like Tardelli used to). If he has a 
good World Cup, Italy should go far.

ROBERTO BAGGIO (Forward, Juventus):
Not much to say. The FIFA and the European Player of the Year. Has enough 
talent to make the 1994 World Cup ``His'', just like Maradona in 1986 and Pel{\'e} 
in 1970...Although his talent is unquestionable, his drive and intensity have 
been often questioned and are the only Achilles' heel for Baggio. He can be 
marked out of a game either physically or psychologically. He is not a leader 
the way Maradona or Pele were and it may cost him (and Italy) dearly. 
Everything else he has. If he catches fire, there is no-one like him in the 
World today...

Sacchi will manage to blend the talent into a real team and Italy will be in 
the final four. Once you reach the semis you need a bit of luck to win it all 
and so I won't guess any further, but I do expect to see Italy anywhere in the 
first four slots.
\section{1994 Preparation}
\begin{figure}[H]
\begin{tabular}{l l r c l}
Feb 16 & H & Italy & 0-1 & France \\
Mar 23 & A & Germany & 1-2 & Italy \\
May 27 & H & Italy & 2-0 & Finland \\
Jun 3 & H & Italy & 1-0 & Switzerland \\
Jun 10 & USA & Italy & 1-0 & Costa Rica \\
\end{tabular}
\end{figure}
