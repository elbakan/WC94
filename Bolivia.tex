\chapter{Bolivia}
\chapterprecis{Rodrigo Sanchez}
\newline
\newline
Coach: Xavier Azkargorta
\begin{figure}[H]
\begin{tabular}{c c l l c c c}
P & & & Club & Age & Caps & Goals \\ \hline
G & 1 & Carlos Trucco & Bolivar &  35 & 20 & 0 \\
G & 12 & Dario Rojas & Oriente Petrolero & 34 & 13 & 0 \\
G & 19 & Marcelo Torrico & The Strongest & 22 & 1 & 0 \\ \hline
D & 2 & Juan Manuel Pe{\~n}a & Santa Fe & 21 & 18 & 1 \\
D & 3 & Marco Sandy & Bolivar & 22 & 21 & 2 \\
D & 4 & Miguel Angel Rimba & Bolivar & 26 & 29 & 0 \\
D & 5 & Gustavo Quinteros & The Strongest & 29 & 19 & 1 \\
D & 13 & Modesto Soruco & Blooming & 28 & 18 & 0 \\
D & 16 & Luis Cristaldo & Bolivar & 24 & 23 & 1 \\
D & 17 & Oscar Sanchez & The Strongest & 32 & 3 & 0 \\ \hline
M & 6 & Carlos Borja & Bolivar & 36 & 72 & 1 \\
M & 7 & Mario Pinedo & Oriente Petrolero & 30 & 16 & 2 \\
M & 8 & Jose Milton Melgar & The Strongest & 34 & 59 & 5 \\
M & 14 & Mauricio Ramos & Destroyers & 24 & 7 & 0 \\
M & 15 & Vladimir Soria & Bolivar & 25 & 11 & 0 \\
M & 20 & Ivan Ramiro Castillo & Platense & 27 & 17 & 1 \\
M & 21 & Erwin Sanchez & Boavista (Por) & 24 & 26 & 7 \\
M & 22 & Julio C{\'e}esar Baldivieso & Bolivar & 22 & 26 & 1 \\ \hline
F & 9 & {\'A}lvaro Pe{\~n}a & Temuco & 29 & 37 & 4 \\
F & 10 & Marco Antonio Etcheverry & Colo Colo (Chile) & 23 & 23 & 5 \\
F & 11 & Jaime Moreno & Blooming & 20 & 17 & 2 \\
F & 18 & William Ramallo & Oriente Petrolero & 30 & 21 & 9 \\ \hline
\end{tabular}
\end{figure}
\section{Preview}
It all started in January of 93 when the Bolivian Federation hired the basque 
coach Xabier(Javier) Azkargorta.  The fans and the press didn't like him at 
first because he didn't have a good record in Spain. At the same time the 
Bolivian soccer league was suspended for the first half of 1993 because many 
players weren't getting paid. This allowed the national team to get together
and concentrate on the qualifying tournament for a full six months before they
began. This, as it turned out, was very advantageous since Bolivia qualified 
for a World Cup finals for the first time. (Bolivia participated in the World 
Cups of Uruguay 1930 and Brazil 1950 but at that time participation was by 
invitation.)

The backbone of the team consisted of at least seven players who graduated from 
a juvenile soccer academy called Tahuichi Aguilera.  This Academy has 
represented the country in various occassions conquering many tournaments 
abroad.  Many of these players play in clubs outside of Bolivia. Among these 
players are Marco Antonio Etcheverry (\#1), Luis Hector Cristaldo, Erwin Sanchez,
Milton Melgar, Juan Manuel Pen~a and Alvaro Pen~a.

The first game of the qualifying tournament was in Venezuela. Bolivia won 7-1,
with goals from William Ramallo (3), Erwin Sanchez (3), and Luis Cristaldo. Two 
weeks later was the big game against Brazil in La Paz, Bolivia. The game was 
fairly even until about 12 minutes from the end when Bolivia were awarded a 
penalty, but they missed. Incredibly, the players recovered from this disastrous
moment to score goals through Etcheverry and Pen~a in the last 2 minutes of the 
game, running out 2-0 winners. The following week, Uruguay were beaten 3-1 by
goals from Etcheverry, Sanchez and Milton Melgar, and this was swiftly followed
by a poor game against Ecuador, Ramallo scored the only goal for a 1-0 win.

After a 7-0 victory against Venezuela, this left Bolivia needing a draw from 
their last three games to qualify. The problem being that the final three games
were all away from home against Brazil, Uruguay and Ecuador. The Brazilian game
was a disaster for Bolivia as they went down 6-0, and Uruguay scraped a 2-1
victory with Sanchez scoring Bolivia's goal. Finally, a 1-1 draw in Ecuador got
Bolivia through, Ramallo scoring.

All of the teams in qualifying complained about the altitude of Bolivia's home
base, La Paz which is situated at 3600m above sea level. The altitude can 
affect some teams, but there is more to Bolivia's qualifying performance than
this.

Bolivia have been drawn in a group with Germany, Spain and South Korea. Although
this will be a tough group to progress from, it is not beyond the bounds of
possibility for the Bolivians to qualify for the second round. However, I don't
think we can expect much more than this.
\section{Pre World Cup Friendlies}
\begin{figure}[H]
\begin{tabular}{l l c l}
Feb 18 & USA & 1-1 & Miami \\
Feb 20 & Colombia & 0-2 & A \\
Mar 23 & USA & 2-2 & Dallas \\
Apr 7 & Colombia & 1-0 & A \\
Apr 20 & Romania & 0-3 & A \\
May 4 & Saudi Arabia & 1-0 & Cannes \\
May 11 & Cameroon & 1-1 & Athens \\
May 13 & Greece & 0-0 & Athens \\
May 19 & Iceland & 0-1 & A \\
May 24 & Ireland & 0-1 & A \\
Jun 8 & Peru & 0-0 & USA \\
Jun 11 & Switzerland & 0-0 & Canada \\
\end{tabular}
\end{figure}
\section{Key Players}
MARCO ANTONIO ETCHEVERRY (Midfield, Colo-Colo(CHILE)):
Has torn knee ligaments playing for his Chilean club side, but is expected to
recover for the finals).

ERWIN SANCHEZ (Midfield, Boavista(POR)):
A great striker of the ball.

LUIS CRISTALDO (Defender, Bolivar):

MIGUEL ANGEL RIMBA (Defender, Bolivar):

WILLIAM RAMALLO (Forward, Oriente Petrolero):
He is nicknamed ``fisherman of the area’’ because of his knack for picking up
all the passes and loose balls. Top scorer with 8 goals in qualifying.
