\chapter{Venues}
\chapterprecis{Greg Keklak and Edward Porter}
\newline
\newline
(A) denotes stadium with artificial surface, e.g. AstroTurf, which will be
covered with grass for World Cup play.

An asterisk highlights notes of special significance to WC'94.

In each case, the capacity shown is for American football. Capacities for World
Cup matches may differ from those shown below.

``Football'' in this chapter should be understood to mean American football (Gridiron).

\section{Boston, Groups C and D}
Foxboro Stadium, Foxboro, Massachusetts, (cap. 60,794)

Home of New England Patriots (NFL)

Site of U.S. victory over England, U.S. Cup 1993 *

Stadium originally named Schaefer Stadium after Brooklyn-based brewery which 
partially funded construction of the stadium - has also been known as Sullivan
Stadium, named for Billy Sullivan, former owner of New England Patriots 
franchise.  Closed and Demolished in 2002.   Replaced by Gillette Field
\section{New York/New Jersey, Groups E and F}
Giants Stadium (The Meadowlands), East Rutherford, New Jersey (cap. 77,301)

Home of New York Giants and New York Jets (both NFL)

Former home of defunct New York Cosmos (NASL)

The Meadowlands complex also includes facilities for harness racing and an 
arena for ice hockey and basketball - capacity crowds were common for Cosmos 
matches in the 1970s as fans came out to watch Pel{\'e}, Beckenbauer, Chinaglia and
others in NASL action.  Closed and Demolished in 2010.  Replaced by MetLife Stadium.
\section{Washington, Groups E and F}
Robert F. Kennedy Stadium (RFK), Washington, D.C. (cap. 56,454)

Home of Washington Redskins (NFL) until 1997.  Home of MLS DC United from 1994 to 2017.

Named D.C. Stadium when opened, but was renamed after 1968 assassination of 
then presidential candidate Robert F. Kennedy - also used as a baseball stadium
by Washington Senators (AL) until franchise moved to Arlington, Texas in 1971.

With one of the smallest stadiums in the NFL, the Redskins' ownership is 
looking to replace RFK with a 70,000 seat stadium in the near future.  (They moved
to FedEx Field in 1997)  A Major League Baseball franchise was relocated from
Montr{\'e}al in 2005 and became the Washington Nationals which played at RFK until 2007,
sharing the field with DC United.  DC United remained in RFK until 2017.
\section{Orlando, Groups E and F}
Citrus Bowl, Orlando, Florida (cap. 70,000)

Home of Univ. of Central Florida football team (NCAA Div. I-AA)

Site of annual Florida Citrus Bowl (college football)

Former home of defunct Orlando Thunder (WLAF)

As recently as 1971, the Citrus Bowl had a capacity of only 9,000
\section{Chicago, Groups C and D}
Soldier Field, Chicago, Illinois (cap. 66,950)

Home of Chicago Bears (NFL)

Site of opening ceremonies *

Opened in 1920, Soldier Field is the oldest stadium in use in the NFL - stadium
originally built as an elongated oval seating 100,000.  Major Renovations took place 
in 2002-2003
\section{Detroit, Groups A and B}
Pontiac Silverdome, Pontiac, Michigan (cap. 80,500).  Astroturf FIeld

Home of Detroit Lions (NFL)

First indoor venue in World Cup history - site of first international match 
held indoors on grass (Germany v. England, U.S. Cup 1993)

Fans enter and exit through revolving doors to prevent the air pressure which 
supports the roof from escaping the building.  Not used anymore, the Silverdome was sold for \$583,000 and
currently lies in ruin.
\section{Dallas, Groups C and D}
Cotton Bowl, Dallas, Texas (cap. 71,615) Astroturf Surface

Site of annual Cotton Bowl (college football)

Former home (1960-1972) of Dallas Cowboys (NFL)
\section{San Francisco, Groups A and B} 
Stanford Stadium, Palo Alto, California (cap. 85,500)

Home of Stanford University football team (NCAA Div. I-A)
 
Has served as site for Super Bowl XIX.  Demolished in 2005-06 and reconstructed.
\section{Los Angeles, Groups A and B} 
Rose Bowl, Pasadena, California (cap. 99,563)

Home of UCLA football team (NCAA Div. I-A, now called Football Bowl Series)

Site of annual Rose Bowl (college football) and has hosted Super Bowl

Site of World Cup final match *
\section{Grass Pitches to be installed over artificial surfaces}
A note on the grass pitches to be used at stadiums with artificial surfaces:

It is my understanding that the stadiums with artificial surfaces will be 
covered with grass much as the grass pitch was put into place in the Pontiac 
Silverdome for the US Cup last summer.  Sections of grass are grown at a remote
site in a sort of greenhouse, and then the sections of grown grass are shipped
via refrigerated truck to the stadium(s).  While I am not an expert on such 
groundskeeping matters, the grass installed over the turf at the Silverdome 
looked good and appeared to hold up well under the feet of the Germans and 
English.  

The process is expensive, but if it is consistantly successful, this will be a
large step towards eliminating the need for artificial turf, which has been 
shown to be a factor in increased injuries in American football.

\section{Abbreviations used in text}
\begin{figure}[H]
\begin{tabular}{l l}
AL & American League \\
DC & District of Columbia \\
NASL & North American Soccer League (defunct) \\
NCAA & National Collegiate Athletic Associtation \\
NFL & National Football League \\
UCLA & University of California--Los Angeles \\
WLAF & World League of American Football (defunct) \\
\end{tabular}
\end{figure}

