\chapter{Preview of Group D}
\chapterprecis{Simon Gleave}
\newline
\newline
\begin{figure}[H]
\small
\begin{tabular}{l c c c c c c c l}
Group D & Played & Won & Drawn & Lost & For & Agst & Apps & Best Performance\\ \hline
Argentina & 48 & 24 & 9 & 15 & 82 & 59 & 10 & Winners(1978, 1986) \\
Bulgaria & 16 & 0 & 6 & 10 & 11 & 35 & 5 & Second Round (1986) \\
Greece & & & & & & & & First Appearance \\
Nigeria & & & & & & & & First Appearance \\ \hline
\end{tabular}
\normalsize
\end{figure}
I'm sure we all remember sitting down to watch another boring World Cup opener
four years ago between Argentina and Cameroon, realising during the game that
Cameroon were a bit special, particularly in their tackling and then staring
with disbelief when they took the lead and went on to beat the World Champions.

Well, this time Argentina find themselves in the same group as the latest 
African pretender, Nigeria, who are fresh from winning the African Nations Cup.
I don't think anybody would disagree with me that Nigeria are a more 
accomplished team than Cameroon were and I think Group D will be between 1990's
losing finalists and 1994's African champions.

We all know about Argentina, they've appeared in 3 of the last 4 finals, 
winning twice, but a lot of the players are now veterans, particularly 
Maradona who is surely the greatest player of modern times. They may struggle 
to do as well this time, but they shouldn't have too many problems qualifying 
for the second phase.

Nigeria, on the other hand, are a relatively young and vibrant side who have a
more solid defence than is usually associated with African sides. Personally, I
think Nigeria have a great chance of reaching the quarter finals of the 
tournament and they could well win this group, the only negative is their lack
of experience at this level as they have qualified for the first time in their
history.

Bulgaria have an astonishing record in World Cup finals - they've qualified 
five times, played 16 matches and won none. The Bulgarians have never been in 
the first division of European football, but this is an incredible sequence, 
which they could well break this time around. They showed real fighting spirit
to qualify by means of scoring a last minute goal in Paris, and if they can 
bring this to the finals along with highly skilled players like Stoichkov and
Kostadinov, the second round should be reached, albeit in third place.

Greece are very hard to score against, but if you can score once they lose 
their composure and this is of no use whatsoever in a tournament of this type. 
They may be able to scrape the odd draw, but it is very difficult to see them 
beating any of their three rivals and therefore progressing.

Nigeria and Argentina will have a close fight to win the group with Nigeria 
just coming out on top, with Bulgaria coming in third and Greece bottom.
