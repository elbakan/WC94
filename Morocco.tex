\chapter{Morocco}
\chapterprecis{Reproduced from the Official World Cup Brochure and UPI Reports}
\newline
\newline
Coach: Abdellah Ajri Blinda
\begin{figure}[H]
\begin{tabular}{c c l l c c c}
P & & & Club & Age & Caps & Goals \\ \hline
G & 1 & Khalil Azmi & Raja de Casablanca & 30 & 53 & 0 \\
G & 12 & Said D'Ghay & Olympique Casablanca & 30 & 0 & 0 \\
G & 22 & Zakaria Alaoui & Marrakesh & 28 & 16 & 0 \\ \hline
D & 2 & Abdellah Nacer & Waregem (Bel) & 28 & 10 & 0 \\
D & 3 & Abdelkarim Hadrioui & Rabat & 22 & 12 & 0 \\
D & 5 & Ismail Triki & Chateauxroux & 27 & 5 & 0 \\
D & 6 & Nourredine Naybet & FC Nantes (Fra) & 24 & 49 & 1 \\
D &14 & Ahmed Masbahi & Marrakesh & 28 & 0 & 0 \\
D &18 & Rachid Nekrouz & Mouloudia Oudja & 22 & 0 & 0 \\ \hline
M & 4 & Tahar El Kahlej & Marrakesh & 32 & 50 & 0 \\
M & 7 & Mustapha Hadji & AS Nancy (Fra) & 23 & 1 & 0 \\
M & 8 & Rachid Azouzi & MSV Duisburg (Ger) & 25 & 16 & 0 \\
M & 10 & Mustapha El Hadaoui & SCO Angers (Fra) & 33 & 94 & 23 \\
M & 11 & Rachid Daoudi & WAC Casablanca & 28 & 41 & 11 \\
M & 15 & Larbi Hababi & Olympique Khourigba & 26 & 0 & 0 \\ \hline
F & 9 & Mohamed Chaouch & OGC Nice (Fra) & 28 & 56 & 11 \\
F & 13 & Ahmed Bahja & Marrakesh & 23 & 0 & 0 \\
F & 16 & Hassan Nader & Farense (Por) & 29 & 32 & 6 \\
F & 17 & Abdessalem Laghrissi & WAC Casablanca & 32 & 56 & 15 \\
F & 19 & Mjid Baouyboud & WAC Casablanca & 28 & 33 & 0 \\
F & 20 & Hassan Kachloul & N{\^i}mes Olympique (Fra) & 21 & 0 & 0 \\
F & 21 & Aziz Samadi & Rabat & 24 & 1 & 0 \\ \hline
\end{tabular}
\end{figure}
\section{History}
Football in Morocco can be traced back to the introduction of the sport into 
the region by French colonists. Some clubs, such as Wydadof Casablanca (WAC) 
became centres for anti-colonial sentiment. The US Meknes club, now defunct, 
had a great nationalist tradition, while in the north, Tangier and Larrache 
were notable for their opposition to Spanish occupation. On the other hand, 
Morocco provided some great players for French football. The most notable of 
these was Larbl Ben Barek, ``la perle noir'' (black pearl). He was born in 
Casablanca in 1917, was capped 17 times by France, playing for Olympique de 
Marseille, Stade Francais and Atletico Madrid. After his retirement, he moved 
back to his homeland. One tribute to his ability came from Pele, who visited 
the country and whose first request was to meet the old master.

The Moroccan League is composed of 20 first division teams, which play the 
championship over 15 weeks. There is also a 56-club second division, which is 
subdivided into four regional groups. The second division season is played over 
13 weeks, where the teams play for the cup of the throne, patronised by King 
Hassan.

On the international front, Morocco was the first African country to play in 
the World Cup finals in 1970. It fared well in its first tournament, scaring 
the West Germans by scoring first, and then scoring its first World Cup point 
by drawing with Bulgaria, 1-1. The Moroccan international side was back in the 
World Cup 16 years later, in Mexico in 1986. It played extremely well, 
finishing at the top of its group, ahead of England, Poland and Portugal. 
Morocco went out in the next game losing 1-0 to West Germany in a closely 
contested match.

The glory of the 1986 World Cup has not been repeated since. Morocco lost 
the 1988 African Cup of Nations as organiser, and two years later was 
eliminated again by Mali in Marrakesh. The training and coaching from 1988 has 
gone through several changes to reach its present state. The Brazilian, Valente
took over from his compatriot Farla, who made his mark in Mexico in 1986, 
before handing it over again to Italian-Argentinian, Angelino Angelillo. 
Former coach, German Olk Werner was recruited after the 1990 World Cup in Italy 
before being replaced by Abdellah Blinda most recently.
\section{WC 1994 Qualification}
\begin{figure}[H]
\begin{tabular}{l l c l c l}
1992 & & & & & \\
Oct 11 & Ethiopia & WCQ1 & Won & 5-0 & Casablanca \\
Oct 25 & Benin & WCQ1 & Won & 1-0 & Porto Nova \\
Dec 20 & Tunisia & WCQ1 & Drew & 1-1 & Tunis \\
1993 & & & & & \\
Jan 17 & Ethiopia & WCQ1 & Won & 0-1 & Addis Ababa \\
Jan 31 & Benin & WCQ1 & Won & 5-0 & Rabat \\
Feb 28 & Tunisia & WCQ1 & Drew & 0-0 & Casablanca \\
Apr 18 & Senegal & WCQ2 & Won & 1-0 & Casablanca \\
Jul 4 & Zambia & WCQ2 & Lost & 2-1 & Lusaka \\
Jul 17 & Senegal & WCQ2 & Won & 1-3 & Dakar \\
Oct 10 & Zambia & WCQ2 & Won  & 2-0 & Casablanca \\
\end{tabular}
\end{figure}
\section{1994 Friendlies}
\begin{figure}[H]
\begin{tabular}{l l c c l}
Feb 6 & Sharjah, UAE & win & 2-1 & Slovakia \\
Mar 23 & Luxembourg & win & 2-1 & Luxembourg \\
Apr 20 & Salta & loss & 1-3 & Argentina \\
Jun 1 & Montr{\'e}al & draw & 1-1 & Canada \\
\end{tabular}
\end{figure}
\section{Preview}
USA '94 is the third World Cup finals for both Morocco and Cameroon but, as 
with their fellow African qualifier, this year's Moroccan team looks a pale 
imitation of the past. There is plenty of incentive for the players to do well
though as King Hassan II has promised them each a car and a house if they 
surpass the achievement of the 1986 team and reach the quarterfinals. The team
can have no complaints about its Group F opponents in the States. The seeded 
team is Belgium -- the weakest seed apart from the host, Saudi Arabia is no 
world power and only The Netherlands look a class apart. However, the second 
round looks an unlikely target for a team in disarray. Coach Abdellah Blinda 
took over from the fired Adbel Khalid El Louzani just a month before that vital 
last qualifier in Casablanca and immediately looked to his European-based 
players to add some spark to the squad. This has caused some friction in the 
camp as few of the newcomers played in the qualifiers. Many of them don't speak 
Arabic -- three were even born overseas -- but they add experience and 
discipline to the sometimes wayward natural talents of the home-based players.

The fulcrum of the squad is midfielder Rachid Daoudi, who plays for Wydad
Casablanca. He has five goals from his 23 international appearances and loves
to get forward. He has a terrific long-range shot and is dangerous from free-
kicks. He is complemented by newcomer Mustappha Hajji, who plays for Nancy in
France. The 23-year-old has only played four times but has had rave reviews,
particularly after his debut in the Zambia match. He was offered the chance to 
play for France but opted for Morocco. ``I chose the country of my origin'', he 
said, ``and believe me, after my first match against Zambia I don't think I'll 
regret the choice''.

In defense the key is Noureddine Nybet, who also plays in France, for Nantes.
He is determined and organized and can play in midfield if needed. The other 
key defenders, Lahcen Abrami, Ahmed Mashabi and Adbelkrim El Hadrioui don't 
inspire confidence and even the Saudis will fancy their chances.

Up front things look a bit thin. Abdel Laghressi, who scored the winner against 
Zambia, doesn't appear to feature in Blinda's first- choice lineup, leaving 
Hassen Nader to shoulder the responsibility. However, the 28-year-old, who 
can't stop scoring for Farense in Portugal has only managed six in 32 
internationals and doesn't really look comfortable against international 
defenders. Mohamed Chaouch, another French import from Nice, has a similar 
strike-rate (11 from 52 games) and will be in contention for a place.

However, as with Cameroon, this team looks unlikely to cause too many shocks 
and it looks like Nigeria will lead the way for the African continent this 
time around.
\section{Key Players}
Coach:
ABDELLAH AJRI BLINDA, Age: 43. Appointed a month before the final qualifier
against Zambia and only had one warm-up match to prepare. Has little faith in
domestic league and is pinning his hopes on the European-based contingent.

KHALI AZMI (Goalkeeper, Raja Casablanca) 29 years old, 40 caps:
Ezaki Badou's successor. Good positioning and calm under pressure. Good in the 
air.

NOURREDINE NAYBET (Defender, Nantes(FR)) 24 years old, 20 caps:
Good stopper and consistent. May move to a bigger European club with a good 
showing in America.

AHMED MASBAHI (Defender, KAC Marrakech) 28 years old, 26 caps:
Strong central defender and a certain starter.

TAHAR EL KHALEJ (Defender, KAC Marrakech) 25 years old, 45 caps: 
Strong defender who can play in midfield aswell. Tall and good in the air.

RACHID DAOUDI (Midfield, WAC Casablanca) 28 years old, 35 caps: 
The most popular footballer in Morocco. Skilful and flambuoyant. Likes to get 
wide but also has an eye for goal. A key player in the lineup.

MUSTAPHA HADJI (Midfield, Nancy(FR)) 22 years old, 4 caps: 
Stunning debut against Zambia in vital qualifier. Preferred to play for  
Morocco than France. Full of potential, he has two good feet and loves to get 
forward.

MUSTAPHA EL HADDAOUI (Midfield, Angers(FR)) 32 years old, 74 caps: 
King Hassan II requested that he be restored to the team. Good vision and a 
veteran of the team that did so well in the 1986 Mexico World Cup.

MOHAMED CHAOUCH (Forward, Nice(FR)) 27 years old, 52 caps: 
A classic centre forward and the best hope of the Moroccan front line. Been 
around the clubs in France and now recovered from a long injury lay-off.

HASSAN NADDER (Forward, Farense(PORT)): 28 years old, 32 caps: 
Despite scoring non-stop for his club, he has a poor international return with
just six goals from 32 games.
