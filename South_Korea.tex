\chapter{South Korea}
\chapterprecis{Hans S Cho}
\newline
\newline
Coach: Kim Ho
\begin{figure}[H]
\begin{tabular}{c c l l c c c}
P & & & Club & Age & Caps & Goals \\ \hline
G & 1 & Choi In-Young & Hyundai Horang-I & 32 & 32 & 0 \\
G & 21 & Park Chul-Woo & Daewoo Royals & 28 & 0 & 0 \\
G & 22 & Lee Won-Jae & Kyunghe University & 21 & 0 & 0 \\ \hline
D & 2 & Chung Jong-Son & Hyundai Horang-I & 28 & 7 & 0 \\
D & 3 & Lee Jong-Hwa & Ilhwa Chonma & 28 & 1 & 0 \\
D & 4 & Kim Pan-Keun & L.G. Cheetahs & 28 & 25 & 1 \\
D & 5 & Park Jung-Bae & Daewoo Royals & 27 & 26 & 2 \\
D & 7 & Shin Hong-Gi & Hyundai Horang-I & 26 & 25 & 2 \\
D & 12 & Choi Young-Il & Hyundai Horang-I & 28 & 3 & 0 \\
D & 13 & An Ik-Soo & Ilhwa Chonma & 29 & 4 & 0 \\
D & 17 & Gu Sang-Bum & Daewoo Royals & 30 & 70 & 2 \\
D & 20 & Hong Myung-Bo & POSCO Atoms & 25 & 46 & 4 \\ \hline
M & 6 & Lee Young-Jin & L.G. Cheetahs & 30 & 37 & 1 \\
M & 8 & Noh Jung-Yoon & Sanfrecce Hiroshima (Jap) & 23 & 15 & 1 \\
M & 9 & Kim Joo-Sung & Vfl Bochum (Ger) & 28 & 54& 9 \\
M & 10 & Ko Jeong-Woon & Ilhwa Chonma & 27 & 31 & 5 \\
M & 11 & Seo Jung-Won & L.G. Cheetahs & 23 & 37 & 9 \\
M & 14 & Choi Dae-Sik & Hyundai Horang-I & 29 & 3 & 0 \\
M & 16 & Ha Seok-Ju & Daewoo Royals & 26 & 31 & 13 \\
M & 19 & Choi Moon-Sik & Ilhwa Chonma & 23 & 20 & 4 \\ \hline
F & 15 & Cho Jin-Ho & Ilhwa Chonma & 21 & 4 & 2 \\
F & 18 & Hwang Sun-Hong & POSCO Atoms & 25 & 47 & 20 \\ \hline
\end{tabular}
\end{figure}
\section{Preview}
The Koreans (South) first participated in the World Cup in 1954, losing 
miserably to Turkey and Hungary by ridiculous scores. Between then and the 
early 1980's, Korean soccer was outshone by Malaysian and Middle Eastern teams
regularly, although North Korea shocked the world in the 1966 World Cup by 
upsetting Italy, advancing to the second round, and nearly eliminating 
Portugal, leading 3-0 until Eusebio scored 4 goals to put them out. This is all
ancient history and has no bearing on Korean soccer today.

In 1983, the impossible happened - the South Korean youth team was eliminated 
from the World Youth Championships Asian qualifying, but the North Korean team,
which had qualified, rioted after a game in Kuwait, beating several officials
over call disputes and caused FIFA to disqualify them. They were replaced by
South Korea. The Korean team, trained, led, and mostly handpicked by the 
legendary Park Jung Nam, leapt at this second chance, qualifying for the second
round. In the quarterfinal they met Portugal and BEAT THEM! The country was in
celebration, there was dancing in the streets, and the President declared that
if they beat their next opponents, Brazil, he would declare a national holiday.
All hell broke loose when the Red-Devils (moniker referring to the Korean 
uniform) scored first against Brazil. They eventually lost 2-1, but this was 
the beginning of South Korean prominence as we know it today.

The same youths who played for the 1983 team were the backbone of the 1986 
World Cup team, with two notable exceptions; Cha Beom Gun, the star midfielder
based in the Bundesliga with Bayer Leverkusen, and a 19 year old newcomer named
Kim Joo Sung, who now plays for Vfl Bochum, also in the Bundesliga. Among the
1983 ``veterans'' were Choi Soon Ho, the best forward South Korea have ever had,
and Byon Byung Joo, the fastest player in Asia at the time. This was the best 
team ever fielded by South Korea, but was lacking in defence and in finishing.

They made it to the 1986 World Cup in Mexico with little trouble, but were 
grouped with Argentina, Italy and Bulgaria. The first game against Argentina
was a disaster, ending with the Korean coach in tears after his youngsters had 
been destroyed.

The Koreans scored their first point in World Cup history in the next match by
drawing 1-1 with Bulgaria. This meant that they had to defeat defending 
champions Italy to advance, and they were determined to at least try. Over 90
minutes against a team of overwhelming talent, they gave the Italians a run for
their money, scoring brilliantly and playing entertaining soccer. Cha Beom Gun
had the most attention, being constantly double and triple marked. It was the
defence that was found lacking in the end, and they lost 3-2.

By 1990, South Korea's dominance in Asia was taken for granted by many. 
However, on a global scale, Asian soccer had no respect. Many had hoped that 
the 1990 Italian World Cup would change some of that. Again, most of the team 
comprised of veterans from the former squad. Choi Soon Ho was present again and
captained the team, additions included Hwangbo Kwan, a beefy midfielder who is
an Asian clone of Paul Gascoigne, Noh Jung Yoon, and Hwang Seon Hong, a tall, 
quick forward who plays in Holland.

They qualified with some difficulty, managing a crucial win over Saudi Arabia. 
In Italy, they were grouped with Belgium, Spain, And Uruguay, all enormous 
soccer powers. The first match with Belgium was painful to watch, with 
defensive ineptitude and uninspiring play the story for South Korea, and the 
Belgians ran out easy 2-0 winners. The Korean fans were horrified at the team, 
not because they lost but because their performance was so poor. The Korean 
team carefully reassessed their options, figured out that it was pointless to 
attempt to outplay their opponents, and went back to the only game they could 
recall any success; the 1986 match against Italy.

The next game they returned with the game South Korea loves; midfield-based
fast-attack, counterattack soccer. Spain didn't see it coming, but it hit them
5 minutes into the game, when Byun Byung-Joo materialized out of nowhere to 
receive a perfect pass from the swarming midfield and dashed to find himself
one-on-one with the Spanish goalkeeper. He missed. If he had been more patient
and more controlled, the match could have been very different. The Spanish 
recovered to score first through Michel. It looked like business as usual for
South Korea but an equalizer arrived just before half-time when a direct free 
kick from Hwangbo Kwan, curved over the wall and into the far corner of the 
net. In the second half, though, it was indeed business as usual as Michel 
scored twice more and the South Korean offence spluttered. South Korea now had 
no chance of advancing except if they beat Uruguay by 6 and Belgium won. The 
game seemed pointless, but spoiling Uruguay's chances would have been fun. The 
pace was again slowed down in favour of defence, with absolutely no chances 
until Fonseca scored a late offside goal and got Uruguay, a disappointment that
year, into the second round.

South Korean soccer, once again measured against the powers that be, looked as
miserable as it would ever get. But, in 1991, another surprise happened, again
involving North Korea; the governments of North and South Korea agreed to 
combine the team for the World Youth Championships, which South Korea had 
qualified for and North Korea had not. The team was assembled equally between 
North and South, and they went to Portugal, not knowing what lay in store for 
them.

The first match seemed hopeless; the opponents being Argentina, with young star
Maradona Esnaider. The team contained defenders who were mainly from the South
and forwards mainly from the North. Three defenders in particular, Kang Chul, 
Lee Im-Saeng, and Lee Tae-Hong, all from the South, and all possible members of
this year's World Cup team, were highly effective, because they were shutting 
down the Argentine attack and even threatening themselves. The defenders and 
midfielders from the South put the pressure on, while the North's forwards were
also wreaking havoc upon the Argentine defence with numerous attacks. Esnaider,
marked by Lee Tae-Hong, was thoroughly frustrated and rendered ineffective. In
the 89th minute, the Argentine defenders fouled a Korean outside the penalty 
area, giving away a direct free kick. The shot bounced off of the wall of 
defenders, but Cho In-Chol, a North Korean midfielder struck the ball from 30 
yards out and Korea won 1-0.

The last International tournament the Koreans played in outside Asia was in the
1992 Barcelona Olympics. There they played brilliantly against highly regarded
Sweden, but only managed to draw 1-1. Striker Suh Jung Won controlled a long
pass over the goalkeeper with his chest and spun around him to score. They drew
both their remaining games and failed to advamce further.

The present team has members of the 1990, 1991 and 1992 squad who should have 
a good idea of what World class soccer is about, although it's arguable whether
they can play it.They qualified for the 1994 World Cup Finals by getting 
through the first and second rounds of Asian qualifying. The dramatic end came
on the very last day of the second round when Iraq held Japan allowing South
Korea to qualify with Saudi Arabia.
\begin{figure}[H]
\begin{tabular}{l l l c l}
1994 & & & & \\
Feb 16 & Romania & Lost & 1-2 & H \\
Feb 20 & Malaysia & Won & 5-1 & H \\
Feb 26 & Colombia & Drew & 2-2 & A \\
Mar 12 & USA & Drew & 1-1 & A \\
May 1 & Cameroon & Drew & 2-2 & H \\
Jun 6 & Ecuador & Lost & 1-2 & USA \\
Jun 11 & Honduras & Won & 3-0 & USA \\
\end{tabular}
\end{figure}

\section{Key Players}
NOH JUNG YOON (Midfield, Sanfrecce Hiroshima(JAP)):
In my opinion, the best player in Korea today. A solid defender and smart 
passer, he figures in almost every goal that South Korea scores. His uncanny 
passing is the main reason the team can threaten without a tall, physical 
forward. He has a very good long range shot.

SUH JUNG WON (Forward, L.G. Cheetahs): 
One striker who would do OK even without Noh. Quick and explosive, he is the 
closest thing Asia has to the likes of Voeller and Van Basten. In the second 
round of Asian qualifying he only scored one goal, but this was mainly due to
very heavy marking by the opposition.

KIM JOO SUNG (Midfield/Forward, Vfl Bochum(GER)): 
The only player remaining from the 1986 World Cup finals. A great player, but 
he does tend to disappear in the most important games. Perhaps this, too, is 
on account of heavy marking. Often tries to dribble too much, but is by far 
the best ball player South Korea have. Plays for Bundesliga II's Vfl Bochum.

HWANG SUN HONG (Forward, POSCO Atoms): 
A relatively tall but also quick scorer, who mainly scores with headers from
crosses from Suh and Kim. Plays for a Dutch team.

KOH JUNG WOON (Forward, Ilhwa Chonma):
Once regarded as the best striker in South Korea, but is known to make stupid 
descisions such as trying to beat four defenders when a teammate is free to 
pass to. Shows touches of brilliance but must use his head more.

GU SANG BEOM (Midfield, Daewoo Royals): 
Strong midfield general and captain of the team.
