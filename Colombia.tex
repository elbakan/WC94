\chapter{Colombia}
\chapterprecis{Maurico Pardo}
\newline
\newline
Coaches: Francisco Maturana and Hernan Dario Gomez
\begin{figure}[H]
\small
\begin{tabular}{c c l l c c c}
P & & & Club & Age & Caps & Goals \\ \hline
G & 1 & Oscar Cordoba & America de Cali & 24 & 33 & 0 \\
G & 12 & Farid Mondragon & Argentinos Juniors (Arg) & 22 & 2 & 0 \\
G & 22 & Jose Maria Pazo & Junior de Barranquilla & 30 & 1 & 0 \\ \hline
D & 2 & Andres Escobar & Nacional de Medellin & 26 & 53 & 1 \\
D & 3 & Alexis Mendoza & Junior de Barranquilla & 32 & 41 & 1 \\
D & 4 & Luis Fernando Herrera & Nacional de Medellin & 31 & 60 & 0 \\
D &13 & Nestor Ortiz & Once Caldas & 25 & 6 & 0 \\
D &15 & Luis Carlos Perea & Junior de Barranquilla & 30 & 79 & 2 \\
D &18 & Oscar Cortes & Millionarios & 25 & 2 & 0 \\
D & 20 & Wilson Perez & America de Cali & 26 & 49 & 4 \\ \hline
M & 5 & Hermann Gaviria &  Nacional de Medellin &  24 & 8 & 1 \\ 
M & 6 & Gabriel Jaime Gomez & Nacional de Medellin & 34 & 49 & 1 \\
M & 8 & Harold Lozano & Nacional de Medellin & 22 & 16 & 2 \\
M &10 & Carlos Valderrama & Junior de Barranquilla & 32 & 69 & 6 \\
M &14 & Leonel Alvarez & America de Cali & 28 & 71 & 1 \\
M &17 & Mauricio Serna & Nacional de Medellin & 26 & 13 & 0 \\
M &19 & Freddy Rincon & Palmeiras (Bra) & 27 & 43 & 12 \\ \hline
F & 7 & Anthony De Avila & America de Cali & 31 & 39 & 11 \\
F & 9 & Ivan Rene Valenciano & Junior de Barranquilla & 22 & 19 & 7 \\
F & 11 & Adolfo Valencia &  Bayern Munich (Ger) & 26 & 18 & 11 \\
F & 16 & Victor Aristizabal & Valencia (Spa) & 22 & 22 & 5 \\
F & 21 & Faustino Asprilla & Parma(Ita) & 24 & 14 & 4 \\ \hline
\end{tabular}
\normalsize
\end{figure}
Colombia has qualified only for three world cups, 1962 in Chile, 1990 in Italy
and now for the 1994 USA world cup. A new style to Colombian soccer began in 
1987 when Francisco Maturana was appointed as coach. He was a disciple of the 
late Oswaldo Zubeldia (the legendary coach of Estudiantes de la Plata, during 
their reign as the leading international team of the late 60s and early 70s)
and consolidated a philosophy of attacking and skilful soccer in the best Latin
American tradition.

In the first round of Italia 90, Colombia defeated the UAE 2-0, obtained a 
draw 1-1 with the eventual champions West Germany in a very exciting game and
lost 1-0 against Yugoslavia. These results were good enough to progress to the
second round where Colombia me the Indominatable Lions of Cameroon. The game 
was fairly dull and it was no surprise to find the game going into extra time
with the score 0-0. Roger Milla came on as a substitute for Cameroon and scored
before Rene Higuita, hailed as the World's premier sweeper-keeper tried to beat
Milla, 10 yards outside the penalty area and was robbed allowing Milla to make 
the score 2-0. Colombia scored a late consolation but were out of the World 
Cup.

Some friendlies in Colombia since June 1992 (Sorry no dates or places) and Colombian scorers:
\begin{itemize}
\item vs Chile 1-0 (Asprilla)
\item vs USA 2-1 (Valencia, A.Garcia)
\item vs Venezuela 0-0
\item vs Venezuela 1-1 (A. Garcia) (this game was in Venezuela)
\item vs Chile 1-1 (Aristizabal)
\item vs Chile 1-0 (Asprilla)
\item vs Brazil 1-2 (Lozano)
\item vs France 3-1 (Valencia 3) (game in Martinica)
\end{itemize}
 
Three friendlies against USA in USA in 1993, Colombia won all three games 2-1, 1-0, and 3-1.
 
American Cup (Copa America) (1993)
 
First Round:
vs Mexico      2-1 (Aristizabal 2)
vs Bolivia     1-1 (Orlando Maturana)
vs Argentina   1-1 (Rincon)
 
Quarterfinal
vs Uruguay     1-1 (Perea)
 
Semifinal
vs Argentina   0-0 (lost 4-5, penalities after 90 min. draw)
 
Third Place
vs Ecuador     1-0 (Valencia)
 
 
World Cup 1994, Qualifying games.
 
vs Paraguay (in Colombia) 0-0
vs Peru (in Peru) 1-0  (Rincon)
vs Argentina (in Colombia) 2-1  (Valenciano, Valencia)
vs Paraguay (in Paraguay)  1-1  (Rincon)
vs Peru (in Colombia)  4-0  (Valenciano, Rincon, Mendoza, Perez)
vs Argentina (in Argentina) 5-0  (Rincon 2, Asprilla 2,Valencia)
 
Pre World Cup Friendlies.
\begin{figure}[H]
\begin{tabular}{c l l c l}
Date & Opponent & Location & Score & Colombia Scorers \\ \hline
28.1.94 & Venezuela & Away & 2-1 & (Valenciano, Trellez) \\
9.2 94 & Saudi Arabia & A & 1-0 & (Aristizabal) \\
18.2.94 & Sweden & Miami & 0-0 & \\
20.2.94 & Bolivia & Miami & 2-0 & \\
26.2.94 & South Korea & H & 2-2 & \\
 3.3.94 & Mexico & A & 0-0 & \\
 7.4.94 & Bolivia & H & 0-1 & \\
17.4.94 & Nigeria & Armenia & 1-0 & \\
 3.5.94 & Peru & Miami & 1-0 & \\
 5.5.94 & El Salvador & Miami & 3-0 & \\
 3.6.94 & Northern Ireland & Foxboro & 2-0 & (Perez, Valencia) \\
 4.6.94 & Greece & New Jersey & 2-0 & (Gaviria, Rincon) \\ \hline
\end{tabular}
\end{figure}
The problem for Colombia before the qualifying games was their difficulty in
scoring despite a well planned strategy of skill, short passing and swift
attacking. However, by the end of the qualifying tournament, the Colombian
style looked pretty good as they qualified with ease.

The goalless draw against Paraguay in the first qualifying game raised the 
problem of the Colombians difficulty in scoring, but the next game against Peru
in Lima was won 1-0 with several crucial players injured and this seemed to be 
the catalyst in recovering the players' confidence. For the first game against
Argentina, the two defensive midfielders, Alarez and Gomez were injured and
Maturana decided to experiment with two younger more attacking players, Gaviria
and Lozano. The outcome was a Colombian victory which led the press to champion
this more risky attacking line-up. However, Alvarez and Gomez were recalled for 
the next game in Paraguay for their strength and experience and Colombia drew
1-1.

The fireworks arrived in Colombia's final two games with a 4-0 home victory 
against Peru and then the result that shocked the World, a 5-0 win over 
Argentina in Buenos Aires. At last the three strikers playing in Europe had
gelled in the National side and Colombia had a team to be proud of.

\section{Key Players}
CARLOS VALDERRAMA (Midfield, Junior):
He is currently carrying a knee injury from the friendly with Sweden in 
February, but hopes to be fit in time. Colombia's playmaker is a fantastic 
passer of the ball and it is crucial to the Colombians that he is on top of his
form.

FREDDY RINCON (Forward, Palmeiras(BRA)):
Scored the equaliser against Germany in the last World Cup finals that took
Colombia through to the second phase. A pacey forward with plenty of skill.

FAUSTINO ASPRILLA (Forward, Parma(ITA)):
A rangy and highly skilled striker who has caused all sorts of havoc for Parma,
he looks very awkward on the ball but this is just hiding his deceptive skill. 
He can beat players and shoots from anywhere - the only criticism of his game
would be that he tends to hold onto the ball too long.

ADOLFO VALENCIA (Forward, Bayern Munich(GER))
His scoring record of 11 goals in 18 internationals speaks for itself - a very
dangerous striker who is likely to torment defences this summer.

OSCAR CORDOBA (goalkeeper, America)  
A far more orthodox goalkeeper than Rene Higuita.
