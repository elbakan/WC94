\chapter{Belgium}
\chapterprecis{Claude Detienne}
\newline
\newline
Coach: Paul van Himst
\begin{figure}[H]
\begin{tabular}{c c l l c c c}
P & & & Club & Age \\ \hline
G & 1 & Michel Preud'homme & Mechelen & 34 & 51 & 0 \\
G & 12 & Filip Dewilde & Anderlecht & 29 & 4 & 0 \\
G & 20 & Dany Verlinden & Bruges & 30 & 0 & 0 \\ \hline
D & 2 & Dirk Medved & Bruges & 25 & 13 &  0 \\
D & 3 & Vital Borkelmans & Bruges & 31 & 10 &  0 \\
D & 4 & Philippe Albert & Anderlecht & 26 & 30 &  3 \\
D & 5 & Rudi Smidts & Antwerp &  30 & 12 &  1 \\
D & 13 & Georges Gr{\"u}n & Parma (Ita) & 32 & 69 & 3 \\
D & 14 & Michael De Wolf & Anderlecht & 36 & 37 & 1 \\
D & 22 & Pascal Renier & Bruges & 22 & 1 & 0 \\ \hline
M & 6 & Lorenzo Staelens & Bruges & 30 & 19 & 1 \\
M & 7 & Franky van der Elst & Bruges &   33 & 61 &  0 \\
M & 10 & Enzo Scifo & AS Monaco (Fra) & 28 & 64 & 15 \\
M & 15 & Marc Emmers & Anderlecht & 28 & 33 &  2 \\
M & 16 & Danny Boffin & Anderlecht & 28 & 19 &  0 \\
M & 19 & Eric van Meir & Charleroi & 26 &  2 &  0 \\
M & 21 & Stephan van der Heyden & Bruges & 24 &  3 &  0 \\ \hline
F & 8 & Luc Nilis & Anderlecht & 27 & 25 &  2 \\
F & 9 & Marc Degryse & Anderlecht & 28 & 48 & 17 \\
F & 11 & Alex Czerniatynski & Mechelen & 33 & 30 &  6 \\
F & 17 & Josip Weber & Cercle Bruges & 29 &  2 &  6 \\
F & 18 & Marc Wilmots & Standard Li{\`e}ge & 24 & 23 &  8 \\ \hline
\end{tabular}
\end{figure}
\section{History}
\textbf{WC 1930 Uruguay}
Belgium was one of the rare countries to take part in the first WC. It was
no brilliant participation: two matches lost: USA-Belgium 3-0 and
Paraguay-Belgium 1-0. The results can be explained by the long journey to
the New World, the weather, the food. Yet that WC remained as an
extraordinary souvenir for the players and the team which went to Uruguay

\textbf{WC 1934 Italy}
Second of his preliminary phase group (1st was Holland), Belgium went to
Italy just to play one match... against Germany and bas beaten 5-2 (1-2 for
Belgium at half time though).

\textbf{WC 1938 France}
Belgium did qualify together with Holland (bis repetita). This time too, we
went to the final phase just to play one match against host country France
(1-3). The lack of preparation was evident: Belgian soccer was still very
amateur.

\textbf{WC 1954 Switzerland}
WC 1954 saw the first point won by the Red Devils in the WC history. This
point was the result of an extraordinary match against England which ended
after extra-time with the score of 4-4. It was quite an exhausting match and
the team was severely beaten in the 2nd match against Italy (1-4).

\textbf{WC 1970 Mexico}
WC 1970 saw the first win of Belgium in the first match against El Salvador
(3-0) and the fans had great hopes. But the Soviet Union was too strong for
the Red Devils (4-1). Everything was however still possible but the third
match against host Mexico ended with a Mexican victory (1-0) due to a
penalty more than generously given by the referee (Mexico could not be
eliminated too soon of course, since it was the organizing country).

\textbf{WC 1982 Spain}
WC 1982 was the beginning of the WC love story for Belgium. In the EC 1980
in Italy, Belgium had been a surprising vice-champion. So the team went to
Spain with a much self-confidence. Belgium had the difficult honour to open
the WC in the 1st match against Cup holder Argentina. This match is still
recalled in Belgium as one of the most extraordinary ones in the history of
the national team. We all remember the superb goal of Erwin Vandenbergh
after a brilliant cross of Franky Vercauteren. A difficult victory over El
Salvador (1-0) was not enough: Belgium had to take a point against Hungary,
who had a better goal difference, to go to the next round. It was a
difficult match but the draw was achieved (1-1), thanks to a goal from
Czeniatynski (who could be playing in WC 94!). The second round, with Poland
and the Soviet Union, was a catastrophe for Belgium (Poland-Belgium 3-0, 3
goals from Boniek; and SU-Belgium 1-0).

\textbf{WC 1986 Mexico}
The qualification phase was very special. Belgium had to play two matches
against Holland to be allowed to go to Mexico. The qualification was won at
the very end of the second game by a goal from defender Georges Grun. 
The first round began badly for Belgium who was defeated by Mexico in the
first match. The second match was not convincing: a difficult victory over
Iraq! Fortunately we managed to qualify in the match against Paraguay. This
began a series of extraordinary matches. The one against USSR was surely
the most extraordinary match in the Red Devils' history, 4-3 after extra-time,
although USSR had taken the lead twice. There were less goals in the next
match against Spain but it was still a very tough match and we did qualify
on penalties winning the right to meet Argentina in the semifinals; perhaps in
normal conditions we could have done something but after two very exhausting 
matches it was too difficult. The match for the 3rd place against France was 
lost 2-4, and so we ended 4th.

\textbf{WC 1990 Italy}
After an easy victory over South Korea (2-0), the Red Devils had a very tough
match against Uruguay who were a very hard team; the victory was very difficult 
to achieve and had to be played with many injuries in the defence. Three of the 
four defenders could not play the match against Spain which was lost by 2 goals
to 1, but second round qualification was assured. The match against England was 
a great game. Belgium were the better team and the Englishmen were expecting 
a penalty-shootout, accepting Belgium's undisputed domination. Unfortunately, 
at the very end of the match there was a stupid free quick, a moment of 
distraction in the defence and Platt's goal sent the Red Devils home.
\section{World Cup 1994 Qualifying}
April 22 1992; Parc Astrid, Anderlecht
Belgium  1-0      Cyprus  
Wilmots (23')

Preud'homme, Grun, Van der Elst, Albert, Emmers, Scifo, Walem, Degryse, 
Boffin (Borkelmans 82'), Oliveira, Wilmots (Hofmans 74')

June 3 1992; Toftir, Torshavn
Faroe Islands  0-3      Belgium 
Albert (30') Wilmots (65' and 70')

Preud'homme, Staelens, Grun, Emmers, Albert, De Nil, Van der Elst, Scifo, 
Boffin (Versavel 75'), Degryse, Oliveira (Wilmots 63')

September 2 1992; Prague, Prahov Stadium
Czechoslovakia Rep. 1-2      Belgium 
Kadlec (78')  Chovanec (44', og) Czerniatynski (82')

Preud'homme, Medved, Grun, Emmers, Albert, Smidts, Staelens (Dauwen 88'), 
Van der Elst, Scifo, Degryse (Wilmots 66'), Czerniatynski

October 14 1992; Anderlecht, Parc Astrid
Belgium 1-0      Romania
Smidts (26')

Preud'homme, Medved, Grun, Albert, Smidts, Staelens, Scifo, Van der Elst, 
Boffin, Degryse, Czerniatynski (Wilmots 68')

November 18 1992; Anderlecht, Parc Astrid
Belgium         2-0      Wales 
Staelens (54')
Degryse (59')

Preud'homme, Medved, Grun, Albert, Smidts, Scifo, Staelens (Wilmots 84'), 
Van der Elst, Boffin, Czerniatynski (Nilis 46'), Degryse

February 13 1993; Nicosia, Makarios Stadium
Cyprus          0-3      Belgium 
           Scifo (1', 4')
           Albert (87')
        
Preud'homme, Medved, Grun, Albert, Smidts, Staelens, Van der Elst, 
Scifo (Goossens 88'), Boffin, Degryse, Nilis (Czerniatynski 76')

March 31 1993; Cardiff, Arms Park
Wales    2-0      Belgium 
Giggs (18')
Rush (39')

Preud'homme, Medved (Oliveira 46'), Grun, Albert, Smidts, Boffin, Scifo, 
Van der Elst, Staelens, Degryse, Czerniatynski (Severeyns 67')

May 22 1993; Anderlecht, Parc Astrid
Belgium         3-0      Faroe Islands
Wilmots (33')
Scifo (50' pk)
Wilmots (76')

Preud'homme, Staelens, Emmers, Grun, Smidts (Oliveira 76'), Degryse, 
Van der Elst, Scifo, Boffin, Wilmots, Nilis

October 13 1993; Bucharest, Steaua Stadium
Romania         2-1      Belgium  
Raducioiu (66' pk)       Scifo (88' pk)
Dumitrescu (84')         

Preud'homme, Medved, Grun, Albert, Smidts, Borkelmans (Oliveira 71'), Staelens, 
Van der Elst, Scifo, Boffin, Wilmots (Czerniatynski 78')

November 17 1993; Anderlecht, Parc Astrid
Belgium         0-0      Czechoslovakia Rep.

De Wilde, Medved, De Wolf, Albert, Smidts, Staelens, Scifo, Van der Elst, 
Versavel, Oliveira (Boffin 52'), Nilis (Czerniatynski 80')
\subsection{Belgium's Record Against Other Qualifiers}
\begin{figure}[H]
\begin{tabular}{l c c c c c c}
& P & W & L & D & GF & GA \\
Argentina & 4 & 1 & 3 & 0 & 4 & 10 \\
Brazil & 3 & 1 & 2 & 0 & 6 & 8 \\
Bulgaria & 9 & 4 & 4 & 1 & 16 & 13 \\
Germany & 19 & 9 & 14 & 1 & 22 & 44 \\
Greece & 3 & 1 & 1 & 1 & 2 &  2 \\
Ireland & 12 & 4 & 4 & 4 & 22 & 22 \\
Italy &16 & 2 & 11 & 3 & 15 & 32 \\
Mexico & 5 & 2 & 3 & 0 & 6 &  4 \\
Netherlands & 116 & 38 & 53 & 25 & 203 & 262 \\
Norway & 4 & 3 & 0 & 1 & 7 &  2 \\
Romania & 8 & 4 & 3 & 1 & 12 & 9 \\
South Korea & 1 & 1 & 0 & 0 & 2 &  0 \\
Russia (USSR) & 5 & 1 & 4 & 0 & 5 &  10 \\
Spain & 16 & 5 & 6 & 5 & 19 & 28 \\
Sweden & 12 & 5 & 5 & 2 & 18 & 29 \\
Switzerland & 26 & 12 & 8 & 6 & 49 & 37 \\
USA & 1 & 0 & 0 & 1 & 0 & 3 \\
\end{tabular}
\end{figure}
Belgium have never played Bolivia, Cameroon, Colombia, Morocco, Nigeria 
and Saudi Arabia.
\section{Key Players}
MICHEL PREUD'HOMME (Goalkeeper, KV Mechelen):
Born: Ougree (Liege) 24 January 1959
1,82 m; 75 kg
former club: Standard
junior international
49 times international (1st time: 2nd of May 1979, Austria-Belgium)
Golden Shoe 1987 and 1989
Took part in WC 90

GEORGES GR{\"U}N (Defender, Parma(ITA)):
Born: Etterbeek (Brussels) 25 January 1962
1,85 m; 73 kg
former (and next) club: RSC Anderlecht
67 times international (1st time: 13th of June 1984, Belgium-Yugoslavia)
3 goals for Red Devils
Took part in EC 84, WC 86 and 90

ENZO SCIFO (Midfield, AS Monaco(FRA)):
Born: Haine-Saint-Paul 19 February 1966
1,75 m; 70 kg
former clubs: La Louviere, RSC Anderlecht, Inter Milan, Bordeaux, Auxerre, Torino
62 times international (1st time: 6th of June 1984, Belgium-Hungary)
Golden Shoe 1984
15 goals for Red Devils
Took part in EC 84, WC 86 and 90

MARC DEGRYSE (Forward, Anderlecht):
Born: Roeselare 4 September 1965
1,72 m; 70 kg
junior international
Former clubs: Ardooie, Club Bruges
46 times international (1st time: 5th of May 1984, Belgium-Argentina) 
13 goals for the Red Devils
Golden shoe 1991

LUC NILIS (Forward, Anderlecht):
Born: Hasselt 25 May 1967
1,83 m; 77 kg
former clubs: Halvenberg Zonhoven, Winterslag
23 times international (1st time: 26th of March 1988, Belgium-Hungary)
\section{1994 Friendlies}
\begin{figure}[H]
\begin{tabular}{l l l c}
Feb 16 & A & Malta & 0-1 \\
Jun 4 & H & Zambia & 9-0 \\
Jun 8 & H & Hungary & 3-1 \\
\end{tabular}
\end{figure}
