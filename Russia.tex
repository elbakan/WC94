\chapter{Russia}
\chapterprecis{Reproduced from UPI Reports}
\newline
\newline
Coach: Pavel Sadyrin
\begin{figure}[H]
\begin{tabular}{c c l l c c c}
P & & & Club & Age & Caps & Goals \\ \hline
G & 1 & Stanislav Tsjertsjesov & Dynamo Dresden (Ger) & 30 & 20 & 0 \\
G &16 & Dimitrij Kharin &  Chelsea (Eng) & 25 & 20 & 0 \\ \hline
D & 3 & Sergej Gorlukovitsj & Bayer Uerdingen (Ger) & 32 & 31 & 1 \\
D & 4 & Dimitrij Galjamin & Espanyol (Spa) & 31 & 27 & 0 \\
D & 5 & Jurij Nikiforov & Spartak Moscow & 23 & 6 & 0 \\
D & 6 & Vladislav Ternavskij & Spartak Moscow & 25 & 1 & 0 \\
D &18 & Viktor Onopko & Spartak Moscow & 25 & 19 & 1 \\
D & 21 & Dimitrij Khlestov & Spartak Moscow & 23 & 12 & 0 \\ \hline
M & 2 & Dimitrij Kusnetsov & Espanol (Spa) & 27 & 24 & 2 \\
M & 7 & Andrej Pjatniski & Spartak Moscow & 26 & 9 & 3 \\
M & 8 & Dimitrij Popov & Racing Santander (Spa) & 27 & 10 & 3 \\
M &12 & Omar Tetradse & Dynamo Moscow & 24 & 9 & 0 \\
M &13 & Aleksander Borodjuk & Friburg (Ger) & 31 & 11 & 5 \\
M &14 & Igor Kornejev & Espanyol (Spa) & 26 & 13 & 3 \\
M &17 & Ilja Tsimbalar & Spartak Moscow & 25 & 2 & 1 \\
M & 20 & Igor Ledjakov & Spartak Moscow & 26 & 14 & 1 \\ \hline
F & 9 & Oleg Salenko & Logrones (Spa) & 24 & 4 & 0 \\
F & 10 & Valerij Karpin & Dynamo Moscow & 25 & 10 & 1 \\
F & 11 & Vladimir Besjastnykh & Spartak Moscow & 20 & 5 & 0 \\
F & 15 & Dimitrij Radsjenko & Racing Santander (Spa) & 23 & 12 & 3 \\
F & 19 & Aleksander Mostovoj & Caen (Fra) & 25 & 19 & 5 \\
F & 22 & Sergej Yuran & Benfica (Por) & 25 & 26 & 7 \\ \hline
\end{tabular}
\end{figure}
\section{WC qualifying}
\begin{figure}[H]
\small
\begin{tabular}{l l c l l}
1992 & & & & \\
Oct 14 & Russia & 1-0 & Iceland & Yuran 64' \\
Oct 28 & Russia & 2-0 & Luxembourg & Yuran 4', Ratchenko 23' \\
1993 & & & & \\
Apr 14 & Luxembourg & 0-4 & Russia & Kiryakov 11', 46', Shalimov 58', Kulkov 90' \\
Apr 28 & Russia & 3-0 & Hungary & Kanchelskis 55', Kolivanov 60', Yuran 86' \\
May 23 & Russia & 1-1 & Greece & Dobrovolski 70'; Mitropoulos 45' \\
Jun 2 & Iceland & 1-1 & Russia & Sverrisson 26'; Kiryakov 38' \\
Sep 8 & Hungary & 1-3 & Russia & Nikiforov (og) 20'; Pyatnichki 14',Kiryakov 53', Borodyuk 90' \\
Nov 17 & Greece & 1-0 & Russia & Mahlas 68' \\
\end{tabular}
\normalsize
\end{figure}
\section{Friendlies}
\begin{figure}[H]
\begin{tabular}{l l l c l}
1993 & & & & \\
Jul 28 & France & Lost & 1-3 & Cannes \\
Sep 8 & Hungary & Won & 3-1 & Budapest \\
Oct 10 & Saudi Arabia & Lost & 2-4 & Damman \\
1994 & & & & \\
Jan 29 & USA & Drew & 1-1 & Seattle \\
Feb 2 & Mexico & Won & 4-1 & Oakland \\
Mar 23 & Ireland & Drew & 0-0 & Dublin \\
Apr 20 & Turkey & Won & 1-0 & Bursa \\
May 29 & Slovakia & Won & 2-1 & Moscow \\
\end{tabular}
\end{figure}
\section{Preview}
Russia's midfield has changed completely since the qualifying tournament,
lacking Shalimov, Kanchelskis and Kolyvanov. The first-choice lineup will most 
likely contain Ilya Tsymbalar on the left, Andrei Pyatnitsky in the middle and 
Dmitri Popov on the right. The ideal front line for Sadyrin will be Racing's 
Dmitri Radchenko and Alexander Mostovoi providing support for No. 1 spearhead 
Oleg Salenko of Logrones. If this does not work, Sadyrin always can turn to 
Sergei Yuran of Benfica.

The Soviet national team has achieved greater success in the European 
Championships than in the World Cups. It became the first European champion in
1960, and has finished second three times since. Although it has played in six 
World Cups, the USSR managed to reach the World Cup semifinals only once, in 
1966 in England.

Sadyrin can only dream of a squad like Valery Lobanovsky's Soviet team, which
made one of the brightest impressions in the 1986 finals in Mexico and then
reached the final of the European Championships two years later, losing to
Holland.

The crux of Lobanovsky's team was one club -- Dynamo Kiev -- and that meant
good mutual understanding between the players, a thing the current Russian 
squad is lacking even more than speed or technique. Getting ready for the World 
Cup, Sadyrin has never been able to assemble his squad for more than a few days 
before the match and the often squalid living conditions at training camps have
hardly been conducive to developing team spirit. Before the loss to Greece, 
players had to train on a snowy pitch and sleep fully-clothed in unheated rooms.

In its World Cup group B, Russia is probably stronger than Cameroon, equal to
Sweden and is obviously weaker than the group seed Brazil. ``Brazil is the World 
Cup favorite,'' said Sadyrin. ``We know it and we are realistic comparing our and 
their capabilities.'' ``I don't see superstars in our team now, even among those 
who refused to play,'' he admitted. ``And Brazil has many world-class players.''
``Our main goal is to move to the next round,'' said Sadyrin, adding he wouldn't 
make any further forecasts. Sadyrin believes that high spirits and strong will 
are now as important as ever to his crippled team. ``For such an event like the 
World Cup a maximum of self-sacrifice is needed,'' he said. ``Only those people 
who want to defend the honor of our soccer can play in the national team. 
Eagerness is much more important to us than technical qualities.''

Soccer was introduced to Czarist Russia in 1879, when Englishmen taught the
game to students in St. Petersburg. Russia joined FIFA in 1912 but five years
later disbanded its national team after the Bolshevik revolution. Shortly after 
the Soviet Union collapsed in 1991, Russia's membership in FIFA was restored 
and its national team record resumed. On Aug. 16, 1992 Russia played its first 
match, beating Mexico 2-0 in Moscow, and has lost just two matches since.
\section{Key Players}
VIKTOR ONOPKO (Midfield, Spartak Moscow) 25 years old, 20 caps:
Being Moscow Spartak's obvious leader, he is one of the few Russian players 
capable of taking charge of a game. Very hard-working and amazingly versatile. 
Very good in the air. He was voted Russia's player of the year in 1993 and is 
likely to be team captain in the United States.

DMITRI RADCHENKO (Forward, Racing Santander(SP)) 23 years old, 13 caps:
May become the team's leading striker if he performs to his full potential. An 
expressive, emotive player but doesn't enjoy the physical side of the game.

SERGEI YURAN (Forward, Benfica(POR)): 25 years old, 27 caps:
Bright, rapid striker who hasn't really found the form that made him shine in
the '92 European Championships. Not the most gifted of players but full of
running and aggression. Although good with both feet he prefers the left side.
