\chapter{Norway}
\chapterprecis{Erik Boman \& Odd-Magne Sekkingstad}
\newline
\newline
Coach: Egil ``Drillo'' Olsen
\begin{figure}[H]
\begin{tabular}{c c l l c}
P & & & Club & Age \\ \hline
G & 1 & Erik Thorstvedt & Tottenham Hotspur (Eng) & 31 \\
G & 12 & Frode Grod{\aa}s & Lillestrom & 29 \\
G & 13 & Ola By Rise & Rosenborg & 33 \\ \hline
D & 2 & Gunnar Halle & Oldham (Eng) & 28 \\
D & 3 & Erland Johnsen & Chelsea (Eng) & 27 \\
D & 4 & Rune Bratseth & Werder Bremen (Ger) & 33 \\
D & 5 & Stig Inge Bj{\o}rnebye & Rosenborg & 24 \\
D & 14 & Roger Nilsen & Sheffield United (Eng) & 24 \\
D & 15 & Karl Petter Loken & Rosenborg & 27 \\
D & 18 & Alf-Inge Hal{\aa}and & Nottingham Forest (Eng) & 21 \\
D & 20 & Henning Berg & Blackburn (Eng) & 24 \\ \hline
M & 6 & Jostein Flo & Sheffield United (Eng) & 29 \\
M & 7 & Erik Mykland & Start & 22 \\
M & 8 & {\O}yvind Leonhardsen & Rosenborg & 23 \\
M & 10 & Kjetil Rekdal & Lierse (Bel) &  25  \\
M & 11 & Jahn Ivar Jakobsen & Young Boys (Swi) & 28 \\
M & 17 & Dan Eggen & Brondby (Den) & 23 \\
M & 19 & Roar Strand & Rosenborg &  24 \\
M & 22 & Lars Bohinen & Nottingham Forest (Eng) & 24 \\ \hline
F & 9 & Jan {\AA}ge Fj{\o}rtoft & Swindon (Eng) & 27 \\
F & 16 & Goran Sorloth & Bursaspor (Tur) & 31 \\
F & 21 & Sigurd Rushfeldt & Troms{\o} & 21 \\ \hline
\end{tabular}
\end{figure}
\section{History}
Football came to Norway late in the 19th century. The first Norwegian football
club was Christiania Fodboldsklub founded 28th Mai 1885. NFF, Norges Fotball
Forbund, (The Norwegian Football Association) was founded 30th April 1902. In
1908 NFF became a member of FIFA, and Norways national team played their first
official match, against Sweden.
\begin{figure}[H]
\begin{tabular}{l c c c c c c}
& GP & W & D & L & GF & GA\\
Norwegian National Team & 560 & 171 & 124 & 265 & 829 & 1107 \\
With ``Drillo'' as coach & 38 & 19 & 12 & 7 & 74 & 26 \\
\end{tabular}
\end{figure}
\subsection{Olympics}
\begin{figure}[H]
\begin{tabular}{l c c l l}
Stockholm & 1912 & 0-7 & Denmark & \\
Antwerpen & 1920 & 3-1 & England & \\
& & 0-4 & Czechoslovakia & \\
Berlin & 1936 & 4-0 & Turkey & \\
& & 2-0 & Germany & \\
& & 1-2 & Italy & (after extra time) \\
& & 3-2 & Poland & Won Bronze Medal \\
Moscow & 1980 & & Qualified (boycott) \\
Los Angeles & 1984 & 0-0 & Chile & \\
& & 1-2 & France & \\
& & 2-0 & Qatar & \\
\end{tabular}
\end{figure}
\subsection{World Cup--From the Story of the World Cup by Brian Glanville}
Norway have been in one WC in 1938.  The Norwegians lost to Italy 1-2 after extra time.
Italy were very nearly knocked out at once: in Marseilles, by the Norwegians, 
who had given their Olympic team an arduous run for their money. Norway, 
playing with six of the team which had lost only 2-1 to Italy in the Olympic 
semi-final, were a goal down in only the second minute. Piola found Ferrari, 
whose shot was dropped by the Norwegian goalkeeper. Ferraris II, the left 
winger, shot the ball home. R. Johansen, the Norwegian right-back, now
indicated Piola to his center-half, Eriksen, who nodded and dropped back to
dedicate himself successfully to the big center-forward, Henriksen, the little
right-half, taking his place in midfield. The pendulum swung. Brunyldsen, the 
mighty center-forward, now began to set dreadful problems for the Italian 
defence. He was well abetted by his fast, direct left-winger, Brustad, and 
Kwammen, a composed inside-right. Three times post and bar were hit, and 
finally Brustad, in the second half, received from Brunyldsen, cut inside 
Monzeglio, and equalised. Soon afterwards, Brustad had the ball in the
net again, to be given offside; and just before time, Olivieri made his famous
save from Brunyldsen, whom Pozzo called ``a cruel thorn in my crown of roses''.
Five minutes into extra time, Piola at last evaded the Norwegian defence,
when Paserati shot. Again H. Johansen could only block, and the center-forward
scored. Italy had survived their hardest match of the tournament.
\subsection{Football in Norway Today}
Today football is the largest and most popular sport in Norway. With almost
600,000 members in the football federation and 2,000 clubs, football is the 
number one sport in Norway. In Norway, football is played by everybody, young 
and old, men and women. Norways national women team have won the EC twice, and 
been second in the WC. Many people believe that all sorts of winter sports are 
our favorite sports, but those are rather small in Norway compared to football.
\section{World Cup 1994 Qualifying}
After Norway had been drawn into the same group as The Netherlands, England,
Poland and Turkey, there weren't many who believed that Norway would manage to
qualify for the WC. Our coatch ``Drillo’’ Olsen said that the chances were about 
20\%, he has always been quite realistic. Norway got a flying start in 1992 by 
thrashing San Marino scoring double figures, and after that Norway rode on a 
wave of success throughout the qualifying round.

Norway - San Marino 10-0 (4-0)
9. September 1992
Ullevaal stadion
N: Rekdal (4,78), Halle (6,50,68), Sorloth (15,22), Nilsen (46,66), Mykland (73)
S: -
A: 6 511

Norway - Netherlands 2-1 (1-1)
23. September 1992
Ullevaal Stadion
No: Rekdal (penalty.9), Sorloth (78)
Ne: Bergkamp (10)
A: 19 988

San Marino - Norway 0-2 (0-2)
7. October 1992
Stadio Olimpico di Serravalle
S: -
N: Jakobsen (7), Flo (19)
A: 1 187

England - Norway 1-1 (0-0)
14. October 1992
Wembley
E: Platt (55)
N: Rekdal (77)
A: 51 441

Norway - Turkey 3-1 (2-0)
27. April 1993
Ullevaal stadion
N: Rekdal (14), Fjortoft (16), Jakobsen (53)
T: Feyyaz (55)
A: 21 530

Norway - England 2-0 (1-0)
2. Juni 1993
Ullevaal stadion
N: Leonhardsen (42), Bohinen (47)
E: -
A: 22 256

The Netherlands - Norway 0-0 (0-0)
9. Juni 1993
Feyenoord Stadium, Rotterdam
Ne: -
No: -
A: 50 000

Norway - Poland 1-0 (0-0)
22. September 1993
Ullevaal stadion
N: Flo (53)
P: -
A: 21 968

Poland - Norway 0-3 (0-0)
13. October 1993
KKS Lech stadium, Poznan
P: -
N: Flo (64), Fjortoft (67), Johnsen (90)
A: 11 000

Turkey - Norway 2-1 (2-0)
10. November 1993
Fenerbahce Stadium, Istanbul
T: Ertugrul (5,26)
N: Bohinen (48)
A: 15 000
\begin{figure}[H]
\begin{tabular}{l c c c c c c c}
European Group 2 & GP & W & D & L & GF & GA & Pts \\ \hline
\textbf{Norway} & 10 & 7 & 2 & 1 & 25 & 5 & 16 \\
\textbf{Netherlands} & 10 & 6 & 3 & 1 & 29 & 9 & 15 \\
England & 10 & 5 & 3 & 2 & 26 & 9 & 13 \\
Poland & 10 & 3 & 2 & 5 & 10 & 15 & 8 \\
Turkey & 10 & 3 & 1 & 6 & 11 & 19 & 7 \\
San Marino & 10 & 0 & 1 & 9 & 2 & 46 & 1 \\ \hline
\end{tabular}
\end{figure}
\section{Preparation}
Faeroe Islands - Norway 0-7 (0-4)
11. August 1993
Tofta Leikvollir, Toftir
F: -
N: Lars Bohinen (pen.6), Oyvind Leonhardsen (9), Mons Ivar Mjelde (27, 39) -
   Egil Ostenstad (74, 87), Jan Ove Pedersen (75)
A: 800

Norway - USA 1-0 (1-0)
8. September 1993
Ullevaal stadium, Oslo
N: Stig Inge Bj|rnebye (14)
U: -
A: 16 248

USA - Norway 2-1 (0-1)
15 January 1994
?
U: Marcelo Balboa (55), Cobi Jones (90)
N: Frank Strandli (90)
A: 15 386

Norway - Costa Rica 0-0
19. January 1994
Jack Murphy Stadium, San Diego
N: -
C: -
A: 50 000

Wales - Norway 1-3 (0-1)
9. Mars 1994
Ninian Park
W: Chris Coleman (90)
N: Jostein Flo (6), Erik Mykland (50), Jahn-Ivar Jakobsen (51)
A: 10 000

Norway - Portugal 0-0
20. April
Ullevaal stadion
N: -
P: -
A: 17 509

England - Norway 0-0
22. May
Wembley
E: -
N: -
A: 63 327

Norway-Denmark 2-1 (2-1)
1. June
Ullevaal stadion
N: Jahn Ivar Jakobsen (35), Henning Berg (45)
D: Flemming Povlsen (42)
A: 26 123

Sweden - Norway 2-0 (0-0)
5. June
Rasunda, Stockholm
S: Tomas Brolin (56, pen 62)
N: -
A: 29 961
\section{Predictions}
In the WC, Norway depends on their defence which is one of the best in the 
World. Norway conceded only five goals in the ten qualifying matches (two were 
against Turkey after Norway had qualified), and in seven friendlies they have 
only three goals against. Still, nobody will win the WC without some sort of 
attack, and this will probably be Norway's biggest problem in the USA. Even if 
Norway don't have a great attack, they don't need to score many goals since 
their defence is so strong. It is a long time since anybody has scored more 
than one goal against Norway in an important match. From this perspective we 
believe that Norway will go on to the second round, but from there on it's 
hard to say how far Norway can go.

A quick look at group E: Italy, Mexico, Ireland, Norway.
This is the ``group of death'', definitely the toughest one. Currently all four 
teams are ranked among the top 13 in the world by FIFA. Norway has had a very 
good grip on Italy over recent years, with two wins and a draw in the last 3 
games, but that means the Italians know the Norwegian style by now and will not 
underestimate them again. Expect a close game here, a 0-0 tie is quite likely.

Mexico is a new opponent for Norway, but Norway's physical defence-first style 
works very well against Latin-American teams so we think Norway has a good 
chances of winning this one.

The Ireland game comes last and could be the decisive game for both teams. 
These teams have a very similar style, and with advancement at stake for both 
teams it could turn out to be a ``blood and knuckles'' game. We expect another 
close match with few goals.
\section{Coach Egil ``Drillo'' Olsen}
Born: 22. April 1942
Played 16 times for the national team.
National coach from 14. October 1990.

Ask any Norwegian what is the main reason for Norway qualifying for the WC 
finals for the first time in 56 years and he will answer: ``Drillo''. ``Drillo'' 
is the nickname of the Norwegian coach Egil Olsen. He earned it when he was a 
player in the sixties, where he was a true dribbling wizard (by Norwegian 
standards at least). There are stories about how he would pass a ``tunnel'' 
through his opponents legs, then deliberately wait for the defender to come 
back in position so he could try to repeat the trick! Once he juggled the ball 
for fun for 5 straight hours, that was 32,000 hits. It seems therefore to be a 
paradox that this man is now a coach deeply committed to tactics and analyzing 
games in painstaking detail. But there is really no contradiction here (see 
separate discussion of team tactics).

Egil Olsen is the first Norwegian to have a university degree in football. He 
is currently a senior lecturer at the University of Sport and Physical 
Education in Oslo. His coaching is based on his academic skills. After every 
game, he spends hours watching the videotape over and over again. He carefully 
registers how many times each player touched the ball, in what position, what 
did the player do with the ball, was it a positive or a negative touch, etc.
This information is processed and stored in Olsen's mind and notes and he can 
make player selections based on this data (e.g. who is best at using the left 
foot when playing on the right midfield and facing own goal). His decisions are 
made on a statistical and analytical background, not on feelings or intuition.

``Drillo'' was a club coach for local club teams for years and then coached the 
Norwegian under-21 national team and the Olympic team before he was suddenly 
chosen to head the ``real'' national team when his predecessor withdrew in 1990. 
He made few changes on the team he took over, but his confidence in his 
tactical strategy and emphasis on team spirit lifted the team to a level Norway 
has not seen in modern times. From being ranked along with Malta and Luxembourg, 
Norway was ranked at number 4 on the official FIFA world ranking in 1994. His 
record as head coach is a stunning 19-12-7 (W-D-L) in 38 games (as of May 23, 
1994).

In private ``Drillo'' is a shy and sober man who enjoys walking his dog. 
Politically it is known that he has left-wing sympathies, he was once a member 
of the Marxist-Leninist Party. But today he is only known as the man who led 
Norwegian football into a new era.
\section{tactics}
Coach ``Drillo'' Olsen has a tactical strategy which is of great importance to 
the Norwegian team. The most important factors are:
\begin{itemize}
\item Zone defence
\item aggressivity and fast attacks
\item usually 4-5-1 formation
\end{itemize}
In the first games ``Drillo'' Olsen used a 3-5-2 formation but it turned out to 
be too difficult for the 3 backs to cover the whole width when playing zone 
defence. Since he strongly believes in a pure zone defence he switched to 4-5-1. 
The zone defence with all backs on a line is very demanding, but has been 
developed to near perfection by central defenders Rune Bratseth and Tore 
Pedersen. A solid defence is the cornerstone of the team.

The general philosophy is very simple: When you are defending, make sure to 
have as many players as possible behind the ball, always put pressure on the 
opposition when they are on the ball and try to win the ball from them. Once
possession has been regained, counter-attack quickly whenever feasible. Most 
goals are scored after possession has been won and few passes have been used. 
The break-through is the second most important thing in football and often leads 
to the ultimate achievement: the goal.

Consequently, you will not see Norway try to establish an attack by endless 
short passes. If the situation is right you may see wall-passes or solo break-
aways through the middle by small, technical players like Erik Mykland and 
Lars Bohinen. But more often the team resorts to long passes to the striker 
(Jan Age Fjortoft) or the midfield winger (Jostein Flo). If the pass succeeds, 
the team is substantially closer to the goal. If it fails, at least possession 
was lost high up on the pitch and there is always a chance to break back and 
recover. With only one striker, the midfielders must follow up and come running
from behind whenever the ball is pushed up. This has proven to be very efficient
and most of Norway's goals have come in this way. The central midfielders 
(Kjetil Rekdal and Oyvind Leonhardsen) have an enormous running capacity, they 
work like a steam engine for 90 minutes.

Drillo is constantly refining the tactics. One of his recent pet projects is 
the rapid sideways movement. The idea is to move the entire team closer to the 
flank where the ball is (when defending). The opposite flank is left wide open, 
so if a cross pass occurs, the whole team has to rush over to the other side.  
The advantage is that you have a higher density of players in the area around 
the ball and thus a better chance to win possession. The disadvantage is that 
the strategy is very energy consuming. It looks funny to see players suddenly 
start sprinting when they are nowhere near the ball, but tactics like this are 
the future of football.

The success of the Norwegian team can be explained by a coach the players 
respect and admire and a tactical system which fits the players well. Critics 
claim it is too defensive and anti-constructive. Drillo responds that you do 
not win a football game by having ball possession, only the number of goals 
count. With a rock solid defence, Norway frequently have more shots on goal and 
more scoring chances than their opponents even if the other team has possession 
60-70\% of the time.

Drillo knows perfectly well that his players cannot compete with World-class 
teams in technical skills, and so the team spirit and the tactics compensate 
and make the total product a very good football team. But, as Drillo himself 
puts it: ``If Brazil had played with our system, they would have been invincible.''
\section{Key Players}
RUNE BRATSETH (Defender, Werder Bremen(GER)):
Age/caps/goals: 33/57/4
Clubs: Werder Bremen,Germany, Rosenborg,Norway, Nidelv,Norway
Merits: Won Bundesliga and German cup twice, EC2, all with Werder.
        Ranked best foreign player in Bundesliga twice.

Bratseth is the highly respected captain of the Norwegian team. He is the most 
successful Norwegian football player of all times. He is very tall (193 cm) and 
very fast, which is a rare combination. He plays in the centre of defence, 
although at Werder he often plays libero, but the Norwegian team plays a pure 
zonal defence with no libero. Bratseth ``reads’’ the game and the plays very well,
he knows where to position himself and he chooses the best option of what to do 
99% of the time. He is utterly reliable and never makes a real blunder. Add to 
this his speed (he is faster than most of the forwards he faces) and his 
excellent skills with his head, and you have a picture of a World-class player. 
Bratseth plays a key role in the Norwegian defence which has proven to be one 
of the best in the world.

Rune Bratseth is a leader both on and off the pitch. He is sober, down-to-earth,
and an active Christian. He is in many ways a role model parents want their kids
to look up to. He strongly follows his beliefs: When a Norwegian brewery 
started running ads with pictures of the Norwegian football team, he said stop
(alcohol and sports don't mix in Norway). ``Either you stop this campaign or I 
won't play in the WC’94’’ he said. Guess who had to give in...

Buying Rune Bratseth is probably the best deal Werder Bremen has ever done. He 
cost them only about \$100,000 when they bought him from the Norwegian semi-
professional club Rosenborg in 1986. At the height of his career, he got offers 
from several Italian clubs but turned them all down. His family life (wife 
and 2 kids) and his loyalty to Werder counted more than billions of lire. Now, 
after 7 years in Germany, he celebrated his last game by leading Werder to 
victory in the cup final. Now he faces his final challenge as a player: To lead 
the Norwegian squad in their first WC appearance since 1938. After that he 
retires and will move back to Norway and work as a sports executive at his old
club Rosenborg.


ERIK THORSTVEDT (Goalkeeper, Tottenham(ENG)):
Age/caps/goals: 31/84/0
Clubs: Tottenham (Eng), IFK G{\”o}teborg SWE, Borussia Monchengladbach (Ger),
       Eik (Nor), Viking (Nor)
Merits: Won the FA cup with Tottenham.

Thorstvedt has been the no. 1 goalkeeper in Norway for almost a decade now, his 
82 caps makes him the most experienced player on the team. He is a good all-
round 'keeper. and his height (194 cm) and long arms makes him a natural talent 
for this position. He has quick reactions and is also good in one-on-ones in 
the penalty box. However, as any Spurs supporter knows, he makes a terrible 
blunder once in a while. Such an incident occurred in his first game at White 
Hart Lane, so he was called ``Erik the Horrible''. But not for long, he has been 
a key player in Tottenham's side over the last few years and is now known as 
``Erik the Viking''.


JOSTEIN FLO (Midfield/Forward, Sheffield Utd(ENG)):
Age/caps/goals: 29/23/7
Clubs: Sheffield United (Eng), Sogndal (Nor), Lierse (Bel), Molde (Nor),
       Stryn (Nor)
  
Jostein Flo is yet another tall player (194 cm) and he definitely plays best 
when the ball hits his head and not his feet. A few years ago, his critics 
claimed that he could only use his head, but Flo has improved his technical 
skills at a mature age and is now a reasonably good player also with his feet. 
On his club teams, Flo has normally played forward (like a good old tank centre). 
Coach ``Drillo'' Olsen put Flo on the national team in the WC qualifiers against 
San Marino and Turkey, based on the philosophy that a player like him would be 
useful against short opponents. Surprisingly to many observers, Flo played very 
well and earned a spot on the team in the other qualifiers. However, Drillo
put him on the right wing of the midfield. This turned out to be very 
successful, the long cross passes from left back Bjornebye to Flo was an
important feature of the Norwegian style.

JAN {\AA}GE FJ{\O}RTOFT (Forward, Swindon(ENG)):
Age/caps/goals: 27/50/15
Clubs: Swindon (Eng), Rapid Wien (Austria), Lillestrom (Nor), Ham-Kam (Nor),
       Hodd (Nor), Gursken (Nor)

Coach Drillo Olsen's main head-ache is that Norway has no top world-class
forward. That is a big problem when you are using a 4-5-1 formation. During 
the WC qualifiers, Fjortoft has emerged as the first choice for this spot but 
nothing is certain. Fjortoft is quite quick, knows how to dribble, but is not 
very strong or powerful. His raids often ends in a free-kick (he is quite good
at getting those :-) but he frequently he loses the ball, too. However, he has 
the ability of a goalgetter to be in the right place at the right time. He was 
a topscorer in Norway for Lillestrom and scored many goals for Rapid, too.  


KJETIL REKDAL (Midfield, Lierse(BEL)):
Age/caps/goals: 25/32/5
Clubs: Lierse (Bel), Molde (Nor), Fiksdal/Rekdal (Nor)

Rekdal is the ``steam engine'' on the midfield. A hard working guy with large 
defensive responsibilities. Not brilliant, not the type you immediately notice, 
but still very important for the team. Famous in the UK for his ``dream shot'' 
in the WC qualifier at Wembley which gave Norway the 1-1 tie.


ERIK MYKLAND (Midfield, Start):
Age/caps/goals: 22/25/2
Clubs: Start (Nor), Bryne (Nor), Risor (Nor)

Mykland is one of the exciting young players on the team. He is very small 
(<170 cm) but technically brilliant. In Norway he is called ``Myggen'' which 
means ``The Mosquito''. He showed his talent at early age and has played on the 
national teams for juniors and Under-21 before he made it to the real A-team. 
His strengths are good dribbling skills and energetic checking of the opponents
which frequently results in winning possession. On a good day he is very
creative and can make those unexpected passes which penetrate the defence. 
Unfortunately, he is quite moody and the quality of his play can be anything 
from poor to excellent. Mykland is still a ``non-amateur'' at Start in Norway, 
and would probably be one of the best buys in Europe today.  

LARS BOHINEN (Midfield, Nottingham Forest(ENG)):
Age/caps/goals: 24/29/7
Clubs: Nottingham Forest (Eng), Lillestrom (Nor), Young Boys (Swi),
       Valerenga (Nor)

Bohinen is another young, technically gifted player. He is, in many ways similar 
to Mykland, so sometimes only one of them is allowed to play. He is also quite 
up-and-down. On a bad day he is so anonymous that you don't notice him at all, 
but on a good day he dominates the midfield. He knows some nice ``fakes'' and 
can carry out astonishing dribbling raids from time to time.  Most famous is 
his goal against Italy in a EC'92 qualifier where he rounded the whole defence 
(including Baresi) before putting the ball in the net. This gave Norway a 
stunning 2-1 upset win.
