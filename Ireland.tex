\chapter{Ireland}
\chapterprecis{William McClatchie}
\newline
\newline
Coach: Jack Charlton (Member of 1966 World Cup Winning Team)
\begin{figure}[H]
\begin{tabular}{c c l l c c c}
P & & & Club & Age & Caps & Goals \\ \hline
G & 1 & Pat Bonner & Glasgow Celtic (Sco) & 34 & 72 & 0 \\
G & 22 & Alan Kelly & Sheffield United (Eng) & 25 & 3 & 0 \\ \hline
D & 2 & Denis Irwin & Manchester United (Eng) & 27 & 26 & 1 \\
D & 3 & Terry Phelan & Manchester City (Eng) & 27 & 21 & 0 \\
D & 4 & Kevin Moran & Blackburn Rovers (Eng) & 38 & 71 & 6 \\
D & 5 & Paul McGrath & Aston Villa (Eng) & 34 & 64 & 7 \\
D & 12 & Gary Kelly & Leeds United (Eng) & 19 & 4 & 1 \\
D & 13 & Alan Kernaghan & Manchester City (Eng) & 27 & 10 & 1 \\
D & 14 & Phil Babb & Coventry City (Eng) & 23 & 4 & 0 \\ \hline
M & 6 & Roy Keane & Manchester United (Eng) & 22 & 21 & 0 \\
M & 7 & Andy Townsend (c) & Aston Villa (Eng) & 30 & 44 & 4 \\
M & 8 & Ray Houghton & Aston Villa (Eng) & 32 & 58 & 3 \\
M & 10 & John Sheridan & Sheffield Wednesday (Eng) & 29 & 19 & 3 \\
M & 11 & Steve Staunton & Aston Villa (Eng) & 25 & 46 & 5 \\
M & 17 & Eddie McGoldrick & Arsenal (Eng) & 28 & 11 & 0 \\
M & 18 & Ronnie Whelan & Liverpool (Eng) & 32 & 50 & 3 \\
M & 19 & Alan McLoughlin & Portsmouth (Eng) & 27 & 17 & 1 \\
M & 21 & Jason McAteer & Bolton Wanderers (Eng) & 23 & 4 & 0 \\ \hline
F & 9 & John Aldridge & Tranmere Rovers (Eng) & 35 & 56 & 13 \\
F & 15 & Tommy Coyne & Motherwell (Sco) & 31 & 13 & 4 \\
F & 16 & Tony Cascarino & Chelsea (Eng) & 31 & 49 & 12 \\
F & 20 & David Kelly & Wolverhampton Wanderers (Eng) & 27 & 16 & 7 \\ \hline
\end{tabular}
\end{figure}
\section{History}
The Republic of Ireland football team opened its account in international 
soccer in the 1930's. A team made up of amateurs and professionals, they were 
inconsistent in producing results, though at that stage they were always able 
to pull off a shock or two, like when they became the first non-British team to
beat England at Wembley. Since then they haven't exactly set the World alight.
A series of near misses in qualifying for the World Cup in the 1970s and 
1980s made the Irish people impatient with the team. That was until February 
1986 when The Football Association of Ireland appointed the first ever 
non-Irish manager, Jack Charlton, a World Cup winner as a player with England 
who had also enjoyed success as a manager with Middlesbrough and Sheffield 
Wednesday.

Although he lost his first game in charge, 1-0 to Wales, people noticed that 
the shape of the team was changing. Charlton took the initiative to use the 
FIFA parentage rule to full advantage, and introduced players of high class to 
the team. His first mission was to qualify for the European Championships to be
held in Germany in 1988. Ireland were placed in a tough group with Belgium, 
Bulgaria, Scotland and Luxembourg. Despite some dodgy performances, the Irish 
team achieved good results. However the group was to go right to the wire and 
Ireland only won it because Scotland managed to shock Bulgaria in Sofia.

Thus qualification for Ireland's first ever major tournament sent a wave of 
interest and excitement around the country. People were now beginning to take 
pride and interest in their national team.  

When the summer of 1988 came around, Ireland were in a group with England, 
Russia and The Netherlands. The first game was against England and the English
did not appear to be worried about the underdogs, and took a more relaxed 
attitude to the game. However, to everyone's shock, Ireland scored from a 
header by Ray Houghton after only six minutes. The Irish fans had never 
experienced such euphoria before, to score against the "old enemy" was 
previously unthought of. The team held on for the rest of the match to achieve
a somewhat more that historic victory. The match made overnight heroes of all
those concerned, especially Houghton and Bonner, whose fantastic performance in
goal ensured that the ball would not pass him.

Next on the list were Russia and Ireland put on another commendable performance
and drew 1-1. Ireland opened the scoring with a spectacular 18-yard bicycle 
kick from Ronnie Whelan. Yet the Russians equalised and Ireland felt somewhat 
robbed at the end of the 90 minutes.  

The final group match to be played was against The Netherlands, and victory or 
even a draw would put Ireland through to the semi-finals. Yet it was not to be,
the Dutch scored with a header from Kieft with only six minutes left and 
Ireland were out. Although the Irish didn't get past the group stages, they had
achieved more than they could possibly have dreamed of and when the team 
returned to Dublin they were greeted as if they had won the tournament.

Following the success of 1988 great feats were expected from the team. It was 
thought that they now had a squad that could reach the World Cup Finals for the
first time. In qualification for Italia '90 Ireland were drawn against Spain, 
Hungary, Northern Ireland and Malta. Although Competition was provided courtesy
of Hungary throughout the campaign, the one time 'Mighty Magyars were left 
behind in the end and Ireland finished second in the group behind Spain.

And so December of 1989 brought the draw for the World Cup in Italy, to be held
the following Summer. Already it had been decided that England would be seeded
and put in Sicily and Sardinia. Knowing this, the Irish would have settled for
anything but to be drawn in that group, yet the luck of the Irish failed them
on this occasion and they were drawn in a tough group with England, Holland
and Egypt. The first match against England was an entertaining one with 
Bobby Robson's men taking the lead after 11 minutes through a Gary Lineker 
goal, yet Ireland equalised in the 73rd minute through a smart low drive from 
the left foot of Kevin Sheedy. When the final whistle blew it was a case of 
mission accomplished. Ireland set out for a point, and they got exactly that.

Next on the list was an encounter with Egypt under the blazing afternoon sun of
Palermo. A win was widely expected from the Irish and was demanded by the 
manager, considering that Egypt were thought to be the weakest team in the 
group and Ireland might have been on a roll. While the Egyptians certainly had 
some skill, they weren't really of a standard that would beat Ireland. However,
they were very defiant from the start and it was clear that they had no 
intention of letting Ireland win. 'They played a game of "negative tactics"' 
said Charlton after the frustrating 0-0 draw, which now meant that Ireland 
would have to try and beat the Dutch, then the reigning European Champions. 
That match ended up 1-1, The Netherlands taking an early lead, and Ireland 
equalising late on through Niall Quinn - something which has now become a 
pattern of Irish play against the stronger nations. Three points from the group
matches booked Ireland their place in the last 16 of the tournament. As their 
record was identical to that of the Dutch, lots were drawn to decide second and
third place, and here Ireland had the luck by being drawn as runners-up. This 
meant that they would play Romania in Genoa in the second round, an easier game 
than the Dutch had as they played West Germany.

The Romanian game was satisfactory from an Irish point of view, yet while 
Ireland played the better football they could not convert their chances. The
match went to extra time and then to penalties, a first for the Irish team who 
had never been involved in a shoot-out before. Packie Bonner, the Irish keeper,
had put on superb performances before and during this game and so now his 
status as a world class keeper would be even more scrutinised as he faced five
Romanian penalties. While he let in four, he stopped one, and one was all that 
was needed as veteran David O'Leary scored Ireland's fifth penalty to put them
through to a quarter final meeting in Rome's Olympic Stadium with the hosts,
Italy.

Ireland battled and sometimes struggled against a superior side who were 
playing excellent football. Italy scored the only goal of the game, a neatly 
taken rebound from Salvatore Schillachi after Bonner failed to hold on to 
Donadoni's powerful drive. While the Irish were eliminated there was not a
disappointed atmosphere from the team nor from back home. What they felt was
pride, honour and a sense of achievement.

The campaign for qualification for Euro '92 in Sweden was not a great one. Poor
performances against other teams saw Ireland draw at home to Poland and England
and then drop a lead of two goals to Poland in Poznan. Yet two victories were
achieved against the Turks, and an outstanding performance against England at
Wembley didn't leave the Irish two disheartened. The competition in the group
went right to the end and only a Gary Lineker equaliser for England in Poland
eliminated the Irish. After a sense of disappointment the Irish kept their 
heads up and looked towards USA '94.

In 1992 Ireland were invited to play in a mini tournament in America, to make
up for missing out on the European Championships. Like England's involvement
in the next tournament in 1993 Ireland's was a disappointing one. They opened
the tournament in the Foxboro Stadium with a bitterly and shockingly 
disappointing 3-1 loss to the USA. Then followed a match with Italy which
resulted in another loss for Ireland, a 2-0 loss, and to make matters worse
for the first time in his career Packie Bonner, the Irish goalkeeper was sent
off. His offense was rugby tackling an Italian in the box which was then 
awarded a penalty, which the Italians converted gracefully.
\section{Qualification for the World Cup}
Ireland were drawn in a group which was thought to be extremely tough, as it 
contained Spain, Denmark (the European Champions), Northern Ireland, Albania,
Lithuania and Latvia, the only European group with 7 teams, this meant two 
more matches in the qualifying programme for all group members, which is the
main reason for the recent lack of friendly matches for the Republic.

Although the two Baltic states and Albania were thought to be easy, it was 
always going to be a difficult task in the away games, and the group went right
to the wire in the end, Ireland only qualifying by having scored more goals 
than Denmark, their goal differences were equal.  

Rather than go through Ireland's path chronologically I shall go through it
team by team:

Albania: The Albanians just about got to Dublin to play this match, they had 
barely enough for a full team. They didn't even have a kit with them 
to play in. Luckily they were sponsored and well catered for on their
arrival. The match itself was frustrating for the Irish as only 
towards the end did they manage to get their 2-0 victory. In the away
leg, Albania took a shock early lead, however Ireland quickly 
equalised and then scored the winner late on. Final score 2-1.

Denmark: The match in Copenhagen showed Ireland near their very best. They 
attacked, passed and defended well to earn a 0-0 draw. It was a point
secured rather than one lost, and one that was to prove crucial in the
final reckoning. The game in Dublin was highlighted by the defensive 
error through which Ireland conceded their goal. Niall Quinn got on 
the end of a quick corner to equalise later on in the game though. A
1-1 draw was how the match finished.

Latvia:  When Latvia came to Lansdowne Road they presented Ireland with their 
biggest win in the group. Four goals from Ireland and none from the 
opposition ensured a comfortable victory. The away game was in a 
strange place, in uncomfortable conditions and it was unknown 
territory for Ireland. Even so, Ireland were victorious in a 2-0 win.

Lithuania: The scoreline of the home match did not tell the full story. The 2-0
victory for Ireland does not convey the fact that numerous chances 
were squandered by both sides. The away fixture was of vital importance 
to Ireland. Both Spain and Denmark had dropped a point here so it was
important that Ireland should take advantage. As it happened Ireland 
won 1-0. Mission Accomplished.

Northern Ireland: Within the first third of the home match Ireland were 3-0 up, 
totally outplaying their Northern counterparts in all areas and so 
great things were expected for the second half. Yet characteristically,
the Irish flopped under great expectations and the score finished 3-0. 
The away fixture was the most tense experience any Irish player, 
supporter, or official will experience for a long time. The Irish were
verbally abused, spat on and jeered for the duration of the evening.
The Northern Ireland manager, Billy Bingham, was seen to be inciting 
the crowd. This was a match that Ireland could not lose and so the 
nation's hearts sank when Jimmy Quinn volleyed a fantastic goal past 
Bonner. As always though, Ireland picked themselves up to equalise. 
The equaliser was scored by Alan McLoughlin, the substitute, who has 
now become a national hero. Final score: 1-1.

Spain:   The match in Seville was on one hand a great achievement for Ireland,
but on the other a huge disappointment. Although the Irish got a point
from the 0-0 draw, they missed one great chance and had a perfectly 
legal goal disallowed for offside. When Spain came to Lansdowne Road, 
a draw was the least expected of the Irish, but Spain went 3-0 up 
after half an hour and Ireland could only pull one goal back from a 
performance riddled with incompetence and errors. 3-1 the final score.
\section{Preview}
It's very difficult to predict how Ireland will do after being drawn in a group
with Italy, Mexico and Norway. In the two major tournaments Ireland have played
in to date they have delivered the goods and so a lot is expected of them now.
They will do very well to emulate the success they had in Italia '90 when they
reached the quarter finals, I doubt whether they'll have the luck to get beyond
that. Prediction: 7th to 10th place.
\section{Key Players}
ROY KEANE (Midfield, Manchester United):
Although his recent move to Manchester United for 3.7 million pounds provoked 
shouts of rip-off etc. he is now looking more and more like a World Class 
player. His resilience and amazing stamina make sure he keeps going for the 
entire length of the match. He is young and holds the important role of centre-
midfield alongside captain Andy Townsend, and is one of a handful of players 
that can win a match single-handedly.

ANDY TOWNSEND (Midfield, Aston Villa):
Ireland's captain holds a great deal of influence over his team. Like Keane, he
is a battling attacking midfielder who loves to venture into the box. His
surging runs frequently set up the Irish attack and in what looks to be his 
second and last World Cup he won't want to go out unnoticed. This influential 
kingpin has what it takes to bring out the best in players.

STEVE STAUNTON (Midfield, Aston Villa):
Recently voted Irish Player of the Year this man filled in the gap left on the
left side of midfield after Kevin Sheedy fell out of favour. Originally a full
back, he has adapted to his role superbly well and has created an amazing 
understanding with left back Terry Phelan. He can also score from dead ball
situations.

PAUL McGRATH (Central Defence, Aston Villa):
The 1992-1993 Player of the Year in the English Premier Division has not played
as well this year as he has done in the past. Still, he rarely fails Ireland
and if his knees survive the Summer he will no doubt turn out to be one of the
most important squad members. He is one of the greatest centre-halves playing
in the World today and will play a vital role for Ireland.

The full backs Irwin and Phelan always play an important role and are a great 
strength in the team. The World will know if the Irish really are lucky if 
Niall Quinn is fit in time. As it stands he won't be playing till the new 
season.
\section{1994 Preparation}
\begin{figure}[H]
\begin{tabular}{l l r c l}
Mar 26 & Dublin & Ireland & 0-0 & Russia \\
Apr 20 & Tilburg & Netherlands & 0-1 & Ireland \\
May 24 & Dublin & Ireland & 1-0 & Bolivia \\
May 29 & Germany & Germany & 0-2 & Ireland \\
Jun 5 & Dublin & Ireland & 1-3 & Czechoslovakia \\
\end{tabular}
\end{figure}
