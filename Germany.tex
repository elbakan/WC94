\chapter{Germany}
\chapterprecis{Olaf Hendel}
\newline
\newline
Coach: Hans-Hubert Vogts
\begin{figure}[H]
\begin{tabular}{c c l l c c c}
P & & & Club & Age & Caps & Goals \\ \hline
G & 1 & Bodo Illgner & 1.FC K{\"o}ln & 27 & 47 & 0 \\
G & 12 & Andreas Koepcke & Eintracht Frankfurt & 32 & 14 & 0 \\
G & 22 & Oliver Kahn & Bayern M{\"u}nchen & 24 & 0 & 0 \\ \hline
D &  2 & Thomas Strunz & VfB Stuttgart & 26 & 11 & 0 \\
D &  3 & Andreas Brehme & 1.FC Kaiserslautern & 33 & 79 & 8 \\
D &  4 & Juergen Kohler & Juventus (Ita) & 28 & 63 & 0 \\ 
D &  5 & Thomas Helmer & Bayern M{\"u}nchen & 29 & 23 & 0 \\
D &  6 & Guido Buchwald & VfB Stuttgart & 32 & 73 & 4 \\
D & 10 & Lothar Matth{\"a}us & Bayern M{\"u}nchen & 33 & 110 & 19 \\
D & 14 & Thomas Berthold & VfB Stuttgart & 29 & 51 & 1 \\
D & 17 & Martin Wagner & 1.FC Kaiserslautern & 26 & 2 & 0\\ \hline
M & 7 & Andreas Moeller & Juventus (Ita) & 26 & 39 & 13 \\
M &  8 & Thomas H{\"a}{\ss}ler &  AS Roma (Ita) & 28 & 48 & 6 \\
M & 15 & Mauritio Gaudino & Eintracht Frankfurt & 27 & 4 & 1 \\
M & 16 & Matthias Sammer &  Borussia Dortmund & 26 & 45 & 1 \\
M & 20 & Stefan Effenberg & Fiorentina (Ita) & 25 & 29 & 5 \\
M & 21 & Mario Basler & Werder Bremen & 25 & 3 & 0\\ \hline
F & 9 & Karl-Heinz Riedle & Borussia Dortmund & 28 & 37 & 15 \\
F & 11 & Stefan Kuntz & 1.FC Kaiserslautern & 31 & 2 & 1 \\ 
F & 13 & Rudi Voeller & Olympique Marseille (Fra) & 34 & 86 & 44 \\
F & 18 & J{\"u}rgen Klinsmann & AS Monaco (Fra) & 29 & 58 & 19 \\
F & 19 & Ulf Kirsten & Bayer Leverkusen & 28 & 59 & 3\\ \hline
\end{tabular}
\end{figure}
\section{Preview}
\begin{center}``...together with the new players from the German Democratic Republic, the German National Team will be unbeatable for the near future.''\end{center}
 - Franz Beckenbauer after the WC Final, July 1990.

The success of the German soccer team in the WC90 tournament and this statement 
by ``Kaiser Franz'' were dominating the first period of Berti Vogts' management
after he was appointed as the chief-coach of the German Football Association 
(DFB) in 1990.

His first task was the qualification to the EC-tournament in 1992. Vogts didn't 
change the core-team around Matth{\"a}us (who didn't play during the EC 92 due to 
an injury), Brehme, Buchwald, Voeller, Kohler but he also gave chances to some 
new players. Although they performed very moderately, Germany reached the final 
of the EC tournament in Sweden (1992), but lost against the tournament's
surprise-package, Denmark by 2 goals to nil. This EC-final match was the last 
match under ``real'' competition conditions, because as the title-holder, Germany 
didn't have to qualify for the WC 94. This was one reason for Vogts to play a 
lot of friendlies (21 all together) against strong teams from all over the 
world.
 
Vogts continued his experiments, but most of the new players didn't succeed.
Altogether there have been 39 players used since the EC-final 1992, with 12 of
these making less than three appearances.

One of the most important changes made was to bring Lothar Matth{\"a}us back into
the sweeper position. Doing this has reduced the rivalry in midfield, where the 
German team has a lot of excellent players. A problem in the German team may be 
the forward-line because Karl-Heinz Riedle couldn't fill the gap that Rudi 
Voeller has left. So, Vogts has only one top class striker with international 
experience, Juergen Klinsmann. Stefan Kuntz (top-scorer of the Bundesliga 92/93) 
is injured and Ulf Kirsten didn't impress last season.

One of the most discussed subject in the German soccer-community was the 
goalkeeping position. During the last two years Vogts changed between Bodo 
Illgner and Andreas Koepcke. However, it was Oliver Kahn, who has had a great
season, who was demanded as the No.1. Vogts seemed unable to make up his mind 
and hesitated until the end of May: two days before the friendly against 
Ireland when he appointed Bodo Illgner as the No.1 in the German goal because 
of his international experience (he said it was the most difficult decision in 
his career as chief-coach).

The aim of the German team in the USA is to defend the title successfully 
(the public expectations are very high here in Germany). If they continue the 
old German virtue as a ``tournament-team'', it is possible. Maybe...
\section{Lineups}
\begin{figure}[H]
\begin{itemize}
\item WC Final 1990: Illgner,Augenthaler,Brehme,Kohler,Buchwald,Littbarski, Matth{\"a}us, H{\"a}{\ss}ler,Berthold (Reuter),Klinsmann,V{\"o}ller   
\item EC-Final 1992: Illgner,Helmer,Reuter,Kohler,Buchwald,Brehme,H{\"a}{\ss}ler,Effenberg (Thom),Sammer (Doll),Klinsmann,Riedle
\end{itemize}
\end{figure}
\section{WC Qualifying and Friendlies since June 1993}
\begin{figure}[H]
\begin{tabular}{l l l c l}
1993 & & & & \\
Jun 6 & Brazil & Drew & 3-3 & Washington \\
Jun 13 & USA & Won & 3-4 & Chicago \\
Jun 18 & England & Won & 1-2 & Pontiac, MI \\
Sep 9 & Tunisia & Drew & 1-1 & A \\
Oct 13 & Uruguay & Won & 5-0 & H \\
Nov 17 & Brazil & Won & 2-1 & H \\
Dec 12 & Argentina & Lost & 2-1 & A \\
Dec 18 & USA & Won & 0-3 & A \\
Dec 22 & Mexico & Drew & 0-0 & A \\
1994 & & & & \\
Mar 23 & Italy & Lost & 2-1 & H \\
Apr 27 & UAE & Won & 2-0 & A \\
May 29 & Ireland & Lost & 0-2 & A \\
Jun 2 & Austria & Won & 1-5 & A \\
Jun 8 & Canada & Won & 0-2 & A \\
\end{tabular}
\end{figure}
Germany qualified as defending champions.
