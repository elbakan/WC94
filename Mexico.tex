\chapter{Mexico}
\chapterprecis{Arturo Tena Colunga}
\newline
\newline
Coach: Miguel Mejia-Baron
\begin{figure}[H]
\begin{tabular}{c c l l c c c}
P & & & Club & Age & Caps & Goals \\ \hline
G & 1 & Jorge Campos & Universidad de Mexico & 27 & 41 & 0 \\
G &12 & Felix Fernandez & Atlante & 27 & 5 & 0 \\
G & 22 & Adrian Chavez & America & 27 & 1 & 0 \\ \hline
D & 2 & Claudio Suarez & Universidad de Mexico & 25 & 33 & 2 \\
D & 3 & Juan Ramirez Perales & Universidad de Mexico & 25 & 36 & 0 \\
D & 4 & Ignacio Ambriz & Necaxa & 29 & 41 & 5 \\
D & 5 & Jesus Ramon Ramirez & Santos Laguna & 24 & 32 & 6 \\
D & 18 & Jos{\'e} Luis Salgado & Autonoma Guadalajara &  28 & 1 & 0 \\
D & 21 & Raul Gutierrez & Atlante & 27 & 18 & 0 \\ \hline
M & 6 & Marcelino Bernal & Toluca & 32 & 12 & 0 \\
M & 8 & Alberto Garcia Aspe & Necaxa & 27 & 26 & 9 \\
M & 13 & Juan Carlos Chavez & Atlas & 27 & 3 & 1 \\
M & 14 & Joaquin del Olmo & Veracruz & 25 & 16 & 1 \\
M & 15 & Michael Espinoza & Guadalajara & 29 & 13 & 0 \\
M & 16 & Luis Valdez & Leon & 28 & 0 & 0 \\
M & 17 & Benjamin Galindo & Guadalajara &  33 & 27 & 8 \\
M & 20 & Jorge Rodriguez &  Toluca &  26 & 21 & 3 \\ \hline
F & 7 & Carlos Hermosillo &Cruz Azul & 29 & 6 & 2 \\
F & 9 & Hugo Sanchez & Rayo Vallecano (Spa) & 35 & 18 & 6 \\
F & 10 & Luis Garcia & Atletico Madrid (Spa) & 25 & 12 & 2 \\
F & 11 & Luis Alves ``ZAGUE'' & America & 27 & 21 & 15 \\
F & 19 & Luis Miguel Salvador & Atlante & 26 & 22 & 9 \\ \hline
\end{tabular}
\end{figure}
\section{Preview}
Miguel Mejia-Baron was happily selected as the National Coach. He has the
school of UNAM coaches, who are very aggressive and like to play a controlled
but offensive game.

Mejia-Baron has done a terrific job so far. He put together a very competitive
team in less than 3 months. In the last round of Concacaf, Mexico clearly 
outclassed their rivals, regardless of the loss against El Salvador at 
San Salvador, the first game of the qualifying, where El Salvador played a 
superb defensive game AT HOME resisting all Mexico's attacks after they went 
ahead 1-0. El Salvador took advantage of two defensive errors by Mexico to win 
the game 2-1. They may have won the match, but Mexico played the football.In 
any case, Mexico won all their games in Mexico City, and deservedly won their
ticket to USA94 after not losing a point after the El Salvador match.

Mexico, IMHO, played a very good Copa America. The first game against Colombia
was strange; first, Colombia took advantage of the scenic panic of the Mexican
players in the first half, which they clearly dominated. However, our team 
played a superb second half where they dominated Colombia clearly until the 
blackout. Colombia took advantage of the blackout and thanks to the 
``phantom of Machala’’ they were able to get an invented goal by the referee 
to beat us 2-1. Against Argentina, we played a great game, Argentina was 
outclassed all the way, but their experience allowed them to get a draw 1-1.
We were expecting to beat Bolivia, but as it turned out, they were a tough 
defensive team, and we get a draw 0-0 that Mexicans didn't like at the time. 
After what Bolivia did in La Paz later on in the South American qualifying 
games against Brasil, Uruguay and Ecuador, this result looks better than it did
at the time. In the quarterfinals, Mexico played a great game against a highly
motivated but inexperienced Peruvian side, winning 4-2, and, progressed to
a semi-final against the hosts Ecuador, which Mexico dominated and won 2-0.

The final was against Argentina and the difference between the teams was the 
greater experience of the Argentinians. Simeone and Batistuta played a great
game for them, and although the match was very even, they deserved to win the
Championship because of their tactics and experience.

After this Mexico won the Concacaf Gold Cup played at Mexico City/Dallas with 
a virtual reserve side, beating the USA in the Aztec stadium.
\section{Key Players}
JORGE CAMPOS (Goalkeeper/Forward, UNAM):
This guy is unique, he is an amazing goalkeeper with great reflexes, and with
a terrific technique with his feet. However, Campos can also play as a forward,
and in the 1991-92 season he scored 15 goals in the Mexican league.

JUAN DE DIOS RAMIREZ-PERALES (Sweeper, UNAM):
By far, the best sweeper in the Mexican league.

CLAUDIO SUAREZ (Defender/Midfield, UNAM):
Although he is recovering from a surgery, he is a key element on the team.
Mejia Baron uses him mainly as a stopper, but he plays at UNAM as a right back
and can also play at midfield.

IGNACIO AMBRIZ (Midfield/Defender, Necaxa):
This guy has a terrific shot which makes him a lethal weapon coming from 
defence. He is a good defensive midfielder, but is not as effective as a
stopper, a position which he has appeared at international level.

RAMON RAMIREZ (Midfield/Defender, Santos):
IMHO, here you have the rising Mexican star!!! He is a very skilful player who 
plays in midfield for Santos, but he is used as an attacking left back in the 
National side. He is quick with a powerful shot, and often takes free kicks.

ALBERTO GARCIA-ASPE (Midfield, Necaxa):
A senior player in the team who was discarded by the previous manager, but has
come back into favour.

BENJAMIN GALINDO (Midfield, Guadalajara):
Another senior player who can destroy a team if he's given the space to play. 
He has excellent vision and is a superb passer of the ball and also has the
ability to score free kicks from both sides.

LUIS GARCIA (Forward, Atletico de Madrid(SP)):
A skilful centre forward who will definitely appear in the squad.

LUIS ROBERTO ALVES ``ZAGUE'' (Forward, AMERICA):
Zague is another type of forward, the one who relies on his pace. He has been 
playing very well for the National team.

HUGO SANCHEZ (Forward, Vallecano(SP)):
