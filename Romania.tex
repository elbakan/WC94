\chapter{Romania}
\chapterprecis{Adrian Weissmann}
\newline
\newline
Coach: Anghel Iord\v{a}nescu
\begin{figure}[H]
\begin{tabular}{c c l l c c c}
P & & & Club & Age & Caps & Goals \\ \hline
G & 1 & Florin Prunea & Dinamo Bucharest & 26 & 23 & 0 \\
G & 12 & Bogdan Stelea & Standard Liege (Bel) & 27 & 14 & 0 \\
G & 22 & Stefan Gabriel Preda & Petrolul Ploiesti & 21 & 0 & 0 \\ \hline
D & 2 & Dan Petrescu & Genoa (Ita) & 27 & 30 & 3 \\
D & 3 & Daniel Claudiu Prodan & Steaua Bucharest & 22 & 12 & 0 \\
D & 4 & Miodrag Belodedici & Valencia (Spa) & 30 & 29 & 4 \\
D & 6 & Gheorghe Popescu & PSV Eindhoven (Neth) & 27 & 42 & 2 \\
D & 13 & Tibor Selimes & Cercle Bruges (Bel) & 24 & 9 & 0 \\
D &1 4 & Gheorghe Mihali &  Dinamo Bucharest & 29 & 18 & 0 \\ \hline
M & 5 & Ioan Lupescu & Bayer Leverkusen (Ger) & 25 & 27 & 3 \\
M & 7 & Dorinel Munteanu & Cercle Bruges (Bel) & 26 & 25 & 2 \\
M & 8 & Iulian Chirita & Rapid Bucharest & 27 & 1 & 0 \\
M & 10 & Gheorghe Hagi (c) & Brescia (Ita) & 29 & 82 & 21 \\
M & 15 & Besarab Panduru &  Steaua Bucharest & 24 & 11 & 0 \\
M & 18 & Constantin Gilca & Steaua Bucharest & 22 & 4 & 0 \\
M & 19 & Corneliu Papur\v{a} & Univ. Craiova & 20 & 1 & 0 \\
M & 20 & Ovidiu Stinga & Univ. Craiova & 22 & 9 & 0 \\ \hline
F & 9 & Florin R\v{a}ducioiu & AC Milan (Ita) & 24 & 22 & 10 \\
F & 11 & Ilie Dumitrescu & Steaua Bucharest & 25 & 38 & 11 \\
F & 16 & Ioan Vladoiu & Rapid Bucharest & 26 & 5 & 0 \\
F & 17 & Dinu Moldovan & Dinamo Bucharest & 21 & 2 & 0 \\
F & 21 & Marian Ivan & F.C. Brasov & 25 & 0 & 0 \\ \hline
\end{tabular}
\end{figure}
\section{General Information}
Population: 23.000.000\newline
The Football Federation: founded 1909, affiliated to FIFA  since 1930 and to UEFA since 1955\newline
Number of teams cca. 5800\newline
Registered players: cca. 180.000\newline
\section{Sport in Romania}
Romania has produced some brilliant sport stars such as the gymnast Nadia 
Comaneci or the tennis player Ilie Nastase. In the international arena the
Romanian athletes have produced a lot of good results in sports like 
gymnastics, rowing, track \& field, weightlifting, boxing, wrestling and
fencing. The handball team was 4 times World Champion and the volleyball team
have won the European title. There is no doubt that football is the most 
popular sport in Romania with average attendances at matches amongst the 
highest in Europe.
\section{Romanian Football History}
Football has been played in Romania since 1893, but the first organized teams 
appeared in 1899-1900. The first championship was held in 1910 and the national
side made their debut in 1922, a 1-2 defeat by Yugoslavia in Belgrade. The 
first Romanian Cup Final took place in 1933.

At club level, Steaua Bucharest have been the most successful, winning 15 
championships and adding the European Champions Cup in 1985-86, after  a
penalty shoot-out with Barcelona. After 2 hours of football, the game was a 0-0
stalemate, but although Steaua only scored twice from the spot, Barcelona
failed to score at all, all four attempts being saved by Duckadam, the hero
of that night. Steaua Bucharest also reached the Champions Cup semi finals in
1987-1988 losing 2-0 on aggregate to Benfica, and made the final again in 1989
losing 4-0 to a rampant AC Milan.

Other teams that have done well in European competition are Universitatea 
Craiova who lost on the away goals rule in the semi final of the 1982-1983 
UEFA Cup, beaten by Benfica, and Dinamo Bucharest who have been knocked out of
both the Champions Cup and Cup Winners Cup at the semi final stage by 
Liverpool (1984) and Anderlecht (1990) respectively.

The history of the national side really started when Romania was among only 
four Europeans teams that crossed the Ocean in 1930 to take part in the 
inaugural World Cup in Uruguay. King Carol of Romania proved to be a hero by
not only picking the Romanian team, but ensuring that the players were given 
adequate time off work in order to train for and play in the finals. They beat
Peru 3-1 and lost to the eventual winners, Uruguay by 4 goals to nil.In 1934 
in Italy, the Romanian team was beaten 2-1 by the eventual finalists 
Czechoslovakia after leading 1-0, and in 1938 Romania were surprisingly 
eliminated by the Cuban team, losing 2-1 after drawing 3-3.

Romania therefore took part in all 3 editions of the World Cup played before 
WW II, but after the war the country had to wait until 1970 before their team
qulified again. A very young team, coached  by  Angelo  Niculescu, succeeded in
January 1969 to draw 1-1 with the World Champions, England at Wembley and soon 
afterwards they won their qualification group ahead of Greece (2-2 in Greece 
and 1-1 in Bucharest in the decisive game), Switzerland (2-0 at home and 1-0
away) and Eusebio's Portugal (0-3 in Lisbon and 1-0 in Bucharest). In Mexico 
they played in the toughest group together with the reigning Champions 
(England), the eventual winners (Brazil) and Czechoslovakia, who had appeared
in the final 8 years earlier. After losing 1-0 to England, they beat 
Czechoslovakia 2-1 with goals from Neagu and Dumitrache, and lost to Pel{\'e}'s 
Brazil 2-3 (Dumitrache and Dembrovschi, the scorers). They failed to qualify 
for the second phase and their coach was criticized by the Romanian media for
the team's lack of courage against England and the non selection of the 
nation's most gifted player, Dobrin (a kind of Gazza of Romanian football).
The best Romanian players in the tournament were Ion Dumitru in midfield, the
sweeper Cornel Dinu and Emeric Dembrovschi in attack. The same remarkable team
won their group in the European Championships ahead of Czechoslovakia, Wales 
and Finland and were very close to the semi-finals in 1972, after a dramatic
three game series with Hungary (1-1 in Budapest, 2-2 in Bucharest and 1-2 in
the play-off in Belgrade). The  Hungarians scored the decisive goal in the 88th
minute of the third game.

After 1972 a long string of disappointments were to follow, with Romania 
failing to qualify for the 1974 World Cup finishing second behind East Germany,
and repeating this 4 years later by losing an incredible match against
Yogoslavia in Bucharest 6-4. This qualification had started promisingly for
Romania as they beat Spain 1-0 in Bucharest and had a sensational 2-0 victory 
in Yogoslavia, but Romania had to settle for the runners up spot again. The 
qualification for the World Cup in 1982 was lost in a similar way as Romania
took 3 points from England (2-1 in Bucharest and 0-0 at Wembley), but couldn't
beat Hungary at home and suffered a shocking home defeat, 2-1 to Switzerland.
Unsurprisingly, the coaches were dismissed and the young trainer Mircea 
Lucescu, the captain of the 1970 team was appointed. He was only 36 and was 
still an active player at his club, Corvinul Hunedoara. This is a unique case 
in football history - an active player also operating as national team coach.  

Nobody gave the Romanian team much chance in the 1984 European Championship
qualifying as they were in a group with Italy (reigning World Champions),
Sweden, Czechoslovakia (1976 European Champions) and Cyprus. However, Romania 
qualified after conceding only 3 goals in 8 games. Unfortunately, the team was
far from impressive in the finals in France where they lost to West Germany 
and Portugal and drew with Spain. Their performance in qualification though was
enough for them to be favourites to qualify for the 1986 World Cup.

The team again played very well against England (0-0 in Bucharest and 1-1 at 
Wembley), and had few problems with Finland and Turkey, but qualification was
lost when they soemhow managed to lose all four points to Northern Ireland. In
Belfast, Northern Ireland were victorious 3-2, but the crucial game was in 
Bucharest where a draw would have been enough to take the Romanian side 
through. However, the game was lost 1-0 and the only hope was that England 
would beat the Northern Irish at Wembley, but the English offered the 
qualification to Northern Ireland by drawing 0-0. History repeated itself as
Romania had got into a good position to qualify before losing a crucial match
on home soil.

Seven months later came Romanian soccer's big triumph, Steaua Bucharest's
Champions Cup victory and their coach, Emeric Jenei replaced Mircea Lucescu 
as head of the national team, bringing with him the offensive spirit of Steaua.
This was reflected in the 1988 European Championship qualifying when Romania 
just failed to qualify after winning all their home matches impressively, 4-0 
against Austria, 5-1 against Albania and 3-1 against Spain (the group winners).
After this, the long string of failures in World Cup qualifiers came to an end,
the crucial games being against Denmark in the autumn of 1989. In Copenhagen 
the Danish side won 3-0 and when Povlsen scored in Bucharest after only 5 
minutes it looked like another disaster for the Romanian squad, but this time
they managed to score three times winning 3-1 and qualifying for the first time
in 20 years. The coach Jenei was praised for picking 26 year old striker 
Gavrila Pele Balint, who scored twice in his first international.

In Italy the team began by beating USSR 2-0 with Lacatus scoring twice, but 
then lost to the tournament's surprise team, Cameroon 1-2, Balint scoring. The 
final group game was against World Champions Argentina and Balint scored again
in a 1-1 draw. This put Romania through to the second round as group runners up
behind Cameroon. The second round game was against Jack Charlton's Ireland and
after a scoreless game, Ireland won 5-4 on penalties. As in 1970 the Romanian
fans had the feeling that the team was capable of doing better.

The falling of the communist regime opened the country's gates for the football
players too and most of the stars that played in the World Cup finals moved
abroad: Gheorghe Hagi to Real Madrid, Gheorghe Popescu to PSV Eindhoven, Marius
Lacatus to Fiorentina, Ionut Lupescu to Bayer Leverkusen, and the young Florin
Raducioiu to Bari. This created a new situation for the national side as 
players ceased to be available two weeks before every important game - this 
created a problem for the coaches: select the overseas based players who would
not have the benefit of a fortnight with the national side before a game or 
reward those who had stayed in Romania.

The new manager that was appointed after the World Cup, Gheorghe Constantin, 
paid a big price for not knowing how to solve these problems: he was forced to 
resign after a disastrous start in the European Championship, particularly a 
humiliating 0-3 home defeat by Bulgaria. The new coach, Mircea Radulescu 
managed some good results, but in the last game of the qualifiers Romania
required a 2-1 away win over Bulgaria,but only managed to draw 1-1 (Hagi missed
a penalty kick). Mircea Radulescu resigned a few months after that and was
replaced by the controversial Cornel Dinu.

\section{Qualification to USA 1994}
In the spring of 1992 Romania started the campaign with an easy 7-0 (Balint 3,
Hagi, Lacatus, Lupescu, Pana) victory in Bucharest against the Faroe Islands. 
The second game was also played in Bucharest and the first half ended with the
Romanians holding an incredible 5-0 lead against Wales! The game ended 5-1 
(Hagi 2, Lupescu 2, Balint) greatly boosting Romanian hopes. In the autumn,
Romania visited Brussels and despite having a number of good scoring 
opportunities, they lost 0-1. The same lack of effectiveness lead to a 1-1 draw
at home to the Republic of Czechs and Slovaks (RCS), Ilie Dumitrescu scored 
after the break, but the visitors equalized with a penalty in the dying minutes
of the game. A 4-1 win in Cyprus ended the year (G. Popescu, Raducioiu, Hagi 
and Hanganu the scorers).

In April 1993 they beat Cyprus 2-1 in Bucharest with Steaua's Ilie Dumitrescu 
scoring both goals and in June the team suffered a disaster playing the Czech
and Slovak team in Kosice. Raducioiu equalized RCS goals twice, but despite 
the expulsion of 2 RCS players, Romania went on to lose 2-5. A 2-5 defeat was 
hard to swallow and led to the end of the international career of the veteran 
goalkeeper Silviu Lung and the dismissal by the Romanian Federation of coach
Cornel Dinu, who was replaced by Anghel (Puiu) Iordanescu who had coached
Steaua Bucharest to their most recent Romanian title. Iordanescu was aware that
he had a very tough task ahead.

After winning in 4-0 in The Faroes, all goals scored by the new AC Milan 
striker, Raducioiu, Romania entertained Belgium in Bucharest. The Belgians 
needed a draw to assure their place in the USA finals, but the Romanians won 
2-1 to keep their own hopes very much alive. The first goal from a Raducioiu
penalty and the same player assisted Ilie Dumitrescu in scoring the second. 
The Belgians narrowed the margin with an 89th minute penalty, but could not 
prevent Romanian victory.

Now, everything would be decided for the Romanian side in the last game, 
against Wales in Cardiff. The game produced great drama with both teams needing
victory for qualification. Romania controlled the game for virtually the whole
of the first half, but only held a 1-0 lead at the break, a drive from Hagi in
the 32nd minute. Wales equalised through Dean Saunders in the 62nd minute and a
minute later the hosts were awarded a penalty. Peter Bodin stepped up to take 
it but clearly couldn't handle the pressure as he blasted the ball against the
bar. After this disappointment for the Welsh, the Romanians came back strongly,
finally securing victory in the 83rd minute through Raducioiu. The Welsh might
claim bad luck, but on the night Romania deserved their victory. So, Romania 
qualified to the World Cup for the second time in succession, their 29 goals
in qualifying sharing the record with the Dutch.

\section{Preparation}
\begin{figure}[H]
\begin{tabular}{l c r c l}
Hong Kong &13 Feb 94 &  Romania &  2-1  & USA \\
South Korea & 16 Feb 94 &  South Korea & 1-2 & Romania \\
Belfast & 23 Mar 94 &  Northern Ireland & 2-0 & Romania \\
Bucharest & 20 Apr 94 & Romania & 3-0 & Bolivia \\
Bucharest & 25 May 94 &  Romania &  2-0 & Nigeria \\
Bucharest & 1 Jun 94 &  Romania &  0-0 & Slovenia \\
\end{tabular}
\end{figure}

\section{The coaches}
Anghel (Puiu) Iord\v{a}nescu was born in 1950, in Bucharest. Since the age of 12 he
played for Steaua Bucharest, making his first team debut at the age of 18. He 
was considered to be one of the most gifted players of his generation, but 
injuries delayed his debut in the national squad until 1971. He was capped 64
times in the national squad scoring 26 goals (only one player in the whole
history of the squad has scored more). With Steaua Bucharest he was 2 times 
champion and won the Cup 4 times. In 1982 he went to Greece and played until 
1984 for OFI Creta.  On his return to Romania, he was appointed as the 
assistant coach of Steaua. Together with the main coach, Emeric Jenei, 
Iordanescu led Steaua to the great successes of 1984-1986 that culminated with
the  winning  of  the European Cup. On the day of the final with Barcelona,  
Iordanescu was 36 and everybody was surprised to see him enter as a substitute.
He had not played an official game since 1984 and was only supposed to play 
for a maximum of 15 minutes,  but he played the whole of extra time and showed
again, for the last time, his wonderful skills as a player. In the autumn of 
the same year, when Jenei was appointed as manager of the national team, 
Iordanescu became Steaua's main coach and some excellent results were to come
(including the winning of the European Super Cup, reaching the semifinals of 
the Champions Cup in 1988 and the final one year later).

In 1990 he went to Cyprus spending 2 years as coach of a local club, before 
coming back to coach his old team, Steaua. Their archrivals Dinamo seemed to be
much stronger, but Iordanescu succeeded again in winning the National title.  
Last summer he was appointed manager of the national side and again did a  
splendid job by winning all the autumn games including the decisive games with
Belgium and Wales.

Iordanescu is assisted by Dumitru Dumitriu, also a former player and coach for
Steaua.

\section{Key Players}
GHEORGHE HAGI (midfield, Brescia(ITA)):
29 year old team captain. One of the greatest Romanian soccer stars of all 
time. An offensive midfielder with brilliant technique and a fearsome shot.

GHEORGHE POPESCU (sweeper, PSV Eindhoven(NETH)):
26 year old sweeper or defensive midfielder who is very strong in the tackle.

ILIE DUMITRESCU (midfield, Steaua Bucharest):
The Romanian rising star who can also play in attack, he has excellent 
technique, is very pacey and scores regularly for the national side.

MIODRAG BELODEDICI (sweeper, Valencia(SP)):
A Beckenbauer-style player who became the first ever player to win the 
European Cup with 2 different teams (Steaua in 1986 and Red Star in 1991).
 
FLORIN ``RADU’’ R\v{A}DUCIOIU (striker, AC Milan(ITA)):
The 24 year old scored 8 goals in the last 4 qualifying games and is a very 
exciting talent.

DAN PETRESCU (right back, Genoa(ITA)):
Very consistent 27 year old full back who gets forward regularly and is capable
of scoring.

IONUTZ LUPESCU (midfield, Bayer Leverkusen(GER)):
Highly skilled 25 year old who's father Nicolae Lupescu played in the World Cup
finals of 1970.

ION VLADOIU (striker, Rapid Bucharest):
A short powerful striker who resembles Gerd Muller in the way he runs, shoots
and looks. A very aggressive striker who is capable of winning the ball from
defenders and using it to his advantage. Likely to start as a substitute, but
watch out for him when he comes on.
