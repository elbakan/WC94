\chapter{Preview of Group B}
\chapterprecis{Simon Gleave}
\newline
\newline
\begin{figure}[H]
\small
\begin{tabular}{l c c c c c c c l}
Group B & Played & Won & Drawn & Lost & For & Agst & Apps & Best Performance\\ \hline
Brazil & 65 & 42 & 12 & 11 & 158 & 68 & 14 & Winners ('58,'62,'70) \\
Cameroon & 8 & 3 & 3 & 2 & 8 & 10 & 2 & Quarter-Final (1990) \\
Russia (USSR) & 31 & 15 & 6 & 10 & 53 & 34 & Fourth (1966) \\
Sweden & 31 & 11 & 6 & 14 & 37 & 34 & 8 & Runners-Up (1958) \\ \hline
\end{tabular}
\normalsize
\end{figure}
Everybody is describing Group E as the ``Group of Death'' and forgetting about
Group B which contains the favourites, Brazil, the 1990 quarter finalists, 
Cameroon, the 1988 Euro Championship finalists, Russia and the 1992 Euro
Championship semi-finalists, Sweden. Russia and Sweden in particular will want 
to make up for their disappointing performance in Italia '90 when they both 
exited at this group stage.

It's difficult to see past Brazil for group winners, although it is nearly 25
years since they actually won the trophy and I do have to wonder whether their
reputation has more to do with triumphs in the distant past rather than 
anything the team has done recently. However, having said that, I can't see 
any of the others troubling them.

Cameroon thrilled us all in 1990 and I'm sure nobody will forget Francois 
Oman-Biyik's header against Argentina in Italy when the Africans defeated the
World Cup holders in the opening game. Much of that team is still playing 
including the veteran Roger Milla who is officially 42 years old, but is 
rumoured to be 46! I can't see them performing as well in this tournament,
although they are well capable of causing a few ripples.

Russia have got a lot of problems at the moment with a number of top players
including Manchester United's Andrei Kanchelskis refusing to play under the 
present coach. They are not the force they once were anyway and only qualified 
in second place behind Greece for this year's tournament. I'm afraid that 
another disappointment is on the cards for the Russians this time.

Sweden played some appalling stuff in 1990 when losing to both Scotland and
Costa Rica - the team was fairly young and now, with four years more 
experience, I would expect them to fulfill the potential that we all thought 
they had four years ago. Tomas Brolin has been playing in Italy ever since the 
1990 capitulation and the Swedes have uncovered an excellent striker in Martin 
Dahlin, expect them to be runners-up to Brazil.

In summary, Brazil to emerge victorious with Sweden in second place and 
Cameroon sweating in third. Russia to finish bottom for the second World Cup
finals running.
