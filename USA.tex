\chapter{United States of America}
\chapterprecis{Christopher Allen, David Akkarach, Andy Busa \& Marcus Lindroos}
\newline
\newline
\Coach: Bora Milutinovic
\begin{figure}[H]
\begin{tabular}{c c l l c c c}
P & & & Club & Age & Caps & Goals \\ \hline
G & 1 & Tony Meola & USSF & 25 & 84 & 0 \\
G & 12 & Jurgen Sommer & Luton Town (Eng) & 23 & 0 & 0 \\
G & 18 & Brad Friedel & Newcastle United (Eng) & 23 & 25 & 0 \\ \hline
D & 2 & Mike Lapper & USSF & 23 & 39 & 1 \\
D & 4 & Cle Kooiman & Cruz Azul (Mex) & 30 & 10 & 1 \\
D & 17 & Marcelo Balboa & USSF & 26 & 90 & 10 \\
D & 20 & Paul Caligiuri & USSF & 30 & 82 & 4 \\
D & 21 & Fernando Clavijo & USSF & 37 & 58 & 0 \\
D & 22 & Alexei Lalas & USSF & 24 & 45 & 5 \\ \hline
M & 3 & Mike Burns & USSF & 23 & 17 & 0 \\
M & 5 &Thomas Dooley & USSF & 33 & 38 & 4 \\
M & 6 & John Harkes & Derby County (Eng) & 27 & 49 & 4 \\
M & 7 & Hugo Perez & USSF & 30 & 76 & 16 \\
M & 9 & Tab Ramos & Real Betis (Spa) & 27 & 49 & 3 \\
M & 13 & Cobi Jones & USSF & 24 & 49 & 5 \\
M & 15 & Joe-Max Moore & USSF & 23 & 33 & 9 \\
M & 16 & Mike Sorber & USSF & 23 & 35 & 2 \\
M & 19 & Claudio Reyna & USSF & 20 & 16 & 2 \\ \hline
F & 8 & Ernie Stewart & Willem II Tilburg (Neth) & 25 & 16 & 2 \\
F & 10 & Roy Wegerle & Coventry City (Eng) & 30 & 16 & 1 \\
F & 11 & Eric Wynalda & 1. FC Saarbruecken (Ger) & 25 & 52 & 14 \\
F & 14 & Frank Klopas & USSF & 27 & 30 & 9 \\ \hline
\end{tabular}
\end{figure}
The key to how well the USA will perform in the World Cup finals lies with the
question of how well our ``Europeans'' integrate with our ``domestics'' when the 
European leagues finish on May 7. Right now there is a little bit of tension 
between some of the ``domestics'' and Roy Wegerle who, somewhat foolishly said 
that the ``European'' were ``better'' then the ``domestics''. While this is clearly
true for the most part, he didn't need to be quoted in \textit{Soccer America} 
actually saying it! For the moment, he has a knee injury (playing with Coventry
City), but will be back well in time for the finals. Given the insane English
schedule, this is actually a blessing in disguise.

The other ``Europeans'' that will be major players for the USA will be Harkes 
(Derby County), Ramos (Real Betis) in midfield, Stewart (Willem II) and Wynalda
(1. FC Saarbruecken) as forwards, and hopefully Keller (Millwall) as 
goalkeeper.

As for the ``domestics'', Dooley has been wonderful in giving the younger players
experience and perspective. In the past, the US would never be able to come 
back and defeat or draw with top European sides in friendlies or otherwise,
with goals in the last 5 minutes of three matches. If you have seen my posts, 
they are 1-2-0 vs Norway, Switzerland and Russia this year. I know that these 
are only friendlies, but this kind of ``confidence building'' has done wonders
for the younger players such as Cobi Jones, Alexi Lalas, Mike Burns, and even 
``domestic veterans'' like Marcelo Balboa, Desmond Armstrong, Tony Meola, and 
Paul Caliguri. In the Russia match they outshot Russia 18-3 and had the 
possession advantage of about 65-35. I'd never seen  them control the ball 
like that against a European team in the past with the exception of the 3-1 
defeat of Ireland in Washington in 1992, when all the ``Europeans'' were present.

Lastly, there is the coach, Bora Milutinovic. He has taken Mexico and Costa 
Rica much further than anyone would have expected in the 1986 and 1990 World 
Cups, and he now has the US players confident that they can play THEIR style 
and not just respond to other teams. To be sure, Colombia, Switzerland and 
Romania are tough opponents, but I think that the second round is a reasonable
possibility. I just took a look at a tape from Italia '90 when the US lost to 
Italy 1-0. The difference between the level of play then and now is amazing.

Tactically, this is a very competitive team based on the players' tactical
ability; we have only two players who are weak in the tactical aspects of the 
game at international level: Armstrong and Lalas. Everyone else on the team is 
at least up to scratch internationally.

Physically, there are no problems at the back, particularly with Caligiuri, who
is a player not to ever be underestimated despite his small stature (5’8’’). 
Burns (5’8’’/5’9’’) is not a much of a physical presence either but is a very 
good player (IMO, he's better than Harkes was 4 years ago) because of his 
tactical sophistication for an American born player.

In the midfield, we have two short players in Jones and Ramos. Jones isn't as 
effective against bigger teams when compared to playing against Latin teams. 
But, if they do underestimate his ability, he'll certainly wreak havoc on the 
defense with far side runs and crosses. This was evident in the matches against
Norway and Switzerland (near the end of the match). Ramos can play against just
about anybody, I remember him against Spain where he consistently beat bigger 
and stronger Spaniards with his quick transitional moves.

Mentally: How bad do these guys want to win? We know that Meola et al from the 
WC90 team have experienced the atmosphere. I hope they can help prepare the 
others to know what expect when playing in the World Cup. If not, I'm sure 
Milutinovic can, since he's been successful doing it with all his past teams. 
Lastly, there is the potential for turmoil in the team, namely the conflicts 
arising from Meola in the back and Wynalda up front. Let's hope that they 
behave like professionals and not turn the players against each other.

\section{Key Players}

THOMAS DOOLEY (Defence/Midfield, 1. FC Kaiserslautern(GER)):

Height/Weight: 6'1" (185cm) /168 lbs. (76kg)
Born: 12 May 1961 / Bechhofen, Germany
Hometown: Mission Viejo, CA USA \& Bechofen, Germany
US Debut: 30 May 1992 vs Ireland
Caps: 30

Has performed in a superior fashion in all major internationals since his debut
in the US Cup in 1992. Won the 1993 award as ``Honda Player of the Year'' in the 
USA. His eligibility for the US -- his father was a US serviceman stationed in 
Germany in the early 1960s -- had been discovered shortly before the 1992 US 
Cup. He scored the first goal in the USA's historic defeat of England in the 
1993 US Cup, and scored two goals against Germany later in the tournament. Has 
tremendous versatility after 9 years with Kaiserslautern (with a league 
Championship and a DFB Cup Championship during his time with the club). Has 
played defender and midfielder for the USA, but could be the starting American 
sweeper in USA '94.


JOHN HARKES (Defence/Midfield, Derby County(ENG)):

Height/Weight: 5'11" (180cm) / 165 lbs. (75kg)
Born: 8 March 1967 / Kearney, NJ USA
Hometown: Kearney, NJ \& Derby, England
US Debut: 23 May 1987 vs Canada
Caps: 52

Has played the past three years in England, 2 years with Sheffield Wednesday, 
and the last one with Derby County. Has been the most successful American in 
England: English Second Division Champion (1991); League Cup (1991); League Cup
Runner up (1993); FA Cup Runner up (1993); and the first ``Yank'' to score a goal
at Wembley in a professional match. Played for the US Olympic team in 1988 and 
in the 1990 World Cup in Italy. Has played all midfield positions in England
and for the USA, but most likely will play as a right back or ``wingback'' in 
USA '94.


TAB RAMOS (Midfield, Real Betis(SP)):

Height/Weight: 5'7" (170cm)/ 140 lbs. (63.5kg)
Born: 21 September 1966 / Montevideo, Uruguay
Hometown: Hillside, NJ USA \& Valencia Spain
US Debut: 10 January 1988 vs Guatamala
Caps: 50

Moved to USA with his family as a young child and grew up in the US soccer 
community of Kearney, NJ (with Harkes and keeper Tony Meola). Among the most 
skilful US products and an exceptionally talented playmaker, Ramos is a 
midfielder for Real Betis, a club likely to be promoted to the first divison 
for 1994-95. Scored winning goal vs Ireland in the first US Cup game in 1992 
and his cross to Dooley set up the first goal vs England in 1993 US Cup. 
Started his international career at the age of 16 at the 1983 U-20 World 
Championship, also played in the 1988 Olympics and in Italia '90.

ROY WEGERLE - (Forward/Midfield, Coventry City(ENG)):

Height/Weight: 5'11" (180cm)/ 170 lbs. (77kg)
Born: 19 March 1964 / Pretoria, South Africa
Hometown: Tampa, FL USA \& Coventry England
US Debut: 30 May 1992 vs Ireland
Caps: 14

Set up a goal in his US debut, an impressive 3-1 win, though he had never 
practiced with the team. Joined Coventry City after stints with English clubs 
Blackburn Rovers, QPR, Luton Town and Chelsea. In the US, played for the 
University of South Florida, Tampa Bay Rowdies and Tacoma Stars. Brother of 
former English league star, Steve Wegerle. Could possibly have played with at
least 4 other national sides: South Africa (birthplace), Germany (father's 
birthplace), Scotland (mother's ancestral home), England (established 5 year 
residency), but chose to play with the USA, since his marriage to an American 
woman whom he met while in college in the USA. He is a creative forward, but he
generally plays as a ``withdrawn'' forward in Bora Milutinovic's 4-5-1 or 5-4-1 
formations. Has had knee surgery twice in the past 6 months, but is expected to
be ready by June 18.


ERNIE STEWART (Forward, Willem II Tilburg(NETH)):

Height/Weight: 5'9" (175cm)/ 145 lbs. (65kg)
Born: 28 March 1969 / Veghel, Netherlands
Hometown: Uden, Netherlands
US Debut: 19 December 1990
Caps: 13

A speedy front-runner with a nose for the goal. The son of an American father 
and a Dutch mother, moved to the Dutch first division after scoring 14 goals 
for VVV Venlo in the 2nd division in the 1989-90 season. Scored 17 goals for 
Willem II in 1990-91 and scored a goal vs Germany in the 1993 US Cup in 
Chicago. Lived in the USA from the age of 2 to the age of 7.


ERIC WYNALDA (Forward/Midfield, 1. FC Saarbruecken(GER)):

Height/Weight: 5'11" (180cm)/ 170 lbs. (77kg)
Born: 9 June 1969 / Fullerton, CA USA
Hometown: West Lake Village, CA USA \& Saarbruecken, Germany
US Debut: 2 February 1990 vs Costa Rica
Caps: 50

Exploded after joining Saarbruecken for the 92-93 season, scoring 9 goals in 
first 10 games. Led the US in scoring in 1992 with 5 goals in 7 games and was 
named ``Honda Player of the Year''. Played in the US College (NCAA) finals in 
1987 with San Diego State College. Noted for dribbling at high speed, and makes
effective diagonal runs. During 93-94 season has scored 12 goals for
Saarbruecken playing striker, ``withdrawn'' forward, and midfielder.
