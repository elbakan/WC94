\chapter{Greece}
\chapterprecis{Constantinos Tsioras}
\newline
\newline
Coach: Alketas Panagoulias
\begin{figure}[H]
\begin{tabular}{c c l l c c c}
P & & & Club & Age & Caps & Goals \\ \hline
G & 1 & Antonios Minou & Apollon & 35 & 15 & 0 \\
G & 15 & Khristos Karkamanis & Aris Salonika & 24 & 5 & 0 \\
G & 20 & Ilias Atmatsidis & AEK Athens &  24 & 3 & 0 \\ \hline
D &  2 & Stratos Apostolakis & Panathinaikos & 29 & 59 & 3 \\
D &  3 & Thanasis Kolitsidakis & Panathinaikos & 27 & 12 & 0 \\
D &  4 & Stelios Manolas & AEK Athens & 32 & 68 & 6 \\
D &  5 & Yiannis Kalitsakis & Panathinaikos & 27 & 34 & 0 \\
D & 13 & Vaios Karagiannis & AEK Athens & 27 & 6 & 0 \\
D & 18 & Kyriakos Karataidis & Olympiakos & 28 & 14 & 0 \\
D & 20 & Aleksis Alexiou & PAOK Salonika & 31 & 8 & 0 \\ \hline
M &  6 & Yiannis Tsaloukhidis & Olympiakos & 31 & 59 & 11 \\
M &  8 & Nikos Nioplias & Panathinaikos & 29 & 36 & 1 \\
M & 10 & Thasos Mitropoulos (c) & AEK Athens & 36 & 72 & 8 \\
M & 11 & Nikos Tsiantakis & Olympiakos & 30 & 45 & 2 \\
M & 12 & Spyridon Marangos & Panathinaikos & 27 & 10 & 0 \\
M & 17 & Minas Hatsidis & Olympiakos & 29 & 6 & 1 \\
M & 19 & Savas Kofidis & Aris Salonika & 32 & 62 & 1 \\
M & 21 & Aleksandros Aleksandris & Olympiakos Pireaus & 26 & 9 & 0 \\ \hline
F &  7 & Dimitris Saravakos & Panathinaikos & 32 & 74 & 21 \\
F &  9 & Nikos Makhlas & OFI Crete & 20 & 11 & 5 \\
F & 14 & Vasilis Dimitriadis & AEK Athens & 28 & 26 & 2 \\
F & 16 & Aleksandros Aleksoudis & OFI Heraklion & 20 & 3 & 1 \\ \hline
\end{tabular}
\end{figure}
\section{Preview}
Greek football has been around for a while - Panathinaikos FC, for example, 
were founded in 1908 (they wear green and their emblem is the shamrock - an 
Irish connection perhaps?). However the achievements of the national team do 
not match the long history and the passion with which football is followed in 
Greece.

The main reason for the national team's failure to impress over the years had a
lot to do with the average Greek national team player attitude. There were long
periods where the national team was met mostly with indifference by players 
who were more concerned with not injuring themselves on a Wednesday, with 
Sunday's club match coming up.

There were a few near misses over the years, as far as qualification to the 
World Cup and the European Championship finals was concerned, but mediocrity 
was generally the norm. The only bright spot was achieving qualification to the
1980 EC Finals in Italy. In the finals, Greece lost to Holland (0-1 after 
goalkeeper Constantinou gave away a stupid penalty with a few minutes to go), 
lost to the then champions Czechoslovakia (1-3), but then gave the eventual 
champions Germany a fright by holding them to a goalless draw and hitting the 
bar with just two minutes remaining.

The 1980 team was lead by coach Alketas Panagoulias, who returned as coach in
1992 and connected his name again with the second important achievement in 
Greece's football history: qualification to the USA 1994 World Cup. What was 
the key to this recent success? Money was important as coach Panagoulias 
convinced the Greek FA and the Sub-Ministry of Sports that if the national 
team were to progress large financial rewards should be offered to the players.
The authorities listened, the bonuses increased by an order of magnitude, and,
surprise-surprise, so did the players interest and the team's performance. 
After qualification was achieved, the players and FA squabbled for quite a 
while regarding bonuses and perks, but hopefully its now been resolved.

Panagoulias is a good motivator (waving wads of drachmas greatly increased 
this ability of his) and a good tactician that gets the best out of rather 
average players. The coach is definitely Greece's main asset in the upcoming 
WC finals.
\section{WC 1994 Qualifying results}
\begin{figure}[H]
\begin{tabular}{l l l l}
13.05.92 & Greece-Iceland & 1-0 & (Sofianopoulos) \\
07.10.92 & Iceland-Greece & 0-1 & (Tsalouhides) \\
11.11.92 & Greece-Hungary & 0-0 & \\
17.02.93 & Greece-Luxembourg & 2-0 & (Dimitriades [pen], Mitropoulos) \\
31.02.93 & Hungary-Greece & 0-1 & (Apostolakis [pen]) \\
23.05.93 & Russia-Greece & 1-1 & (Mitropoulos) \\
12.10.93 & Luxembourg-Greece & 1-3 & (Mahlas, Apostolakis, Saravakos) \\
17.11.93 &  Greece-Russia & 1-0 & (Mahlas) \\
\end{tabular}
\end{figure}
\begin{figure}[H]
\begin{tabular} {l c c c c r c l c}
\textbf{Greece} & 8 & 6 & 2 & 0 & 10 & - & 2 & 14 \\
\textbf{Russia} & 8 & 5 & 2 & 1 & 15 & - & 4 & 12 \\
Iceland & 8 & 3 & 2 & 3 & 7 & - & 6 & 8 \\
Hungary & 8 & 2 & 1 & 5 & 6 & - & 11 & 5 \\
Luxembourg & 8 & 0 & 1 & 7 & 2 & - & 17 & 1 \\
\end{tabular}
\end{figure}
\section{Preparation}
\begin{figure}[H]
\begin{tabular} {l l l}
23.03.94 &Greece-Poland & 0-0 \\
27.04.94 & Greece-Saudi Arabia & 5-1 \\
9.05.94 & Greece-Cameroon & 0-3 \\
13.05.94 & Greece-Bolivia & 0-0 \\
17.05.94 & England-Greece & 5-0 \\
28.05.94 & USA-Greece & 1-1 \\
5.06.94 & Colombia-Greece & 2-0 \\
\end{tabular}
\end{figure}
\section{Realistic Analysis}
This is the first WC finals for Greece. The team has difficulty scoring and
playing fluent football. At the same time, the defence does not generally 
concede many goals either. The team plays a tight and sometimes rough game 
allowingthe other team to have  most of the possession, and then attempts to
score on the break, or from a set piece (there are quite a few tall Greek
players that can score with headers from corner-kicks or free-kicks). The 
biggest disadvantage for Greece is that the team loses concentration every
so often. On the other hand, if the game 'suits' the team (\textit{i.e.} Greece don't
have to do most of the attacking) then they feel more confident and comfortable.

In the first game vs. Argentina, if Greece concedes a goal in the first half,
it will be very difficult to come back. Expect Argentina to win 2-0 in this
case. If the score is 0-0 at half-time, expect a 0-0 final score, or Greece 
scoring on a breakaway and then putting 10 men in defense to hold out for a 1-0
win. Against Bulgaria, the game should be a lot more open as the two teams know
each other well and are not afraid of each other. Possibly a 1-1 draw. Against 
Nigeria, it obviously depends on what the teams need from the game. If Greece 
need a win to progress, they will have to go forward and that's not the team's 
style, so expect Nigeria to win 1-0 or 2-1. However, if Greece need a draw to 
advance, I can see a 0-0 or 1-1 scoreline.

In general, Greece is not favoured by the rule change of awarding three points
for a win. I can see Greece getting a couple of draws, but it will be very
difficult to get a win. The only realistic chance for Greece to progress past 
the first round will be to finish third and hope to advance as one of the best 
4 third placed teams. Then again, the saying in Greece is that ``the ball is 
roun'', and you never know which way it will roll. Stranger things have 
happened in football.
\section{Key Players}
STRATOS APOSTOLAKIS (Defender, Panathinaikos): 
Good technique, very attacking-minded and adventurous for a full back. He can 
get carried away and leave his position though.

STELIOS MANOLAS (Sweeper, AEK Athens): 
A veteran international who looks very comfortable on the ball. He reads the 
game well and can be dangerous on set pieces. A great leader who is full of
confidence, but is not as quick as he used to be.

NIKOS NIOPLIAS (Midfield, Panathinaikos): 
A player blessed with good vision and great passing ability. He will run all 
day and can be a fierce tackler. However, he is very inconsistent and can be
very quiet in some games.

NIKOS MAKHLAS (Forward, OFI Crete): 
Young (20 years old) and enthusiastic, he runs hard and hassles defenders. He 
can create and score goals and although he is inexperienced he is definitely a 
future star.
