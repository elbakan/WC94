\chapter{World Cup Trivia}
\chapterprecis{Stephen Halchuk}
\section{First Time Meetings and Rematches}

27 of the 36 first round pairs of opponents in WC94 have never met each other 
in a World Cup final tournament before. Of the 9 games that have been played 
in WC finals before, the longest wait for a rematch is Italy v Norway, who last
met in a WC in 1938 (2:1 in favour of Italy).

Four of the first round matches are grudge matches from Italia 1990 (1990 results in
brackets).

Brazil v Sweden (2:1)
Russia v Cameroon (4:0)
Italy v Ireland (1:0)
Spain v South Korea (3:1)

When Saudi Arabia plays Morocco on June 25, It will mark the first time in a WC
final tournament that an African team has played an Asian team.

\section{Multiple Meetings}

Several pairs of teams appear to meet each other in the finals with alarming 
regularity. The most consistent recent matchup is Italy and Argentina, who have
met each other in the past 5 consecutive WC finals (1974, 1978, 1982, 1986, 1990)! 
Other 5 time matchups are:

Brazil v Spain (1934, 1950, 1962, 1978, 1986)
Brazil v Czechoslovakia (1938, 1938, (a replay was required to decide the first game), 1962 (first round), 1962 (final match), 1970)

Brazil and Sweden will join this group of 5 time matchups after their first 
round game in Detroit, having already met in 1950, 1958, 1978, and 1990.

Other possible 5 time meetings after this summer's finals are:
Germany v Argentina (1958, 1966, 1986, 1990)
Germany v Italy (1962, 1970, 1978, 1982)
Italy v Brazil (1938, 1970, 1978, 1982)

No two teams, to my knowledge, have met each other more than 5 times in the 14
WC finals.

\section{Past Appearances and Performances of the 24 Qualified Nations}
PAST APPEARANCES AND PERFORMANCES OF THE 24 QUALIFIED NATIONS:

\begin{figure}[H]
\begin{tabular}{c l c l c l}
14 & Brazil & 7 & Russia & 2 & Cameroon \\
12 & Germany  &  6 & Switzerland & 2 & Colombia  \\         
11 & Italy & 5 & Bulgaria & 2 & Morocco  \\
10 & Argentina  & 5 & Netherlands & 1 & Norway  \\
 9 & Mexico  & 5 & Romania & 1 & Ireland \\
 8 & Belgium & 3 & South Korea & 0 & Greece \\    
 8 & Spain & 3 & United States & 0 & Nigeria \\    
 8 & Sweden & 2 & Bolivia & 0 & Saudi Arabia \\
\end{tabular}                                                   
\end{figure}
\begin{figure}[H]
\begin{tabular}{c l c l c l}
& Semi-finals  & & Finalists & & Champions \\                                                                         
9 & Germany & 6  & Germany & 3  & Germany  \\
5 & Brazil & 4  & Brazil & 3  & Brazil \\
4 & Italy & 4  & Italy & 3 & Italy \\
4 & Argentina & 4  & Argentina & 2  & Argentina \\
2 & Netherlands & 2  & Netherlands & & \\        
2 & Sweden & 1  & Sweden & & \\
1 & USA & & & & \\
1 & Belgium & & & & \\
1 & Russia  & & & & \\
\end{tabular}
\end{figure}
Of those who have played in the finals before

Bolivia are looking for their first point, indeed their first goal. They have a
cumulative WC record of 6 played 6 lost, 0 goals for, 16 goals against.

Bulgaria (played 16, drawn 6, lost 10, scored 11, conceded 35),
South Korea (played 8, drawn 1, lost 7, scored 5, conceded 27),
Norway (played 1, lost 1, scored 1, conceded 2) and 
Rep. of Ireland (played 5, drawn 4, lost 1, scored 2, conceded 3) 
are all looking for their first win.

Brazil, Germany, Italy, Argentina, Russia, Spain, Holland, and Cameroon are the
only teams (of those qualified for WC94) who have a cumulative WC finals 
winning record (i.e. more points than games played).
\section{Championship Match Miscellany}
CHAMPIONSHIP MATCH MISCELLANY:

Largest attendance for a WC championship match:
1950 Uruguay 2:1 Brazil (Maracana stadium, Rio de Janeiro, 199000)

Smallest attendance for a WC championship match:
1958 Brazil 5:2 Sweden (Rasunda stadium, Stockholm, 49000)

The Rose Bowl, in Pasadena, host of the championship match in WC94, has a 
capacity of approximately 100000.

In only 3 of the 14 previous World Cups has the eventual winner made it through
the tournament without dropping a point:
1930 Uruguay (4 matches)
1938 Italy (4 matches)
1970 Brazil (6 matches)

In two WC finals, two teams who played each other in the first round met again 
in the final match:

1962 Brazil v Czechoslovakia
1954 West Germany v Hungary

Most goals in a championship match: 7
1958 Brazil 5-2 Sweden

Fewest goals in a championship match: 1
1990 Germany 1-0 Argentina

In all of the other championship matches at least 3 goals were scored.

3 of the 14 championship matches were decided in extra time. No championship 
match has been decided by penalty kicks/replay:

1978 Argentina 3-1 Holland
1966 England 4-2 Germany
1934 Italy 2-1 Czechoslovakia

8 of the 14 teams who scored the first goal in the championship match went on 
to win the game.
\section{Decreased Scores}
The number of goals per game appears to have decreased dramatically over the 
history of the World Cup. One must remember that the game has become more 
defensive over the years and that the gap between the``superpowers'' and the 
``minnows'' has shrunk. Since 1962 the number of goals per game has been 
remarkably consistent, with the unfortunate exception of Italia 90. Let's hope 
this is an aberration rather than the start of a new trend.
\begin{figure}[H]
\begin{tabular}{l c c l}
Year & Goals & Matches & Goals per Match \\ \hline
1930 & 70 & 18 & 3.89 \\
1934 & 70 & 17 & 4.12 \\
1938 & 84 & 18 & 4.67 \\
1950 &  88 & 22 & 4.00 \\
1954 & 140 & 26 & 5.38 \\
1958 & 126 & 35 & 3.60 \\
1962 & 89 & 32 & 2.78 \\
1966 & 89 & 32 & 2.78 \\
1970 & 95 & 32 & 2.97 \\
1974 & 97 & 38 & 2.55 \\
1978 & 102 & 38 & 2.68 \\
1982 & 146 & 52 & 2.81 \\
1986 & 132 & 52 & 2.54 \\
1990 & 115 & 52 & 2.21 \\ \hline
\end{tabular}
\end{figure}
