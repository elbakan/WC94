\chapter{Argentina}
\chapterprecis{Sergio Adeff}
\newline
\newline
Coach: Alfio Basile
\begin{figure}[H]
\begin{tabular}{c c l l c c c}
P & & & Club & Age & Caps & Goals \\ \hline
G & 1 & Sergio Goycochea & River Plate &  30 & 46 & 0 \\
G & 12 & Luis Islas & Independiente & 28 & 26 & 0 \\
G & 22 & Hugo Scopponi & Newell's Old Boys & 33 & 0 & 0 \\ \hline
D & 2 & Sergio Vazquez & Universidad Catolica (Chile) & 28 & 33 & 0 \\
D & 3 & Jos{\'e} Antonio Chamot & Foggia (Ita) & 25 & 8 & 1 \\
D & 4 & Roberto Sensini & Parma (Ita) & 28 & 19 & 0 \\
D & 6 & Oscar Ruggeri & San Lorenzo & 32 & 94 & 9 \\
D & 13 & Fernando Caceres & Real Zaragoza (Spa) & 25 & 9 & 0 \\
D & 15 & Jorge Borelli & Racing Club & 29 & 15 & 0 \\
D & 16 & Hernan Diaz & River Plate & 29 & 17 & 1 \\ \hline
M & 5 & Fernando Redondo & Tenerife (Spa) & 24 & 22 & 1 \\
M & 8 & Jos{\'e} Horacio Basualdo & Velez Sarsfield & 31 & 29 & 0 \\
M & 10 & Diego Maradona (C) & unattached & 33 & 88 & 33 \\
M & 14 & Diego Simeone & Seville (Spa) & 23 & 22 & 4 \\
M & 18 & Hugo Perez & Independiente & 25 & 5 & 1 \\
M & 20 & Leonardo Rodriguez & Borussia Dortmund (Ger) & 27 & 29 & 3 \\
M & 21 & Alejandro Mancuso & Boca Juniors & 26 & 8 & 1 \\ \hline
F & 7 & Claudio Caniggia & AS Roma (Ita) & 27 & 41 & 12 \\
F & 9 & Gabriel Batistuta & Lazio (Ita) & 25 & 32 & 20 \\
F & 11 & Ramon Medina Bello & Yokohama Marinos (Jap) & 27 & 16 & 5 \\
F & 17 & Ariel Ortega & River Plate & 20 & 4 & 0 \\
F & 19 & Abel Balbo & AS Roma (Ita) & 27 & 18 & 5 \\ \hline
\end{tabular}
\end{figure}
\section{Preview}
Argentina appeared in ten previous World Cup finals (1930-34, 1958-66, 1974-90)
having won two of them (1978 and 1986).  Only once (in 1970) Argentina did not
qualify for the WC finals. (the other absences being of its own choosing;
before WWII to protest the purchase of its players by European clubs and after
WWII to protest not being allowed to host the WC.) Argentina is the current
runner up, having lost the WC'90 Final with Germany (0-1) with a doubtful  
penalty kick and finishing the game with nine players.  However, other than 
reaching the final, Argentina's overall performance in WC'90 was disappointing,
as the team played a defensive game that at times seemed too harsh. They won 
two games thanks only to the goalkeeper, Goycoechea, who during and after the 
World Cup displayed an uncanny ability to stop penalty kicks, and their best 
player, Maradona, had a foot injury from the outset and never reached the 
outstanding level of play he exhibited in WC'86 while nevertheless, the team 
revolved around his play.

Argentina attained an excellent level after WC'90 and brilliantly won the 
Chile'91 South America Cup.  Since losing the Final in WC'90, Argentina, having
changed coach from Bilardo (the coach from WC'86 to WC'90) to Basile (current)
produced an extraordinary streak of 33 undefeated games which also included the
first America Cup (Ecuador'93), an expanded version of the traditional South
America Cup. However, Argentina's game was increasingly criticized as reverting
to the defensive, rough play of WC'90, and many were expecting a hard fall, 
which came, not surprisingly against the excellent team of Colombia during the 
qualifiers, in a game played in Colombia (2-1). Basile wanted to fend off 
criticism and recover confidence, thus he ordered an all-out attack for the 
second leg against Colombia played in Buenos Aires.  It ended in a nightmare 
for Argentina, with a 5-0 defeat in the glorious stadium of the WC'78 where 
Argentina had won its first World Cup.  A crisis ensued, and Maradona, who 
wasn't expected to play for Argentina any longer, made a much heralded return 
to the team, and was indeed valuable in securing the tickets to USA'94 
overcoming Australia in two hard-fought games.  A diminished Argentina reached
the finals last in the 24-team count and went into the pre World Cup friendlies
severely criticized, locally and abroad. However, the team produced a 2-1 win 
against a full strength Germany in Miami, in a game were it attacked with 
confidence and outplayed the current champs in their first encounter after the 
'90 Final.

A second friendly was lost 2-0 to Brasil in Recife, Brasil (the first time that
Brasil had beaten Argentina in five years), and a third friendly was won 3-1
against Morocco in Salta, Argentina.  Participation in the Kirik Cup was 
cancelled because Japan denied an entry visa to the team Captain, Maradona (the
team thus refused to go), due to his past cocaine addiction, and another  
friendly was hastily arranged with Chile in Santiago. This friendly ended in a 
3-3 draw that exhibited both Argentina's continuous - in the last few years - 
problems in defense and its new found attacking effectiveness. In fact 
Argentina now have an extraordinary array of attacking players both in the 
midfield line (with Maradona, Redondo, Simeone, Leo Rodriguez), and in attack 
(with Caniggia, Balbo, Batistuta, Medina Bello, Ortega). However, the defence
is in disarray mainly because Basile insists in maintaining a leader who is now
old and slow in Ruggeri (who was an extraordinary player in past WCs,) and some
rather mediocre players (Basualdo and Borelli), while the few with quality and 
youth are isolated and thus unable to reaffirm themselves (Vazquez, Chamot, 
Sensini).

It remains to be seen if Basile will be able to come out with the right 
defence, with the main obstacle being his insistence on Ruggeri. If he does, 
Argentina will become, in the eyes of many, a prime candidate to win WC'94, at 
the level of Brazil and Germany, while one month before the finals it still is 
considered, at best, one of the second group of contenders together with Italy.

\section{Preparation}
\begin{figure}[H]
\begin{tabular}{l lr c l}
23 Mar 94 & Recife, Brazil & Brazil & 2-0 & Argentina \\
20 Apr 94 & Salta, Argentina & Argentina & 3-1 & Morocco \\
18 May 94 & Santiago, Chile & Chile & 3-3 & Argentina \\
31 May 94 & Ramat-Gan, Israel & Israel & 0-3 & Argentina \\
4 Jun 94 & Toronto, Canada & Argentina & 0-0 & Croatia \\ 
\end{tabular}
\end{figure}
